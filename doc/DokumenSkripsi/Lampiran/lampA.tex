\chapter{Kode Sumber Perkakas \textit{Command Line} KIRI}
\label{lamp:A}

%terdapat 2 cara untuk memasukkan kode program
% 1. menggunakan perintah \lstinputlisting (kode program ditempatkan di folder yang sama dengan file ini)
% 2. menggunakan environment lstlisting (kode program dituliskan di dalam file ini)
% Perhatikan contoh yang diberikan!!
%
% untuk keduanya, ada parameter yang harus diisi:
% - language: bahasa dari kode program (pilihan: Java, C, C++, PHP, Matlab, C#, HTML, R, Python, SQL, dll)
% - caption: nama file dari kode program yang akan ditampilkan di dokumen akhir
%
% Perhatian: Abaikan warning tentang textasteriskcentered!!
%

Bagian ini akan memuat seluruh kode sumber (\textit{source code}) perkakas \cl\xspace KIRI, mulai dari tiap-tiap fungsinya, hingga terakhir, keseluruhan kode dalam satu potongan kode.

\begin{lstlisting}[label={appdx:A-printhelp}, language=C, caption=\texttt{print\textunderscore help()}]
void print_help() {
    puts("KIRI Command Line Tool, version 1.2.13");
    puts("Use the KIRI tool through the command line.");
    putchar('\n');
    puts("USAGE:");
    puts("    kiritool <COMMAND> [OPTIONS...] [<ARGUMENTS>]");
    putchar('\n');
    puts("COMMAND:");
    puts("    -h, --help          Display usage tutorial.");
    puts("    -m, --mode <MODE>   Set the tool operation mode.\n");
    puts("    You can only choose one mode per operation.");
    putchar('\n');
    puts("OPTIONS:");
    puts("    -F, --regfinish <REGION>    Set region to search for finish location in.");
    puts("    -f, --finish <FINISH>       Set finish location.");
    puts("    -l, --locale <LANG>         Set tool output language.");
    puts("    -q, --query <KEYWORD>       Set keyword for searching.");
    puts("    -r, --region <REGION>       Set region to search for location in.");
    puts("    -S, --regstart <REGION>     Set region to search for starting location in.");
    puts("    -s, --start <START>         Set starting location.");
    putchar('\n');
    puts("ARGUMENTS:");
    puts("    <FINISH>");
    puts("        Route finish location.");
    puts("        For 'findroute' mode, input the location's latitude and longitude coordinates.");
    puts("        For 'direct' mode, input the location's search keyword (<KEYWORD> argument).");
    puts("    <KEYWORD>");
    puts("        Keywords used by the tool to search for the desired location.");
    puts("        For multiple words queries, encase the keywords in quotation marks (\" \").");
    puts("    <LANG>");
    puts("        Tool output language.");
    puts("        Available languages: id, en");
    puts("          - id (Indonesian - Default)");
    puts("          - en (English)");
    puts("    <MODE>");
    puts("        Tool operation mode. Note that not all options are needed for all modes.");
    puts("        Should the user input extra options, they will simply not be used.");
    puts("        Available modes:");
    puts("          - searchplace");
    puts("                Required options: --query, --region");
    puts("                Optional options: --locale");
    puts("          - findroute");   
    puts("                Required options: --start, --finish");
    puts("                Optional options: --locale");
    puts("          - direct");   
    puts("                Required options: --regstart, --start, --regfinish, --finish");
    puts("                Optional options: --locale");
    puts("    <REGION>");
    puts("        Region to search for locations in.");
    puts("        Available regions:");
    puts("          - cgk (Jakarta)");
    puts("          - bdo (Bandung)");
    puts("          - mlg (Malang)");
    puts("          - sub (Surabaya)");
    puts("    <START>");
    puts("        Route starting location.");
    puts("        For 'findroute' mode, input the location's latitude and longitude coordinates.");
    puts("        For 'direct' mode, input the location's search keyword (<KEYWORD> argument).");
}
\end{lstlisting}

\begin{lstlisting}[label={appdx:A-replacespace}, language=C, caption=\texttt{replace\textunderscore space()}]
void replace_space(char* string) {
    // Escapes all whitespace characters in string
    int i;
    int j = 0;

    for (i = 0; i < 100; i++) {
        if (string[i] == '\0') {
            break;
        }

        if (string[i] == ' ') {
            escape[j] = '%';
            escape[j + 1] = '2';
            escape[j + 2] = '0';
            j = j + 2;
        }
        else escape[j] = string[i];

        j++;
    }
}
\end{lstlisting}

\begin{lstlisting}[label={appdx:A-buildurl-searchplace}, language=C, caption=\texttt{build\textunderscore url\textunderscore searchplace()}]
void build_url_searchplace(int locale, int region, char* query) {
    strcat(URL, "&mode=searchplace");

    // Locale check - tool adition
    if (locale == 2) {
        error = 1;
        fputs("\nError:\n", stderr);
        fputs("Anda telah memasukkan pilihan bahasa (locale) yang tidak valid.\n", stderr);
        fputs("Mohon periksa kembali apakah pilihan bahasa yang anda masukkan valid atau tidak.\n", stderr);
        fputs("Pilihan locale: id, en\n", stderr);
        fputs("--------------------\n", stderr);
        fputs("You have inputted an invalid language (locale) option.\n", stderr);
        fputs("Please recheck whether the language code you inserted was supported or not.\n", stderr);
        fputs("Locale available: id, en\n", stderr);
        exit(1);
    }

    // Region check
    switch (region) {
        case -1:
            error = 1;
            fputs("\nError:\n", stderr);

            if (locale == 1) {
                fputs("Location searching requires a region to be set.\n", stderr);
                fputs("Please make sure you have chosen between one of the four available regions.\n", stderr);
                fputs("Regions available: cgk, bdo, mlg, sub\n", stderr);
            }
            else {
                fputs("Fitur pencarian lokasi memerlukan pengaturan region lokasi yang ingin dicari.\n", stderr);
                fputs("Mohon pastikan anda sudah memasukkan salah satu dari empat kode region yang tersedia.\n", stderr);
                fputs("Pilihan region: cgk, bdo, mlg, sub\n", stderr);
            }
            exit(1);
            break;
        
        case 1:
            strcat(URL, "&region=cgk");
            break;
        
        case 2:
            strcat(URL, "&region=bdo");
            break;
        
        case 3:
            strcat(URL, "&region=mlg");
            break;
        
        case 4:
            strcat(URL, "&region=sub");
            break;

        default:
            error = 1;
            fputs("\nError:\n", stderr);

            if (locale == 1) {
                fputs("You have inputted an invalid region.\n", stderr);
                fputs("Please recheck whether the region code chosen was one of the four region codes supported.\n", stderr);
                fputs("Regions available: cgk, bdo, mlg, sub\n", stderr);
            }
            else {
                fputs("Anda telah memasukkan region yang tidak valid.\n", stderr);
                fputs("Mohon periksa kembali apakah kode region yang anda masukkan merupakan salah satu dari empat kode region yang tersedia.\n", stderr);
                fputs("Pilihan region: cgk, bdo, mlg, sub\n", stderr);
            }
            exit(1);
            break;
    }

    // Query check
    if (strcmp(query, "\0") == 0) {
        error = 1;
        fputs("\nError:\n", stderr);

        if (locale == 1) {
            fputs("Location searching requires a search query.\n", stderr);
            fputs("Please make sure you have inputted a query to be used in the search.\n", stderr);
        }
        else {
            fputs("Fitur pencarian lokasi memerlukan sebuah kata kunci pencarian.\n", stderr);
            fputs("Mohon pastikan anda sudah memasukkan kata kunci untuk melakukan pencarian lokasi.\n", stderr);
        }
        exit(1);
    }
    else {
        strcat(URL, "&querystring=");
        strcat(URL, query);
    }
    strcat(URL, "&apikey=68CD281C8A8EE97C");
}
\end{lstlisting}

\begin{lstlisting}[label={appdx:A-buildurl-findroute}, language=C, caption=\texttt{build\textunderscore url\textunderscore findroute()}]
void build_url_findroute(int locale, char* start, char* finish) {
    strcat(URL, "&mode=findroute");

    // Locale check
    switch (locale) {
        case 1:
            strcat(URL, "&locale=en");
            break;
        
        case 2:
            error = 1;
            fputs("\nError:\n", stderr);
            fputs("Anda telah memasukkan pilihan bahasa (locale) yang tidak valid.\n", stderr);
            fputs("Mohon periksa kembali apakah pilihan bahasa yang anda masukkan valid atau tidak.\n", stderr);
            fputs("Pilihan locale: id, en\n", stderr);
            fputs("--------------------\n", stderr);
            fputs("You have inputted an invalid language (locale) option.\n", stderr);
            fputs("Please recheck whether the language code you inserted was supported or not.\n", stderr);
            fputs("Locale available: id, en\n", stderr);
            exit(1);
            break;
        
        default:
            strcat(URL, "&locale=id");
            break;
    }

    // Starting location check
    if (strcmp(start, "\0") == 0) {
        error = 1;
        fputs("\nError:\n", stderr);

        if (locale == 1) {
            fputs("You did not input the coordinates of the starting location.\n", stderr);
            fputs("Please input the coordinates of the starting location through the corresponding option.\n", stderr);
        }
        else {
            fputs("Anda belum memasukkan sebuah koordinat untuk lokasi awal.\n", stderr);
            fputs("Mohon masukkan koordinat untuk lokasi awal pencarian rute melalui opsi yang sesuai.\n", stderr);
        }
        exit(1);
    }
    else {
        strcat(URL, "&start=");
        strcat(URL, start);
    }

    // End location check
    if (strcmp(finish, "\0") == 0) {
        error = 1;
        fputs("\nError:\n", stderr);

        if (locale == 1) {
            fputs("You did not input the coordinates of the end location.\n", stderr);
            fputs("Please input the coordinates of the end location through the corresponding option.\n", stderr);
        }
        else {
            fputs("Anda belum memasukkan sebuah koordinat untuk lokasi akhir.\n", stderr);
            fputs("Mohon masukkan koordinat untuk lokasi akhir pencarian rute melalui opsi yang sesuai.\n", stderr);
        }
        exit(1);
    }
    else {
        strcat(URL, "&finish=");
        strcat(URL, finish);
    }

    strcat(URL, "&presentation=desktop");
    strcat(URL, "&apikey=68CD281C8A8EE97C");
}
\end{lstlisting}

\begin{lstlisting}[label={appdx:A-buildurl-reset}, language=C, caption=\texttt{reset\textunderscore url()}]
void reset_url() {
    // Reset URL for multiple API processes at once
    strcpy(URL, "https://projectkiri.id/api?version=2");
}
\end{lstlisting}

\begin{lstlisting}[label={appdx:A-curl-execute}, language=C, caption=\texttt{execute\textunderscore curl()}]
void execute_curl() {
    CURL *curl;
    CURLcode response;
    memset(&responsedata, 0, sizeof(responsedata)); // Empty response data chunk for multiprocess modes

    curl_global_init(CURL_GLOBAL_DEFAULT);
    curl = curl_easy_init();

    if (curl) {
        curl_easy_setopt(curl, CURLOPT_URL, URL);
        curl_easy_setopt(curl, CURLOPT_WRITEFUNCTION, write_memalloc);
        curl_easy_setopt(curl, CURLOPT_WRITEDATA, (void *)&responsedata);
        response = curl_easy_perform(curl);
        if (response != CURLE_OK) {
            print_curl_error();
        }

        switch (mode) {
            case 2: // searchplace
                write_searchplace();
                break;
            
            case 3: // findroute
                write_findroute();
                break;

            case 4: // directroute
                if (step == 0 || step == 1) { // search start and finish
                    write_searchplace_noreturns();
                }
                else if (step == 2) { // findroute
                    write_findroute();
                }
                break;

            default:
                break;
        }

        curl_easy_cleanup(curl);
    }

    curl_global_cleanup();
}
\end{lstlisting}

\begin{lstlisting}[label={appdx:A-curl-error}, language=C, caption=\texttt{print\textunderscore curl\textunderscore error()}]
void print_curl_error() {
    error = 1;
    fputs("\nError:\n", stderr);

    if (locale == 1) {
        fputs("A connection error has occurred.\n", stderr);
        fputs("Please verify whether the internet connection is up and running.\n", stderr);
    }
    else {
        fputs("Telah terjadi error koneksi.\n", stderr);
        fputs("Mohon cek apakah koneksi internet aktif dan terhubung dengan baik.\n", stderr);
    }
}
\end{lstlisting}

\begin{lstlisting}[label={appdx:A-write-memalloc}, language=C, caption=\texttt{write\textunderscore memalloc()}]
// Allocate the memory of incoming data
// If data size + original allocated size exceeds the memory capability, print an error.
size_t write_memalloc(void *incomingdata, size_t size, size_t nmemb, void *userdata) {
    size_t realsize = size * nmemb;
    struct chunk *memory = (struct chunk *)userdata;
    char *ptr = realloc(memory->data, memory->size + realsize + 1);
    if(ptr == NULL) {
        fprintf(stderr, "Out of memory!\n");
        return 0;
    } 
    memory->data = ptr;
    memcpy(&(memory->data[memory->size]), incomingdata, realsize);
    memory->size += realsize;
    memory->data[memory->size] = 0;

    return realsize;
}
\end{lstlisting}

\begin{lstlisting}[label={appdx:A-write-searchplace}, language=C, caption=\texttt{write\textunderscore searchplace()}]
void write_searchplace() {
    cJSON *status;
    cJSON *result;
    cJSON *resultitem;
    cJSON *resultitemname;
    cJSON *resultitemlocation;
    int indexitem;

    responseJSON = cJSON_Parse(responsedata.data);
    status = cJSON_GetObjectItem(responseJSON, "status");
    
    // Check whether API returned an error
    if (strcmp(status->valuestring, "ok") != 0) {
        error = 1;
        fputs("\nError:\n", stderr);

        if (locale == 1) {
            fputs("API returned an error as its output.\n", stderr);
            fputs("Please recheck whether the query was formatted correctly.\n", stderr);
        }
        else {
            fputs("API mengembalikan error sebagai keluarannya.\n", stderr);
            fputs("Silakan cek ulang apakah kata kunci pencarian sudah diformat dengan benar.\n", stderr);
        }
    }
    else {
        result = cJSON_GetObjectItem(responseJSON, "searchresult");

        // Check whether API managed to find a single result
        if (cJSON_GetArraySize(result) == 0) {
            error = 1;
            fputs("\n", stderr);

            if (locale == 1) {
                fputs("Location not found.\n", stderr);
                fputs("Please recheck whether the search keyword was inputted correctly.\n", stderr);
            }
            else {
                fputs("Lokasi tidak berhasil ditemukan.\n", stderr);
                fputs("Silakan cek ulang apakah kata kunci pencarian sudah dimasukkan dengan benar.\n", stderr);
            }
        }
        else {
            indexitem = 1;
            
            cJSON_ArrayForEach(resultitem, result) {
                resultitemname = cJSON_GetObjectItem(resultitem, "placename");
                resultitemlocation = cJSON_GetObjectItem(resultitem, "location");

                // Print location index only if there were more than one found
                if (cJSON_GetArraySize(result) > 1) {
                    if (locale == 1) {
                        printf("Location %d:\n", indexitem);
                    }
                    else printf("Lokasi %d:\n", indexitem);
                }
                else {
                    if (locale == 1) {
                        puts("Location:");
                    }
                    else puts("Lokasi:");
                }
                puts("--------------------");

                if (locale == 1) {
                    printf("Name: %s\n", resultitemname->valuestring);
                    printf("Coordinates: %s\n", resultitemlocation->valuestring);
                }
                else {
                    printf("Nama: %s\n", resultitemname->valuestring);
                    printf("Koordinat: %s\n", resultitemlocation->valuestring);
                }
                putchar('\n');

                indexitem++;
            }
        }
    }
}
\end{lstlisting}

\begin{lstlisting}[label={appdx:A-write-findroute}, language=C, caption=\texttt{write\textunderscore findroute()}]
void write_searchplace_noreturns() {
    cJSON *status;
    cJSON *result;
    cJSON *resultitem;
    cJSON *resultitemname;
    cJSON *resultitemlocation;

    responseJSON = cJSON_Parse(responsedata.data);
    status = cJSON_GetObjectItem(responseJSON, "status");
    
    // Check whether API returned an error
    if (strcmp(status->valuestring, "ok") != 0) {
        error = 1;
        fputs("\nError:\n", stderr);
        
        if (step == 0) {
            if (locale == 1) {
                fputs("API returned an error as its start coordinates.\n", stderr);
                fputs("Please recheck whether the starting location coordinates were inputted correctly.\n", stderr);
            }
            else {
                fputs("API mengembalikan error sebagai koordinat lokasi awal.\n", stderr);
                fputs("Silakan cek ulang apakah koordinat lokasi awal sudah dimasukkan dengan benar.\n", stderr);
            }
        }
        else if (step == 1) {
            if (locale == 1) {
                fputs("API returned an error as its end coordinates.\n", stderr);
                fputs("Please recheck whether the end location coordinates were inputted correctly.\n", stderr);
            }
            else {
                fputs("API mengembalikan error sebagai koordinat lokasi akhir.\n", stderr);
                fputs("Silakan cek ulang apakah koordinat lokasi akhir sudah dimasukkan dengan benar.\n", stderr);
            }
        }
    }
    else {
        result = cJSON_GetObjectItem(responseJSON, "searchresult");

        // Check whether API managed to find a single result
        if (cJSON_GetArraySize(result) == 0) {
            error = 1;
            fputs("\n", stderr);
            
            if (step == 0) {
                if (locale == 1) {
                    fputs("Starting location not found.\n", stderr);
                    fputs("Please recheck whether the search keyword was inputted correctly.\n", stderr);
                }
                else {
                    fputs("Lokasi awal tidak berhasil ditemukan.\n", stderr);
                    fputs("Silakan cek ulang apakah kata kunci pencarian sudah dimasukkan dengan benar.\n", stderr);
                }
            }
            else if (step == 1) {
                if (locale == 1) {
                    fputs("Finish location not found.\n", stderr);
                    fputs("Please recheck whether the search keyword was inputted correctly.\n", stderr);
                }
                else {
                    fputs("Lokasi akhir tidak berhasil ditemukan.\n", stderr);
                    fputs("Silakan cek ulang apakah kata kunci pencarian sudah dimasukkan dengan benar.\n", stderr);
                }
            }
        }
        else {
            // Direct route mode only supports first location found
            resultitem = cJSON_GetArrayItem(result, 0);
            resultitemname = cJSON_GetObjectItem(resultitem, "placename");
            resultitemlocation = cJSON_GetObjectItem(resultitem, "location");

            if (step == 0) {
                if (locale == 1) {
                    printf("Starting location: %s\n", resultitemname->valuestring);
                }
                else printf("Lokasi awal: %s\n", resultitemname->valuestring);
                strcpy(start, resultitemlocation->valuestring);
            }
            else if (step == 1) {
                if (locale == 1) {
                    printf("Finish location: %s\n", resultitemname->valuestring);
                }
                else printf("Lokasi akhir: %s\n", resultitemname->valuestring);
                strcpy(finish, resultitemlocation->valuestring);
            }
        }
    }
}
\end{lstlisting}

\begin{lstlisting}[label={appdx:A-write-searchplacenoreturns}, language=C, caption=\texttt{write\textunderscore searchplace\textunderscore noreturns()}]
void write_findroute() {
    cJSON *status;
    cJSON *result;
    cJSON *route;
    cJSON *routesteps;
    cJSON *routetime;
    cJSON *routestepitem;
    cJSON *routestepdetail;
    char routetimestring[20]; // Store translated duration string
    char *routetimetemp;
    char *timearraytoken;
    int indexroute;
    int indexstep;

    responseJSON = cJSON_Parse(responsedata.data);
    status = cJSON_GetObjectItem(responseJSON, "status");

    // Check whether API returned an error
    if (strcmp(status->valuestring, "ok") != 0) {
        error = 1;
        fputs("\nError:\n", stderr);

        if (locale == 1) {
            fputs("API returned an error as its output.\n", stderr);
            fputs("Please recheck whether both location's coordinates were inputted correctly.\n", stderr);
        }
        else {
            fputs("API mengembalikan error sebagai keluarannya.\n", stderr);
            fputs("Silakan cek ulang apakah koordinat kedua lokasi sudah dimasukkan dengan benar.\n", stderr);
        }
    }
    else {
        result = cJSON_GetObjectItem(responseJSON, "routingresults");
        indexroute = 1;

        // For each possible routes...
        cJSON_ArrayForEach(route, result) {
            routesteps = cJSON_GetObjectItem(route, "steps");
            routetime = cJSON_GetObjectItem(route, "traveltime");

            if (mode == 4 && indexroute == 1) {
                putchar('\n');
            }

            // Check whether API managed to find a single result
            if (cJSON_IsNull(routetime)) {
                error = 1;
                fputs("\n", stderr);

                if (locale == 1) {
                    fputs("Sorry, we couldn't find a public transport route for your trip.\n", stderr);
                }
                else fputs("Maaf, kami tidak dapat menemukan rute transportasi publik untuk perjalanan anda.\n", stderr);
            }
            else {
                routetimetemp = routetime->valuestring;
                memset(routetimestring, 0, sizeof(routetimestring));
                indexstep = 1;

                // Fix bug from KIRI API in which "minute" time unit is not translated in id
                // Else skip whole process
                if (locale != 1) {
                    timearraytoken = strtok(routetimetemp, " ");

                    while (timearraytoken != NULL) {
                        if (strcmp(timearraytoken, "minutes") == 0) {
                            strcpy(timearraytoken, "menit");
                        }
                        strcat(routetimestring, timearraytoken);
                        strcat(routetimestring, " ");

                        timearraytoken = strtok(NULL, " ");
                    }
                }
                else strcpy(routetimestring, routetimetemp);

                // Print route index only if there were more than one found
                // Otherwise simply print "Route:"
                if (cJSON_GetArraySize(result) > 1) {
                    if (locale == 1) {
                        printf("Route %d:\n", indexroute);
                    }
                    else printf("Rute %d:\n", indexroute);
                }
                else {
                    if (locale == 1) {
                        puts("Route:");
                    }
                    else puts("Rute:");
                }
                puts("--------------------");
                
                if (locale == 1) {
                    printf("Estimated duration: %s\n", routetimestring);
                }
                else printf("Estimasi waktu: %s\n", routetimestring);
                puts("--------------------");
                
                cJSON_ArrayForEach(routestepitem, routesteps) {
                    routestepdetail = cJSON_GetArrayItem(routestepitem, 3);

                    if (locale == 1) {
                        printf("Step %d: ", indexstep);
                    }
                    else printf("Langkah %d: ", indexstep);
                    printf("%s\n", routestepdetail->valuestring);
                    indexstep++;
                }
            }
            if (error != 1) {
                putchar('\n');
            };

            indexroute++;
        }
    }
}
\end{lstlisting}

\begin{lstlisting}[label={appdx:A-main}, language=C, caption=Fungsi utama perkakas (\texttt{main})]
int main(int argc, char **argv) {
    int funct;
    extern int opterr;
    opterr = 0;

    while (1) {
        static struct option long_options[] = {
            {"help",        no_argument,        0,  'h'},
            {"mode",        required_argument,  0,  'm'},
            {"region",      required_argument,  0,  'r'},
            {"query",       required_argument,  0,  'q'},
            {"start",       required_argument,  0,  's'},
            {"finish",      required_argument,  0,  'f'},
            {"regstart",    required_argument,  0,  'S'},
            {"regfinish",   required_argument,  0,  'F'},
            {"locale",      required_argument,  0,  'l'},
            {0, 0, 0, 0}
        };
        int option_index = 0;

        funct = getopt_long (argc, argv, ":hm:r:q:s:S:f:F:l:", long_options, &option_index);

        if (funct == -1) {
            break;
        }

        switch (funct) {
            case 'h':
                // Program help
                if (mode == -1) {
                    mode = 1;
                }
                break;

            case 'm':
                if (mode == -1) {
                    // Program mode
                    if (strcmp(optarg, "searchplace") == 0) {
                        mode = 2;
                    }
                    else if (strcmp(optarg, "findroute") == 0) {
                        mode = 3;
                    }
                    else if (strcmp(optarg, "direct") == 0) {
                        mode = 4;
                    }
                    else {
                        mode = 0;
                    }
                    break;
                }
            
            case 'r':
                // Location regions
                if (strcmp(optarg, "cgk") == 0) {
                    region = 1;
                }
                else if (strcmp(optarg, "bdo") == 0) {
                    region = 2;
                }
                else if (strcmp(optarg, "mlg") == 0) {
                    region = 3;
                }
                else if (strcmp(optarg, "sub") == 0) {
                    region = 4;
                }
                else {
                    region = 0;
                }
                break;

            case 'q':
                // General location search keyword
                if (optarg[0] != '\0') {
                    replace_space(optarg);
                    strcpy(query, escape);
                    memset(escape, 0, sizeof(escape));
                }
                break;
            
            case 's':
                // Starting location input
                if (optarg[0] != '\0') {
                    replace_space(optarg);
                    strcpy(start, escape);
                    memset(escape, 0, sizeof(escape));
                }
                break;
            
            case 'f':
                // Finish location input
                if (optarg[0] != '\0') {
                    replace_space(optarg);
                    strcpy(finish, escape);
                    memset(escape, 0, sizeof(escape));
                }
                break;
            
            case 'S':
                // Starting location regions
                if (strcmp(optarg, "cgk") == 0) {
                    regstart = 1;
                }
                else if (strcmp(optarg, "bdo") == 0) {
                    regstart = 2;
                }
                else if (strcmp(optarg, "mlg") == 0) {
                    regstart = 3;
                }
                else if (strcmp(optarg, "sub") == 0) {
                    regstart = 4;
                }
                else {
                    regstart = 0;
                }
                break;
            
            case 'F':
                // Finish location regions
                if (strcmp(optarg, "cgk") == 0) {
                    regfinish = 1;
                }
                else if (strcmp(optarg, "bdo") == 0) {
                    regfinish = 2;
                }
                else if (strcmp(optarg, "mlg") == 0) {
                    regfinish = 3;
                }
                else if (strcmp(optarg, "sub") == 0) {
                    regfinish = 4;
                }
                else {
                    regfinish = 0;
                }
                break;
            
            case 'l':
                // Locale
                if (strcmp(optarg, "id") == 0) {
                    locale = 0;
                }
                else if (strcmp(optarg, "en") == 0) {
                    locale = 1;
                }
                else {
                    locale = 2;
                }
                break;

            case ':':
                // Error: missing arguments
                error = 1;
                fputs("\nError:\n", stderr);

                if (locale == 1) {
                    fputs("One of the options inputted was missing its required argument.\n", stderr);
                    fputs("Please recheck the input command's syntax.\n", stderr);
                }
                else {
                    fputs("Salah satu dari opsi yang anda masukkan kehilangan argumen yang dibutuhkan.\n", stderr);
                    fputs("Mohon periksa kembali penulisan perintah yang anda masukkan.\n", stderr);
                }
                exit(1);

            case '?':
                // Error: unknown options
                error = 1;
                fputs("\nError:\n", stderr);

                if (locale == 1) {
                    fputs("You have inputted an invalid option.\n", stderr);
                    fputs("Please recheck the input command's syntax.\n", stderr);
                }
                else {
                    fputs("Anda telah memasukkan opsi yang tidak valid.\n", stderr);
                    fputs("Mohon periksa kembali penulisan perintah yang anda masukkan.\n", stderr);
                }
                exit(1);

            default:
                exit(1);
        }
    }

    // Error: extra arguments
    if (optind < argc) {
        error = 1;
        fputs("\nError:\n", stderr);

        if (locale == 1) {
            fprintf(stderr, "You have inserted some extra arguments: ");
            while (optind < argc) {
                fprintf(stderr, "%s ", argv[optind++]);
            }
            fputs("\nPlease recheck the input command's syntax.\n", stderr);
        }
        else {
            fprintf(stderr, "Anda telah memasukkan kelebihan argumen: ");
            while (optind < argc) {
                fprintf(stderr, "%s ", argv[optind++]);
            }
            fputs("\nMohon periksa kembali penulisan perintah yang anda masukkan.\n", stderr);
        }
    }
    else {
        switch (mode) {
            case -1:
                error = 1;
                fputs("\nError:\n", stderr);

                if (locale == 1) {
                    fputs("Please enter tool operation mode.\n", stderr);
                }
                else {
                    fputs("Mohon masukkan mode pengunaan perkakas.\n", stderr);   
                }
                exit(1);
                break;

            case 1:
                print_help();
                break;
            
            case 2:
                build_url_searchplace(locale, region, query);
                execute_curl();
                break;
            
            case 3:
                build_url_findroute(locale, start, finish);
                execute_curl();
                break;
            
            case 4:
                step = 0;
                build_url_searchplace(locale, regstart, start);
                execute_curl();
                if (error == 1) {
                    break;
                }
                else reset_url();

                step = 1;
                build_url_searchplace(locale, regfinish, finish);
                execute_curl();
                if (error == 1) {
                    break;
                }
                else reset_url();

                step = 2;
                build_url_findroute(locale, start, finish);
                execute_curl();
                break;

            default:
                error = 1;
                fputs("\nError:\n", stderr);

                if (locale == 1) {
                    fputs("You have entered an invalid operation mode.\n", stderr);
                    fputs("Please recheck whether the operation mode had been typed correctly.\n", stderr);
                }
                else {
                    fputs("Anda telah memasukkan mode yang tidak valid.\n", stderr);
                    fputs("Mohon periksa kembali apakah mode yang anda masukkan sudah diketik dengan benar.\n", stderr);
                }
                exit(1);
                break;
        }
    }

    exit(0);
}
\end{lstlisting}

\lstinputlisting[label={appdx:A-cmakelists}, language={}, caption=CMakeLists.txt]{./Lampiran/CMakeLists.txt}

\lstinputlisting[label={appdx:A-manpage}, language={}, caption=kiritool.1 (\textit{Source Code man page})]{./Lampiran/kiritool.1}

\lstinputlisting[label={appdx:A-maincode}, language=C, caption=Keseluruhan kode perkakas (main.c)]{./Lampiran/main.c}
