%versi 3 (18-12-2016)
\chapter{Kode Perkakas \textit{Command Line} KIRI}
\label{lamp:A}

%terdapat 2 cara untuk memasukkan kode program
% 1. menggunakan perintah \lstinputlisting (kode program ditempatkan di folder yang sama dengan file ini)
% 2. menggunakan environment lstlisting (kode program dituliskan di dalam file ini)
% Perhatikan contoh yang diberikan!!
%
% untuk keduanya, ada parameter yang harus diisi:
% - language: bahasa dari kode program (pilihan: Java, C, C++, PHP, Matlab, C#, HTML, R, Python, SQL, dll)
% - caption: nama file dari kode program yang akan ditampilkan di dokumen akhir
%
% Perhatian: Abaikan warning tentang textasteriskcentered!!
%
\lstinputlisting[label={appdx:A-cmakelists}, language={}, caption=CMakeLists.txt]{./Lampiran/CMakeLists.txt}
\lstinputlisting[label={appdx:A-maincode}, language=C, caption=main.c]{./Lampiran/main.c}
\lstinputlisting[label={appdx:A-manpage}, language={}, caption=kiritool.1 (\textit{Source Code man page})]{./Lampiran/kiritool.1}
\lstinputlisting[label={appdx:A-output-help}, language={}, caption=Bantuan penggunaan perkakas]{./Lampiran/help.txt}
\lstinputlisting[label={appdx:A-output-manpage}, language={}, caption=man page Perkakas \textit{Command Line} KIRI]{./Lampiran/man.txt}
