%_____________________________________________________________________________
%=============================================================================
% data.tex v11 (24-07-2020) dibuat oleh Lionov - Informatika FTIS UNPAR
%
% Perubahan pada versi 11 (24-07-2020)
%	- Penambahan enumitem dan nosep untuk semua list, untuk menghemat kertas
%   - Bagian V: penambahan opsi Daftar Kode Program dan Daftar Notasi
%   - Bagian XIV: menjadi Bagian XVI
%   - Bagian XV: menjadi Bagian XVII
%   - Bagian XIV yang baru: untuk pilihan jenis tanda tangan mahasiswa
%   - Bagian XV yang baru: untuk pilihan munculnya tanda tangan digital untuk
%     dosen/pejabat
%_____________________________________________________________________________
%=============================================================================

%=============================================================================
% 								PETUNJUK
%=============================================================================
% Ini adalah file data (data.tex)
% Masukkan ke dalam file ini, data-data yang diperlukan oleh template ini
% Cara memasukkan data dijelaskan di setiap bagian
% Data yang WAJIB dan HARUS diisi dengan baik dan benar adalah SELURUHNYA !!
% Hilangkan tanda << dan >> jika anda menemukannya
%=============================================================================

%_____________________________________________________________________________
%=============================================================================
% 								BAGIAN 0
%=============================================================================
% Entri untuk memperbaiki posisi "DAFTAR ISI" jika tidak berada di bagian 
% tengah halaman. Sayangnya setiap sistem menghasilkan posisi yang berbeda.
% Isilah dengan 0 atau 1 (e.g. \daftarIsiError{1}). 
% Pemilihan 0 atau 1 silahkan disesuaikan dengan hasil PDF yang dihasilkan.
%=============================================================================
\daftarIsiError{0}   
%\daftarIsiError{1}   
%=============================================================================

%_____________________________________________________________________________
%=============================================================================
% 								BAGIAN I
%=============================================================================
% Tambahkan package2 lain yang anda butuhkan di sini
%=============================================================================
\usepackage{booktabs} 
\usepackage{longtable}
\usepackage{amssymb}
\usepackage{todo}
\usepackage{verbatim} 		%multiline comment
\usepackage{pgfplots}
\usepackage{enumitem}
%\overfullrule=3mm % memperlihatkan overfull 

% Custom packages
\usepackage{algorithm} % algorithm
\usepackage{algpseudocode} % algorithm - pseudocodes
\usepackage{booktabs} % longtable row spacing
\usepackage{graphicx} % footnotemark & footnotetext
\usepackage{longtable} % scenario case tables
\usepackage[stable]{footmisc} % footnote in section titles
\usepackage{subcaption} % subfigure
\usepackage{verbatim} % block comment
\usepackage{xspace} % force space
%=============================================================================

%_____________________________________________________________________________
%=============================================================================
% 								BAGIAN II
%=============================================================================
% Mode dokumen: menentukan halaman depan dari dokumen, apakah harus mengandung 
% prakata/pernyataan/abstrak dll (termasuk daftar gambar/tabel/isi) ?
% - final 		: hanya untuk buku skripsi, dicetak lengkap: judul ina/eng, 
%   			  pengesahan, pernyataan, abstrak ina/eng, untuk, kata 
%				  pengantar, daftar isi (daftar tabel dan gambar tetap 
%				  opsional dan dapat diatur), seluruh bab dan lampiran.
%				  Otomatis tidak ada nomor baris dan singlespacing
% - sidangakhir	: buku sidang akhir = buku final - (pengesahan + pernyataan +
%   			  untuk + kata pengantar)
%				  Otomatis ada nomor baris dan onehalfspacing 
% - sidang 		: untuk sidang 1, buku sidang = buku sidang akhir - (judul 
%				  eng + abstrak ina/eng)
%				  Otomatis ada nomor baris dan onehalfspacing
% - bimbingan	: untuk keperluan bimbingan, hanya terdapat bab dan lampiran
%   			  saja, bab dan lampiran yang hendak dicetak dapat ditentukan 
%				  sendiri (nomor baris dan spacing dapat diatur sendiri)
% Mode default adalah 'template' yang menghasilkan isian berwarna merah, 
% aktifkan salah satu mode di bawah ini :
%=============================================================================
%\mode{bimbingan} 		% untuk keperluan bimbingan
%\mode{sidang} 			% untuk sidang 1
%\mode{sidangakhir} 	% untuk sidang 2 / sidang pada Skripsi 2(IF)
%\mode{final} 			% untuk mencetak buku skripsi 
%=============================================================================
\mode{final}

%_____________________________________________________________________________
%=============================================================================
% 								BAGIAN III
%=============================================================================
% Line numbering: penomoran setiap baris, nomor baris otomatis di-reset ke 1
% setiap berganti halaman.
% Sudah dikonfigurasi otomatis untuk mode final (tidak ada), mode sidang (ada)
% dan mode sidangakhir (ada).
% Untuk mode bimbingan, defaultnya ada (\linenumber{yes}), jika ingin 
% dihilangkan, isi dengan "no" (i.e.: \linenumber{no})
% Catatan:
% - jika nomor baris tidak kembali ke 1 di halaman berikutnya, compile kembali
%   dokumen latex anda
% - bagian ini hanya bisa diatur di mode bimbingan
%=============================================================================
%\linenumber{no} 
\linenumber{yes}
%=============================================================================

%_____________________________________________________________________________
%=============================================================================
% 								BAGIAN IV
%=============================================================================
% Linespacing: jarak antara baris 
% - single	: otomatis jika ingin mencetak buku skripsi, opsi yang 
%			     disediakan untuk bimbingan, jika pembimbing tidak keberatan 
%			     (untuk menghemat kertas)
% - onehalf	: otomatis jika ingin mencetak dokumen untuk sidang
% - double 	: jarak yang lebih lebar lagi, jika pembimbing berniat memberi 
%             catatan yg banyak di antara baris (dianjurkan untuk bimbingan)
% Catatan: bagian ini hanya bisa diatur di mode bimbingan
%=============================================================================
\linespacing{single}
%\linespacing{onehalf}
%\linespacing{double}
%=============================================================================

%_____________________________________________________________________________
%=============================================================================
% 								BAGIAN V
%=============================================================================
% Tidak semua skripsi memuat gambar, tabel, kode program, dan/atau notasi. 
% Untuk skripsi yang tidak memuat hal-hal tersebut, maka tidak diperlukan 
% Daftar Gambar, Daftar Tabel, Daftar Kode Program, dan/atau Daftar Notasi. 
% Sayangnya hal tsb sulit dilakukan secara manual karena membutuhkan 
% kedisiplinan pengguna template.  
% Jika tidak ingin menampilkan satu/lebih daftar-daftar tersebut (misalnya 
% untuk bimbingan), isi dengan "no" (e.g. \gambar{no})
%=============================================================================
\gambar{yes}
%\gambar{no}
%\tabel{yes}
\tabel{no} 
\kode{yes}
%\kode{no} 
%\notasi{yes}
\notasi{no}
%=============================================================================

%_____________________________________________________________________________
%=============================================================================
% 								BAGIAN VI
%=============================================================================
% Pada mode final, sidang da sidangkahir, seluruh bab yang ada di folder "Bab"
% dengan nama file bab1.tex, bab2.tex s.d. bab9.tex akan dicetak terurut, 
% apapun isi dari perintah \bab.
% Pada mode bimbingan, jika ingin:
% - mencetak seluruh bab, isi dengan 'all' (i.e. \bab{all})
% - mencetak beberapa bab saja, isi dengan angka, pisahkan dengan ',' 
%   dan bab akan dicetak terurut sesuai urutan penulisan (e.g. \bab{1,3,2}). 
% Catatan: Jika ingin menambahkan bab ke-3 s.d. ke-9, tambahkan file 
% bab3.tex, bab4.tex, dst di folder "Bab". Untuk bab ke-10 dan 
% seterusnya, harus dilakukan secara manual dengan mengubah file skripsi.tex
% Catatan: bagian ini hanya bisa diatur di mode bimbingan
%=============================================================================
\bab{all}
%=============================================================================

%_____________________________________________________________________________
%=============================================================================
% 								BAGIAN VII
%=============================================================================
% Pada mode final, sidang dan sidangkhir, seluruh lampiran yang ada di folder 
% "Lampiran" dengan nama file lampA.tex, lampB.tex s.d. lampJ.tex akan dicetak 
% terurut, apapun isi dari perintah \lampiran.
% Pada mode bimbingan, jika ingin:
% - mencetak seluruh lampiran, isi dengan 'all' (i.e. \lampiran{all})
% - mencetak beberapa lampiran saja, isi dengan huruf, pisahkan dengan ',' 
%   dan lampiran akan dicetak terurut sesuai urutan (e.g. \lampiran{A,E,C}). 
% - tidak mencetak lampiran apapun, isi dengan "none" (i.e. \lampiran{none})
% Catatan: Jika ingin menambahkan lampiran ke-C s.d. ke-I, tambahkan file 
% lampC.tex, lampD.tex, dst di folder Lampiran. Untuk lampiran ke-J dan 
% seterusnya, harus dilakukan secara manual dengan mengubah file skripsi.tex
% Catatan: bagian ini hanya bisa diatur di mode bimbingan
%=============================================================================
\lampiran{all}
%=============================================================================

%_____________________________________________________________________________
%=============================================================================
% 								BAGIAN VIII
%=============================================================================
% Data diri dan skripsi/tugas akhir
% - namanpm		: Nama dan NPM anda, penggunaan huruf besar untuk nama harus 
%				  benar dan gunakan 10 digit npm UNPAR, PASTIKAN BAHWA 
%				  BENAR !!! (e.g. \namanpm{Jane Doe}{1992710001}
% - judul 		: Dalam bahasa Indonesia, perhatikan penggunaan huruf besar, 
%				  judul tidak menggunakan huruf besar seluruhnya !!! 
% - tanggal 	: isi dengan {tangga}{bulan}{tahun} dalam angka numerik, 
%				  jangan menuliskan kata (e.g. AGUSTUS) dalam isian bulan.
%			  	  Tanggal ini adalah tanggal dimana anda akan melaksanakan 
%				  sidang ujian akhir skripsi/tugas akhir
% - pembimbing	: pembimbing anda, lihat daftar dosen di file dosen.tex
%				  jika pembimbing hanya 1, kosongkan parameter kedua 
%				  (e.g. \pembimbing{\JND}{} ), \JND adalah kode dosen
% - penguji 	: para penguji anda, lihat daftar dosen di file dosen.tex
%				  (e.g. \penguji{\JHD}{\JCD} )
% !!Lihat singkatan pembimbing dan penguji anda di file dosen.tex!!
% Petunjuk: hilangkan tanda << & >>, dan isi sesuai dengan data anda
%=============================================================================
\namanpm{Alfred Aprianto Liaunardi}{6181801014}
\tanggal{19}{1}{2023}         %isi bulan dengan angka - not final
\pembimbing{\PAN}{}
\penguji{\VAN}{\LNV} 
%=============================================================================

%_____________________________________________________________________________
%=============================================================================
% 								BAGIAN IX
%=============================================================================
% Judul dan title : judul bhs indonesia dan inggris
% - judulINA: judul dalam bahasa indonesia
% - judulENG: title in english
% Petunjuk: 
% - hilangkan tanda << & >>, dan isi sesuai dengan data anda
% - langsung mulai setelah '{' awal, jangan mulai menulis di baris bawahnya
% - gunakan \texorpdfstring{\\}{} untuk pindah ke baris baru
% - judul TIDAK ditulis dengan menggunakan huruf besar seluruhnya !!
%=============================================================================
\judulINA{Perkakas Antarmuka Baris Perintah Untuk Aplikasi Berbasis Web KIRI}
\judulENG{Command Line Tool for KIRI Web-Based Application}
%_____________________________________________________________________________
%=============================================================================
% 								BAGIAN X
%=============================================================================
% Abstrak dan abstract : abstrak bhs indonesia dan inggris
% - abstrakINA: abstrak bahasa indonesia
% - abstrakENG: abstract in english 
% Petunjuk: 
% - hilangkan tanda << & >>, dan isi sesuai dengan data anda
% - langsung mulai setelah '{' awal, jangan mulai menulis di baris bawahnya
%=============================================================================
\begin{comment}
Project KIRI (atau KIRI saja) merupakan sebuah perkakas berbasis \textit{web} yang dapat digunakan untuk membantu memudahkan penggunanya untuk menggunakan angkutan kota (angkot). KIRI merealisasikan hal ini dengan cara menunjukkan kepada penggunanya langkah-langkah yang harus dilakukan untuk pergi dari suatu lokasi ke lokasi lainnya, angkot-angkot mana yang harus dinaiki dalam rute tersebut, serta di mana pengguna harus menaiki atau turun dari angkot-angkot yang bersangkutan. Selain itu, KIRI juga dapat menunjukkan estimasi waktu dari rute-rute yang ditemukan.

Sementara itu, dalam komputer, salah satu dari sekian banyak tipe perangkat lunak adalah perangkat lunak berbasis \cl . Perangkat-perangkat lunak ini selalu terdiri atas sebuah kotak (\textit{window}) yang memuat teks berupa perintah-perintah, yang menerima masukan (berupa perintah-perintah juga) langsung dari pengguna dan menjalankannya. Indikator utama dari tipe perangkat lunak ini adalah bahwa perangkat-perangkat lunak jenis ini tidak memiliki tampilan dengan gambar grafis apapun\textemdash dalam artian bahwa tampilan perangkat tipe ini hanya berupa teks.

Dalam skripsi ini akan dibuat sebuah perangkat lunak berupa perkakas \cl yang dapat menjalankan fungsi-fungsi API KIRI. Seperti jenis umumnya, perkakas ini akan dibuat murni sebagai perkakas yang dijalankan dari \cl , dan tidak akan memiliki tambahan gambar grafis apapun dalam tampilannya. Perkakas ini akan dibuat dalam bahasa C, dengan menggunakan berbagai macam \textit{library}, seperti getopt, cJSON, cURL, dan CMake, serta mengimplementasikan fitur-fitur API KIRI serta fitur-fitur dasar perkakas \cl , seperti mode bantuan, dan \textit{man page} (untuk Linux). Perkakas ini nantinya akan diuji coba dengan cara menguji satu-satu fungsinya, dan juga dengan menguji kasus-kasus umum untuk integrasinya dengan perkakas-perkakas lainnya yang sudah ada.

Hasil dari skripsi ini merupakan sebuah perkakas \cl yang dapat mengutilisasikan fungsi-fungsi API KIRI, serta memiliki fitur-fitur dasar perkakas \cl, yang meliputi sebuah halaman manual (berupa opsi (-{}-help) dan \textit{man page}), pengeluaran pesan \textit{error} yang sesuai dengan \textit{error} yang terjadi, serta kemampuan integrasi keluaran perkakas dengan perkakas-perkakas \cl lainnya.
\end{comment}

\abstrakINA{}

\begin{comment}
Project KIRI (or just KIRI) is a web-based tool which could be used to assist its users in utilizing \textit{angkutan kota} (or \textit{angkot}). KIRI does this by showing to its users the steps needed to go from one location to the other, which \textit{angkots} would have to be taken within said route, as well as where they would have to board or get off of these \textit{angkots}. Aside from that, KIRI also has the ability to show the estimated durations of the available routes.

Meanwhile, in computers, one of the many types of softwares is command line softwares. These programs always consist of a box (window) with texts in the form of commands, which accepts direct inputs (also in the form of commands) from the users, and run them. The main indicator of this type is the fact that these softwares do not have any graphical images in its interface\textemdash in the sense that the interface of this type of programs contains only texts.

In this undergraduate thesis, a command line tool will be made, in which the tool  would be able to run KIRI's API functions. Just like its general type, this tool will be run purely through the command line, and will not have any additional graphical images whatsoever in its interface. This newly-made tool will be made in C language, using various libraries such as getopt, cJSON, cURL, and CMake, along with implementing the KIRI API features, as well as the general features of a command line tool, such as a help mode, and a man page (for Linux). This tool will later be tested by testing each of its functions, as well as the general cases of its integration with other, existing command line tools.

The resulting program of this thesis is a command line tool which utilizes KIRI API's functions, and has the conventional features of a command line tool, including the inclusion of a manual page (-{}-help option and a man page), outputting the corresponding error messages for various errors, along with an output integration ability with other, existing command line tools.
\end{comment}

\abstrakENG{}
%=============================================================================

%_____________________________________________________________________________
%=============================================================================
% 								BAGIAN XI
%=============================================================================
% Kata-kata kunci dan keywords : diletakkan di bawah abstrak (ina dan eng)
% - kunciINA: kata-kata kunci dalam bahasa indonesia
% - kunciENG: keywords in english
% Petunjuk: hilangkan tanda << & >>, dan isi sesuai dengan data anda.
%=============================================================================
\kunciINA{Navigasi, angkot, \textit{Project KIRI}, \textit{command line}, C}
\kunciENG{Navigation, \textit{angkot}, Project KIRI, command line, C}
%=============================================================================

%_____________________________________________________________________________
%=============================================================================
% 								BAGIAN XII
%=============================================================================
% Persembahan : kepada siapa anda mempersembahkan skripsi ini ...
% Petunjuk: hilangkan tanda << & >>, dan isi sesuai dengan data anda.
%=============================================================================
\untuk{Skripsi ini dipersembahkan kepada Tuhan Yang Maha Esa, keluarga, para dosen, rekan-rekan sesama mahasiswa, serta teman-teman yang sudah memberi motivasi, baik di dalam maupun di luar UNPAR.}
%=============================================================================

%_____________________________________________________________________________
%=============================================================================
% 								BAGIAN XIII
%=============================================================================
% Kata Pengantar: tempat anda menuliskan kata pengantar dan ucapan terima 
% kasih kepada yang telah membantu anda bla bla bla ....  
% Petunjuk: hilangkan tanda << & >>, dan isi sesuai dengan data anda.
%=============================================================================
\prakata{Puji syukur penulis panjatkan kepada Tuhan Yang Maha Esa atas berkat dan karunia-Nya, yang memungkinkan diselesaikannya penulisan skripsi yang berjudul ``Perkakas \textit{Command Line} KIRI'' ini. Selama penulisan skripsi ini tentunya penulis menemui berbagai macam kesulitan serta menghadapi berbagai macam halangan, yang puji syukur dapat diselesaikan dengan baik dengan bantuan beberapa pihak. Pada kesempatan ini, penulis ingin mengucapkan terima kasih sebesar-besarnya kepada pihak-pihak tersebut, yaitu:

\begin{itemize}
	\item Bapak Pascal Alfadian Nugroho, M.Comp. selaku dosen pembimbing yang telah membimbing serta membantu memberikan arahan kepada penulis dalam seluruh proses pembuatan skripsi ini.
	\item Ibu Vania Natali, M.T. serta Bapak Lionov, Ph. D. selaku dosen-dosen penguji yang telah memeriksa hasil skripsi ini serta memberikan kritik dan saran yang membangun.
	\item Teman-teman penulis\textemdash Daniel, Yalvi, Rama, JJ, serta teman-teman lainnya, yang telah memberikan dorongan, semangat, serta saran tambahan selama penulisan skripsi ini.
	\item Keluarga penulis, yang telah memberikan bantuan dan dukungan dalam seluruh aspek lainnya yang tidak berhubungan langsung dengan penulisan skripsi ini.
	\item Para staf tata usaha, baik dalam tata usaha FTIS maupun BAA, yang telah membantu penyelesaian semua urusan non-akademik dalam proses pembuatan skripsi ini.
\end{itemize}

Terakhir, penulis menyadari bahwa penulisan skripsi ini jauh dari sempurna, karena \mbox{berbagai} \mbox{macam} keterbatasan yang ada. Oleh karena itu, pada kesempatan ini penulis juga ingin \mbox{mengucapkan} permohonan maaf atas kekurangan-kekurangan yang ada. Walaupun begitu, penulis berharap skripsi ini dapat bermanfaat bagi seluruh pihak yang membacanya.}
%=============================================================================

%_____________________________________________________________________________
%=============================================================================
% 								BAGIAN XIV
%=============================================================================
% Jenis tandatangan di lembar pernyataan mahasiswa tentang plagiarisme.
% Ada 4 pilihan:
%   - digital   : diisi menggunakan digital signature (menggunakan pengolah
%                 pdf seperti Adobe Acrobat Reader DC).
%   - gambar    : diisi dengan gambar tandatangan mahasiswa (file tandatangan
%                 bertipe pdf/png/jpg). Dianjukan menggunakan warna biru.
%                 Letakkan gambar di folder "Gambar" dengan nama ttd.jpg/
%                 ttd.png/ttd.pdf (tergantung jenis file. Hapus file ttd.jpg
%                 yang digunakan sebagai contoh
%   - materai   : khusus bagi yang ingin mencetak buku dan menandatangani di 
%                 atas materai. Sama dengan pilihan ``digital'' dan dicetak.
%   - kosong    : tempat kosong ini bisa diisi dengan tanda tangan yang
%                 digambar langsung di atas pdf (fill&sign via acrobat, tanda
%                 tangan dapat dibuat dengan mouse atau stylus)
%=============================================================================
%\ttd{digital}
%\ttd{gambar}
%\ttd{meterai}
%\ttd{kosong}
%=============================================================================
\ttd{digital}

%_____________________________________________________________________________
%=============================================================================
% 								BAGIAN XV
%=============================================================================
% Pilihan tanda tangan digital untuk dosen/pejabat:
%   - no    : pdf TIDAK dapat ditandatangani secara digital, mengakomodasi 
%             yang akan menandatangani via ``menulis'' di file pdf
%   - yes   : pdf dapat ditandatangani secara digital
% 
% PERHATIAN: perubahan ini harus ditanyakan ke kaprodi/dosen koordinator, 
% apakah harus mengisi ``no" atau ``yes". Default = no 
% Untuk mahasiswa Informatika = yes
%=============================================================================
%\ttddosen{yes}
%=============================================================================
\ttddosen{yes}

%_____________________________________________________________________________
%=============================================================================
% 								BAGIAN XVI
%=============================================================================
% Tambahkan hyphen (pemenggalan kata) yang anda butuhkan di sini 
%=============================================================================
\hyphenation{ma-te-ma-ti-ka}
\hyphenation{fi-si-ka}
\hyphenation{tek-nik}
\hyphenation{in-for-ma-ti-ka}
\hyphenation{ber-da-sar-kan}
%=============================================================================

%_____________________________________________________________________________
%=============================================================================
% 								BAGIAN XVII
%=============================================================================
% Tambahkan perintah yang anda buat sendiri di sini 
%=============================================================================
\renewcommand{\vtemplateauthor}{lionov}
\pgfplotsset{compat=newest}
\setlist{nosep}

% New words/phrases
\newcommand{\api}{\textit{application programming interface}}
\newcommand{\chromewebstorecli}{Chrome \textit{Web Store Item Property} CLI\xspace}
\newcommand{\cl}{\textit{command line}}
\newcommand{\cli}{\textit{command line interface}}
\newcommand{\googlemapscli}{Google \textit{Maps Direction} CLI\xspace}
\newcommand{\gui}{\textit{graphical user interface}}
\newcommand{\itunesapi}{\textit{iTunes Search} API\xspace}
\newcommand{\latlon}{\textit{latitude} dan \textit{longitude}\xspace}
\newcommand{\logoregistered}{\textsuperscript{\textregistered}}
\newcommand{\logotrademark}{\textsuperscript{\tiny{TM}}}
\newcommand{\ubercli}{Uber CLI\xspace}
\newcommand{\websoftware}{perangkat lunak berbasis \textit{web}}

% Algorithmic - switch case
\algnewcommand\algorithmicswitch{\textbf{switch}}
\algnewcommand\algorithmiccase{\textbf{case}}
\algnewcommand\algorithmicdefault{\textbf{default}}
%\algnewcommand\algorithmicassert{\texttt{assert}}
%\algnewcommand\Assert[1]{\State \algorithmicassert(#1)}%
\algnewcommand\algorithmicforeach{\textbf{for each}}

\algdef{SE}[SWITCH]{Switch}{EndSwitch}[1]{\algorithmicswitch\ #1\ \algorithmicdo}{\algorithmicend\ \algorithmicswitch}%
\algdef{SE}[CASE]{Case}{EndCase}[1]{\algorithmiccase\ #1}{\algorithmicend\ \algorithmiccase}%
\algdef{SE}[CASE]{Default}{EndDefault}[1]{\algorithmicdefault\ #1}{\algorithmicend\ \algorithmiccase}%
\algdef{S}[FOR]{ForEach}[1]{\algorithmicforeach\ #1\ \algorithmicdo}
% remove "end x" text
%\algtext*{EndSwitch} 
\algtext*{EndCase}
\algtext*{EndDefault}
%=============================================================================
