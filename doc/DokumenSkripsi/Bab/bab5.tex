\chapter{Implementasi dan Pengujian}
\label{chap:testing}

Bab ini akan membahas hal-hal mengenai implementasi kode dari tiap-tiap fungsi dalam perkakas, serta skenario-skenario pengujian yang akan digunakan dalam pengujian fungsional perkakas.

\section{Implementasi Kode}
\label{sec:testing-implementation}

Subbab ini akan memaparkan implementasi dari fungsi-fungsi yang ada dalam perkakas secara singkat tetapi menyeluruh, serta membahas secara detail sebagian kecil dari fungsi-fungsi tersebut yang memerlukan penjelasan lebih lanjut.

\subsection{\texttt{print\textunderscore help()}}
\label{sec:testing-implementation-printhelp}

Fungsi ini akan menampilkan \textit{string-string} (dalam bentuk \textit{array-array} karakter) yang merupakan versi singkat dari bantuan penggunaan perkakas ke pengguna. Implementasi dari fungsi ini ada di Lampiran \ref{appdx:A-printhelp}.

\subsection{\texttt{replace\textunderscore space()}}
\label{sec:testing-implementation-replacespace}

Seperti yang telah dijelaskan di subbab \ref{sec:design-code-replacespace}, fungsi ini akan menerima sebuah \textit{array} karakter, mengganti semua karakter spasi yang ada di dalamnya menjadi ``\%20'', dan meletakkan hasilnya di variabel global \verb|escape|. Implementasi dari fungsi ini dapat dilihat di \textit{source code} murni perkakas dalam Lampiran \ref{appdx:A-replacespace}.

\subsection{\texttt{build\textunderscore url\textunderscore searchplace()}}
\label{sec:testing-implementation-buildurl-searchplace}

Fungsi ini bekerja dengan menambahkan URL ke variabel global \verb|url|, sesuai dengan opsi-opsi yang telah dimasukkan (dan diperlukan oleh API) untuk fitur \textit{searchplace}. Adapun implementasi dari fungsi ini dapat dilihat di Lampiran \ref{appdx:A-buildurl-searchplace}.

\subsection{\texttt{build\textunderscore url\textunderscore findroute()}}
\label{sec:testing-implementation-buildurl-findroute}

Fungsi ini bekerja dengan menambahkan URL ke variabel global \verb|url|, sesuai dengan opsi-opsi yang telah dimasukkan (dan diperlukan oleh API) untuk fitur \textit{findroute}. Adapun implementasi dari fungsi ini dapat dilihat di Lampiran \ref{appdx:A-buildurl-findroute}.

\subsection{\texttt{reset\textunderscore url()}}
\label{sec:testing-implementation-buildurl-reset}

Fungsi ini hanya terdiri atas satu baris kode yang akan mengembalikan isi dari variabel \verb|url| ke nilai awalnya. Implementasi dari fungsi ini dapat dilihat langsung di Lampiran \ref{appdx:A-buildurl-reset}.
	
\subsection{\texttt{execute\textunderscore curl()}}
\label{sec:testing-implementation-curl-execute}

Fungsi ini merupakan fungsi yang mengatur seluruh proses yang berhubungan langsung dengan cURL dalam perkakas, mulai dari inisialisasi \textit{easy handle} cURL (serta proses pembersihannya di akhir fungsi), pengaturan penerimaan data keluaran proses cURL (\verb|write_memalloc()|), pengaturan variabel apa dalam perkakas yang akan diisi oleh data keluaran, serta \textit{error handler} apabila proses cURL gagal. Selain itu, fungsi ini juga mengatur proses apa yang harus dilakukan setelah data tersebut berhasil diterima, dengan cara memanggil fungsi yang sesuai dengan mode operasional yang telah ditentukan dalam masukan yang diberikan oleh pengguna. Adapun implementasi dari fungsi ini dapat dilihat di Lampiran \ref{appdx:A-curl-execute}.

\subsection{\texttt{print\textunderscore curl\textunderscore error()}}
\label{sec:testing-implementation-curl-error}

Fungsi ini akan dipanggil apabila proses curl dari fungsi \verb|execute_curl()| tidak mengembalikan ``OK'' sebagai kode respons cURL-nya. Cara kerja fungsi ini adalah dengan mengecek kode bahasa (variabel \verb|locale|) dan mengeluarkan pesan \textit{error} yang sesuai. Implementasi fungsi ini dapat dilihat di Lampiran \ref{appdx:A-curl-error}.
	
\subsection{\texttt{write\textunderscore memalloc()}}
\label{sec:testing-implementation-write-memalloc}

Fungsi ini adalah fungsi yang bertugas untuk memastikan bahwa data keluaran dari API yang diterima tidak melebihi ukuran maksimal yang diperbolehkan untuk dimasukkan ke dalam variabel tujuan. Implementasi dari fungsi ini dapat dilihat di Lampiran \ref{appdx:A-write-memalloc}.

Penjelasan detail dari proses yang dilakukan dalam fungsi ini adalah sebagai berikut.

\begin{itemize}
	\item Fungsi ini akan memiliki \textit{return value} bertipe \textit{unsigned integer} \verb|size_t|. Hal ini merupakan kewajiban dari\textit{library} cURL sendiri.
	\item Seperti yang sudah disebutkan di bagian rancangan implementasi, fungsi ini akan memiliki empat variabel masukan, yaitu:
	
	\begin{itemize}
		\item \verb|incomingdata| \\
		Variabel ini merupakan data keluaran dari API yang diterima dalam proses cURL.
		\item \verb|size| \\
		Variabel ini merupakan ukuran dari satu buah objek data. Variabel ini selalu bernilai 1, yang juga merupakan kewajiban dari libcurl.
		\item \verb|nmemb| \\
		Variabel ini merupakan ukuran dari data keluaran tersebut.
		\item \verb|userdata| \\
		Variabel ini merupakan penunjuk ke variabel dalam perkakas yang akan diisi oleh data yang diterima proses cURL.
	\end{itemize}
	
	\item Implementasi cara kerja fungsi ini adalah sebagai berikut:
	
	\begin{enumerate}
		\item \textbf{Baris 4}: Hitung ukuran dari data yang masuk.
		\item \textbf{Baris 5\textendash 10}: Cek apakah ukuran data melebihi yang diperbolehkan.
		\item \textbf{Baris 11\textendash 14}: Setor data yang dimasukkan ke dalam variabel tujuan.
		\item \textbf{Baris 16}: Kembalikan ukuran data yang masuk untuk verifikasi keutuhan data.
	\end{enumerate}
\end{itemize}

\subsection{\texttt{write\textunderscore searchplace()}}
\label{sec:testing-implementation-write-searchplace}

Fungsi ini adalah fungsi yang bertugas untuk memproses respons dari API untuk fitur pencarian lokasi dari mode \verb|searchplace|. Penerapan fungsi ini hanya meliputi pengambilan data yang diterima, mengubah format data tersebut dari JSON ke tipe data yang bisa langsung diproses dalam fungsi-fungsi bawaan bahasa C, dan kemudian menampilkan hasilnya ke pengguna. Implementasi dari fungsi ini dapat dilihat di Lampiran \ref{appdx:A-write-searchplace}.
\newpage\vspace*{-3em} % prevent awkward cutoff
\subsection{\texttt{write\textunderscore findroute()}}
\label{sec:testing-implementation-write-findroute}

Fungsi ini adalah fungsi yang bertugas untuk memproses respons dari API untuk fitur pencarian rute angkot. Sama seperti fungsi \verb|write_searchplace()|, penerapan fungsi ini hanya meliputi pengambilan data yang diterima, konversi format data tersebut dari JSON, dan kemudian menampilkan hasilnya ke pengguna. Implementasi dari fungsi ini dapat dilihat di lampiran kode murni perkakas, baris 218 sampai baris 334. Perlu ditekankan kembali bahwa fitur ini memiliki pembetulan \textit{bug} tambahan, yang diimplementasikan di Lampiran \ref{appdx:A-write-findroute}.
\vspace*{-0.5em} % prevent awkward cutoff
\subsection{\texttt{write\textunderscore searchplace\textunderscore noreturns()}}
\label{sec:testing-implementation-write-searchplacenoreturns}

Fungsi ini adalah fungsi yang bertugas untuk memproses respons dari API untuk fitur pencarian lokasi dari mode \verb|direct|. Seperti yang sudah dijelaskan di bagian perancangan alur kerja dari fungsi ini (subbab \ref{sec:design-code-write-searchplacenoreturns}), fungsi ini mirip dengan fungsi \verb|write_searchplace()|, hanya saja untuk keluarannya, fungsi ini tidak akan menampilkan koordinat lokasi yang ditemukan. Implementasi dari  fungsi ini ada di Lampiran \ref{appdx:A-write-searchplacenoreturns}.
\vspace*{-0.5em} % prevent awkward cutoff
\subsection{Fungsi utama (\texttt{main})}
\label{sec:testing-implementation-main}

Fungsi \verb|main| di perkakas ini dapat dibagi menjadi dua buah proses utama, yaitu penerimaan masukan dari pengguna, serta penentuan langkah-langkah yang harus dijalankan untuk tiap-tiap mode operasional. Adapun implementasi dari fungsi utama ini ada di Lampiran \ref{appdx:A-main}, dengan proses-proses internalnya meliputi langkah-langkah berikut.

\begin{enumerate}
	\item \textbf{Baris 2\textendash 183}: Implementasi \verb|getopt-long| dan pentransferan argumen ke dalam variabel-variabel internal perkakas.
	\item \textbf{Baris 3 \& 4}: Inisialisasi \verb|opterr| dengan nilai 0, untuk mencegah \verb|getopt-long| mengeluarkan pesan \textit{error default}-nya untuk opsi-opsi yang tidak diketahui (\textquotesingle\verb|?|\textquotesingle). 
	\item \textbf{Baris 186\textendash 204}: Pengecekan kelebihan argumen dalam perintah masukan.
	\item \textbf{Baris 206\textendash 271}: Penentuan langkah-langkah yang harus ditempuh untuk setiap kemungkinan mode operasional.
	\item \textbf{Baris 273}: \textit{Return code} 0, menandakan bahwa perkakas berhasil berjalan tanpa ada masalah.
\end{enumerate}
\vspace*{-0.5em} % prevent awkward cutoff
\subsection{CMakeLists}
\label{sec:testing-implementation-cmakelists}

Implementasi dari file CMakeLists.txt untuk perkakas ini dapat dilihat di Lampiran \ref{appdx:A-cmakelists}. Adapun cara kerja file CMakeLists ini dapat dibagi menjadi beberapa bagian sebagai berikut:

\begin{enumerate}
	\item \textbf{Baris 1 \& 2}: Pengaturan nama proyek serta file-file utama dari proyeknya.
	\item \textbf{Baris 4}: Pengaturan versi minimum CMake. Versi CMake yang dibutuhkan minimal adalah 3.18, karena fitur \textit{file compression} (untuk instalasi \textit{man page}) baru didukung di versi tersebut.
	\item \textbf{Baris 5}: Pendefinisian nama proyek, versi, serta bahasa pemrograman dari proyek tersebut.
	\item \textbf{Baris 7}: Penghubungan file-file utama proyek ke proyek yang sudah didefinisikan di baris 5.
	\item \textbf{Baris 9\textendash 14}: Pencarian dan pengintegrasian \textit{library-library} yang diperlukan perkakas.
	\item \textbf{Baris 16\textendash 32}: Perintah-perintah spesifik untuk sistem operasi berbasis Linux, yang berhubungan dengan instalasi halaman manual (\textit{man page}). File yang berisi kode murni \textit{man page} akan di-\textit{compress}, dan kemudian diatur untuk dapat diinstal langsung ke direktori yang benar di Linux.
\end{enumerate}
\vspace*{-0.5em} % prevent awkward cutoff
\subsection{Halaman manual (\textit{man page})}
\label{sec:testing-implementation-man}

Kode ini diperlukan untuk fitur halaman manual (man \textit{page}) di sistem operasi berbasis Linux, yang pada dasarnya merupakan bantuan penggunaan perkakas yang lebih panjang dan lebih detail. Implementasi dari perkakas ini ada di Lampiran \ref{appdx:A-manpage}, di mana potongan kode tersebut dapat dibagi menjadi bagian bagian sebagai berikut, berdasarkan sintaks yang mengawali bagian-bagian dalam kode tersebut\footnote{\href{https://www.linuxhowtos.org/System/creatingman.htm}{https://www.linuxhowtos.org/System/creatingman.htm}}, yaitu:

\begin{itemize}
	\item \textbf{.TH} \\
	Judul dari halaman manual. Tanggal dibuat, versi, serta judul dari halaman manualnya sendiri diatur di bagian ini.
	\item \textbf{.SH} \\
	Setiap baris yang diawali dengan sintaks ini akan di-format menjadi judul dari tiap-tiap bab di dalam halaman manual.
	\item Sisa dari file yang tidak diawali dengan kedua perintah di atas merupakan deskripsi dari tiap-tiap bagian yang ada di dalam halaman manual nantinya.
\end{itemize}

\section{Pengujian}
\label{sec:testing-experiments}

Bagian ini akan menjelaskan hal-hal yang seputar pengujian perkakas yang telah dibuat\textemdash lingkungan pengujian, cara instalasi dan penggunaan perkakas, serta pengujian fungsional perkakas.

\subsection{Lingkungan Perangkat Keras}
\label{sec:testing-experiments-hardware}

Berikut merupakan spesifikasi perangkat keras yang digunakan dalam pengujian perkakas ini:

\begin{itemize}
	\item \textit{Processor}: Intel\logoregistered\xspace Core\logotrademark\xspace i5-10300H @ 2.50 GHz
	\item RAM: 8 GB
	\item \textit{Hard disk}: SSD 512 GB (NVMe\logotrademark\xspace M.2)
	\item Perangkat keras pendukung: Keyboard
\end{itemize}

\subsection{Lingkungan Perangkat Lunak}
\label{sec:testing-experiments-software}

Berikut merupakan spesifikasi perangkat lunak yang digunakan dalam pengujian perkakas ini:

\begin{itemize}
	\item Windows:
	
	\begin{itemize}
		\item OS: Windows 10 Home Single Language (64-bit)
		\item \textit{Compiler}: MinGW (GNU GCC\textemdash versi 12.1.0)
		\item \textit{Library}:
		
		\begin{itemize}
			\item curl (versi 7.83.1)
			\item cmake (versi 3.24.1)
		\end{itemize}
		
	\end{itemize}
	
	\item Linux:
	
	\begin{itemize}
		\item OS: Ubuntu Jammy (22.04)
		\item \textit{Compiler}: GNU GCC\textemdash versi 11.3.0
		\item \textit{Library}:
		
		\begin{itemize}
			\item curl (versi 7.81.0)
			\item cmake (versi 3.22.1)
		\end{itemize}
		
	\end{itemize}
	
\end{itemize}

\subsection{Pembangunan dan Instalasi}
\label{sec:testing-experiments-installation}

\subsubsection{Syarat Instalasi}
\label{sec:testing-experiments-installation-requirements}

Instalasi perkakas ini tentunya mengharuskan \textit{library-library} yang telah dibahas untuk diinstal terlebih dahulu. Karena banyaknya perbedaan dari detil-detil yang ada di dalam persyaratan instalasi untuk kedua sistem operasi yang didukung, maka bagian ini akan dibagi dua, menjadi satu bagian per sistem operasi.

\begin{itemize}
	\item Windows \\
	Untuk sistem operasi Windows, pengguna perlu menginstal hal-hal berikut:
	
	\begin{itemize}
		\item vcpkg
		\item cURL
		\item CMake
	\end{itemize}
	\newpage\vspace*{-1.5em} % prevent awkward cutoff
	Perlu diperhatikan bahwa cURL (di Windows) harus diinstal melalui vcpkg\textemdash cURL bawaan dari Windows tidak mengandung \textit{library-library development} sekunder yang dibutuhkan oleh perkakas ini. Di lain hal, instalasi cURL secara manual hanya memungkinkan perkakas cURL-nya sendiri untuk diakses dari mana saja (melalui variabel \textit{environment}), tetapi hal ini tidak berlaku untuk \textit{library development} sekundernya.
	
	\item Linux \\
	Untuk sistem operasi berbasis Linux, pengguna perlu menginstal hal-hal berikut:
	
	\begin{itemize}
		\item cURL
		\item CMake
		\item libcurl4-openssl-dev
		\item GNU Make (opsional)
	\end{itemize}
	
\end{itemize}
\noindent
Perkakas ini juga menggunakan \textit{library} cJSON, tetapi untuk alasan kompatibilitas antar Windows dan Linux, \textit{source code} dari \textit{library} ini langsung diikutkan di dalam perkakasnya sendiri, sehingga tidak perlu diinstal oleh pengguna lagi.
\vspace*{-0.5em} % prevent awkward cutoff
\subsubsection{Cara Instalasi}
\label{sec:testing-implementation-installation-howto}

Untuk memakai perkakasnya sendiri, perkakas ini perlu dibangun dan diinstal terlebih dahulu. Berikut merupakan langkah-langkah yang perlu diambil untuk seluruh proses tersebut.

\begin{enumerate}
	\item Buka \textit{folder} ``build'' di dalam \textit{folder} perkakas.
	\item Buka \textit{terminal/command prompt} di dalam \textit{folder} tersebut.
	\item Sesuai dengan sistem operasi tempat perkakas akan digunakan, ketik dan jalankan perintah berikut di \textit{terminal}:
	
	\begin{itemize}
		\item Windows:
		\begin{verbatim}
cmake -DCMAKE_BUILD_TYPE:STRING=Release -DCMAKE_TOOLCHAIN_FILE="<direktori
file toolchain vcpkg>" -G "<compiler>" ../
		\end{verbatim}
	
		\item Linux:
		\begin{verbatim}
cmake ../
		\end{verbatim}
	\end{itemize}		

	Untuk apa yang harus menggantikan variabel \verb|<compiler>| dapat dilihat dengan \mbox{perintah} \verb|cmake --help|. Daftar \textit{compiler} yang didukung oleh cmake dapat dilihat di bagian \linebreak ``Generators'', dan pengguna tinggal menyesuaikan dengan \textit{compiler} yang telah diinstal sebelumnya.  Ada beberapa hal yang perlu dijelaskan/diperhatikan untuk langkah ini.
	
	\begin{itemize}
		\item Windows
			
		\begin{itemize}
			\item Opsi \verb|-DMAKE_TOOLCHAIN_FILE| merupakan metode pengintegrasian vcpkg untuk proyek CMake. Untuk direktori persisnya (dan sintaks lengkap dari opsi ini) dapat dilihat setelah langkah ``Using vcpkg with MSBuild/Visual Studio'' di halaman panduan instalasi vcpkg.\footnote{\href{https://vcpkg.io/en/getting-started.html}{https://vcpkg.io/en/getting-started.html}}
			\item Direkomendasikan untuk menginstal \textit{compiler} \textbf{MinGW}, karena \textit{compiler} ini sudah mengikutkan salah satu file \textit{header} yang dibutuhkan oleh perkakas ini. Apabila pengguna menggunakan \textit{compiler} ini, variabel \verb|<compiler>| harus diisi dengan ``\verb|MinGW Makefiles|''.
			\item \textbf{Jangan menggunakan \textit{compiler} Visual Studio}, karena \textit{compiler} ini tidak mengandung file \textit{header} C yang dibutuhkan di perkakas ini. Perlu diperhatikan juga bahwa \textit{compiler} Visual Studio ini merupakan nilai \textit{default} dari \verb|<compiler>| untuk sistem operasi Windows, jadi pengguna juga tidak boleh menghilangkan opsi \verb|-G| tersebut begitu saja.
		\end{itemize}
		
		\item Linux \\
		Untuk sistem operasi berbasis Linux, tidak perlu mengatur \textit{compiler}, karena nilai \textit{default} dari variabel \verb|<compiler>| di sistem operasi berbasis Linux (\textbf{Unix Makefiles}) sudah ideal.
	\end{itemize}
	\newpage\vspace*{-1.5em} % prevent widow
	\item Lanjutkan dengan instalasi perkakas.
	
	\begin{itemize}
		\item Windows: \\
			Jalankan perintah berikut.
			\begin{verbatim}
cmake --build .
			\end{verbatim}
		\item Linux: \\
		Jalankan kedua perintah berikut.
			\begin{verbatim}
cmake --build .
cmake --install .
			\end{verbatim}
		Jika \textbf{GNU Make} terinstal di perangkat pengguna, maka kedua perintah ini dapat digantikan dengan perintah berikut.
		\begin{verbatim}
make install
		\end{verbatim}
		Jika terjadi \textit{error permission}, cukup tambahkan perintah \verb|sudo| di depan perintah yang ingin dijalankan.
	\end{itemize}
	
	\item File \textit{executable} akan terletak di dalam \textit{folder} ``build'', dan siap dijalankan.
\end{enumerate}
\vspace*{-0.5em} % prevent widow
\subsection{Pengujian}
\label{sec:testing-experiments-testing}

Pengujian akan dilakukan untuk setiap fitur untuk memeriksa apakah semua fitur perkakas sudah berfungsi sebagaimanamestinya, serta semua kemungkinan \textit{error} yang ada sudah diatasi dengan benar. Perlu ditekankan bahwa pengujian berikut juga akan dilakukan dengan versi panjang dari opsi-opsi yang ada di dalam perintah (misal \verb|-h| diganti menjadi \verb|--help|). Akan tetapi, untuk alasan keringkasan dokumen, kecuali terjadi kegagalan, tes-tes tersebut tidak akan dicatat.

Tabel \ref{tab:testing-experiments-testing-overview} memaparkan jumlah tes yang akan dilakukan. Adapun penjelasan dari apa persisnya yang akan dites (\textit{scope}) untuk setiap objek tes akan dibahas langsung di tiap-tiap bagiannya.

\begin{table}[H]
    \centering
    \begin{tabular}{| c | c | c |}
    \hline
        \textbf{No.} & \textbf{Objek tes} & \textbf{Jumlah tes} \\
    \hline
    \hline
        1 & Sintaks dasar & 7 \\
    \hline
        2 & Mode bantuan & 4 \\
    \hline
        3 & Mode \textit{searchplace} & 9 \\
    \hline
        4 & Mode \textit{findroute} & 8 \\
    \hline
        5 & Mode \textit{direct} & 10 \\
    \hline
        6 & Integrasi perkakas \cl\xspace lain & 6 \\
    \hline
	\end{tabular}
    \caption{Jumlah kategori dan tes yang dilakukan.}
    \label{tab:testing-experiments-testing-overview}
\end{table}
\vspace*{-2.5em} % prevent widow
\subsubsection{Sintaks dasar}
\label{sec:testing-experiments-testing-basic}

Pengujian ini akan dilakukan untuk mengecek apakah perkakas akan merespons terhadap masukan yang sama sekali tidak sesuai dengan apa yang diharapkan oleh perkakas. Beberapa dari kasus-kasus berikut meliputi kemungkinan-kemungkinan kesalahan masukan untuk fitur-fitur yang disediakan perkakas, jadi tes spesifik per fitur nantinya tidak akan mengikutkan pengujian sintaks lagi.

\begin{enumerate}
	\item Perintah tanpa opsi
	\begin{itemize}
		\item Perintah masukan:
		\begin{lstlisting}
kiritool
		\end{lstlisting}
		\item Keluaran yang diharapkan: \\
		Perkakas akan mengeluarkan pesan \textit{error} yang mengingatkan pengguna untuk memasukkan mode operasional perkakas.
		\item Keluaran perkakas:
		\begin{lstlisting}
Error:
Mohon masukkan mode pengunaan perkakas.
		\end{lstlisting}
		\item Status tes: \textbf{Sukses}
	\end{itemize}
	
	\item Perintah dengan satu atau lebih opsi tidak valid
	\begin{itemize}
		\item Perintah masukan:
		\begin{lstlisting}
kiritool --mode searchplace --region bdo --query unpar --language id
		\end{lstlisting}
		\item Keluaran yang diharapkan: \\
		Perkakas akan mengeluarkan pesan \textit{error} yang memberi tahu pengguna bahwa ada opsi yang tidak valid di dalam perintah masukan.
		\item Keluaran perkakas:
		\begin{lstlisting}
Error:
Anda telah memasukkan opsi yang tidak valid.
Mohon periksa kembali penulisan perintah yang anda masukkan.
		\end{lstlisting}
		\item Status tes: \textbf{Sukses}
	\end{itemize}
	
	\item Perintah tanpa argumen di akhir perintah
	\begin{itemize}
		\item Perintah masukan:
		\begin{lstlisting}
kiritool --mode searchplace --region bdo --query unpar --locale
		\end{lstlisting}
		\item Keluaran yang diharapkan: \\
		Perkakas akan mengeluarkan pesan \textit{error} mengenai adanya opsi yang kehilangan argumennya di dalam perintah masukan.
		\item Keluaran perkakas:
		\begin{lstlisting}
Error:
Salah satu dari opsi yang anda masukkan kehilangan argumen yang dibutuhkan.
Mohon periksa kembali penulisan perintah yang anda masukkan.
		\end{lstlisting}
		\item Status tes: \textbf{Sukses}
	\end{itemize}
	
	\item Perintah tanpa argumen di tengah perintah
	\begin{itemize}
		\item Perintah masukan:
		\begin{lstlisting}
kiritool --mode searchplace --region bdo --query --locale id
		\end{lstlisting}
		\item Keluaran yang diharapkan: \\
		Perkakas akan mengeluarkan pesan \textit{error} mengenai kelebihan argumen yang dimasukkan pengguna, serta mendaftarkan argumen apa saja yang berlebih.
		\item Keluaran perkakas:
		\begin{lstlisting}
Error:
Anda telah memasukkan kelebihan argumen: id
Mohon periksa kembali penulisan perintah yang anda masukkan.
		\end{lstlisting}
		\item Status tes: \textbf{Sukses}
		\item Catatan tambahan: \\
		Kasus ini idealnya berakhir dengan pesan \textit{error} yang menyatakan bahwa ada opsi yang kehilangan argumennya. Akan tetapi, akibat batasan teknis, \textit{library} getopt sendiri tidak bisa menangani kasus di mana opsi yang kehilangan argumennya berada di tengah perintah, karena getopt akan menginterpretasikan opsi selanjutnya sebagai argumen dari opsi yang kehilangan argumennya. Satu-satunya solusi yang mungkin adalah mengecek apakah argumen dimulai dengan karakter tanda hubung (`-'), tetapi solusi ini tidak dapat diimplementasikan, karena argumen dari beberapa opsi berpotensi untuk diawali dengan karakter tersebut.
	\end{itemize}
	
	\item Perintah dengan terlalu banyak argumen
	\begin{itemize}
		\item Perintah masukan:
		\begin{lstlisting}
kiritool --mode searchplace --region bdo --query unpar --locale id en
		\end{lstlisting}
		\item Keluaran yang diharapkan: \\
		Perkakas akan mengeluarkan pesan \textit{error} mengenai kelebihan argumen yang dimasukkan pengguna, serta mendaftarkan argumen apa saja yang berlebih.
		\item Keluaran perkakas:
		\begin{lstlisting}
Error:
Anda telah memasukkan kelebihan argumen: en
Mohon periksa kembali penulisan perintah yang anda masukkan.
		\end{lstlisting}
		\item Status tes: \textbf{Sukses}
	\end{itemize}
	
	\item Perintah dengan mode yang tidak valid
	\begin{itemize}
		\item Perintah masukan:
		\begin{lstlisting}
kiritool -m help
		\end{lstlisting}
		\item Keluaran yang diharapkan: \\
		Perkakas akan mengeluarkan pesan \textit{error} yang memberitahu pengguna bahwa mode yang dimasukkan tidak valid.
		\item Keluaran perkakas:
		\begin{lstlisting}
Error:
Anda telah memasukkan mode yang tidak valid.
Mohon periksa kembali apakah mode yang anda masukkan sudah diketik dengan benar.
		\end{lstlisting}
		\item Status tes: \textbf{Sukses}
	\end{itemize}
	
	\item Pengunaan banyak mode sekaligus
	\begin{itemize}
		\item Perintah masukan:
		\begin{lstlisting}
kiritool --mode searchplace --region bdo --query unpar --help --locale en
		\end{lstlisting}
		\item Keluaran yang diharapkan: \\
		Perkakas hanya akan merespons terhadap mode operasional pertama yang dimasukkan (beserta opsi-opsinya).
		\item Keluaran perkakas:
		\begin{lstlisting}
Location:
--------------------
Name: Universitas Katolik Parahyangan
Coordinates: -6.87520,107.60492
		\end{lstlisting}
		\item Status tes: \textbf{Sukses}
	\end{itemize}

\end{enumerate}

\subsubsection{Mode bantuan}
\label{sec:testing-experiments-testing-help}

Pengujian ini akan dilakukan untuk mengecek apakah fungsi-fungsi perkakas yang berhubungan dengan fitur bantuan penggunaan perkakas sudah berfungsi dengan baik.

\begin{enumerate}
	\item Panggilan bantuan normal
	\begin{itemize}
		\item Perintah masukan:
		\begin{lstlisting}
kiritool --help
		\end{lstlisting}
		\item Keluaran yang diharapkan: \\
		Perkakas akan mengeluarkan bantuan penggunaan perkakas.
		\item Keluaran perkakas: \\
		Perkakas akan mengeluarkan bantuan penggunaan perkakas dengan benar. Keluaran dari kasus tes ini dapat dilihat di lampiran \ref{appdx:A-output-help}.
		\item Status tes: \textbf{Sukses}
	\end{itemize}
	
	\item Panggilan bantuan dengan tambahan opsi valid yang tidak relevan
	\begin{itemize}
		\item Perintah masukan:
		\begin{lstlisting}
kiritool --help --start unpar
		\end{lstlisting}
		\item Keluaran yang diharapkan: \\
		Perkakas akan mengeluarkan bantuan penggunaan perkakas, tanpa memedulikan opsi serta argumen tambahan di dalam perintah.
		\item Keluaran perkakas: \\
		Keluaran dari kasus tes ini sama seperti kasus sebelumnya, yang juga dapat dilihat di lampiran \ref{appdx:A-output-help}.
		\item Status tes: \textbf{Sukses}
	\end{itemize}
	
	\item Pemanggilan \textit{manual page} (khusus Linux)
	\begin{itemize}
		\item Perintah masukan:
		\begin{lstlisting}
man kiritool
		\end{lstlisting}
		\item Keluaran yang diharapkan: \\
		\textit{Terminal} akan menampilkan \textit{man page} dari perkakas yang sesuai.
		\item Keluaran perkakas: \\
		\textit{Terminal} menampilkan \textit{man page} yang sesuai. Hasil dari kasus tes ini dapat dilihat di lampiran \ref{appdx:A-output-manpage}. Perlu diingat bahwa \textit{man page} memiliki \textit{formatting} otomatisnya sendiri, jadi ketika dikonversikan ke file .txt (agar bisa dimasukkan sebagai lampiran), beberapa aspek \textit{formatting}-nya menjadi tidak sempurna.
		\item Status tes: \textbf{Sukses}
	\end{itemize}
	
	\item Pemanggilan versi pendek dari \textit{manual page} (\verb|whatis|\textemdash khusus Linux)
	\begin{itemize}
		\item Perintah masukan:
		\begin{lstlisting}
whatis kiritool
		\end{lstlisting}
		\item Keluaran yang diharapkan: \\
		\textit{Terminal} akan menampilkan nama dan deskripsi singkat perkakas, sesuai dengan isi dari bagian \verb|NAME| dalam \textit{man page} perkakas.
		\item Keluaran perkakas:
		\begin{lstlisting}
kiritool (1)         - search routes using angkot
		\end{lstlisting}
		\item Status tes: \textbf{Sukses}
	\end{itemize}

\end{enumerate}

\subsubsection{Mode \textit{searchplace}}
\label{sec:testing-experiments-testing-searchplace}

Pengujian ini akan dilakukan untuk mengecek apakah fungsi-fungsi perkakas yang berhubungan dengan fitur pencarian lokasi sudah berfungsi dengan baik.

\begin{enumerate}
	\item Pencarian lokasi sukses dengan satu hasil
	\begin{itemize}
		\item Perintah masukan:
		\begin{lstlisting}
kiritool --mode searchplace --region bdo --query unpar --locale id
		\end{lstlisting}
		\item Keluaran yang diharapkan: \\
		Perkakas akan menampilkan nama dan koordinat \latlon lokasi, sesuai dengan bahasa yang diminta oleh pengguna.
		\item Keluaran perkakas:
		\begin{lstlisting}
Lokasi:
--------------------
Nama: Universitas Katolik Parahyangan
Koordinat: -6.87520,107.60492
		\end{lstlisting}
		\item Status tes: \textbf{Sukses}
	\end{itemize}
	
	\item Pencarian lokasi sukses dengan lebih dari satu hasil
	\begin{itemize}
		\item Perintah masukan:
		\begin{lstlisting}
kiritool --mode searchplace --region bdo --query ab --locale id
		\end{lstlisting}
		\item Keluaran yang diharapkan: \\
		Perkakas akan menampilkan nama dan koordinat \latlon semua kemungkinan lokasi, sesuai dengan bahasa yang diminta oleh pengguna.
		\item Keluaran perkakas:
		\begin{lstlisting}
Lokasi 1:
--------------------
Nama: Blk. A-B
Koordinat: -6.88612,107.65028

Lokasi 2:
--------------------
Nama: Blk. A-B
Koordinat: -6.90845,107.67641

Lokasi 3:
--------------------
Nama: Blk. AB
Koordinat: -6.89748,107.70782
		\end{lstlisting}
		\item Status tes: \textbf{Sukses}
	\end{itemize}
	
	\item Pencarian lokasi gagal
	\begin{itemize}
		\item Perintah masukan:
		\begin{lstlisting}
kiritool --mode searchplace --region bdo --query abasdasd --locale id
		\end{lstlisting}
		\item Keluaran yang diharapkan: \\
		Perkakas akan mengeluarkan pesan keluaran yang memberitahu pengguna bahwa lokasi tidak berhasil ditemukan.
		\item Keluaran perkakas:
		\begin{lstlisting}
Lokasi tidak berhasil ditemukan.
Silakan cek ulang apakah kata kunci pencarian sudah dimasukkan dengan benar.
		\end{lstlisting}
		\item Status tes: \textbf{Sukses}
	\end{itemize}
	
	\item Pencarian lokasi tanpa opsi region
	\begin{itemize}
		\item Perintah masukan:
		\begin{lstlisting}
kiritool --mode searchplace --query unpar --locale id
		\end{lstlisting}
		\item Keluaran yang diharapkan: \\
		Perkakas akan menampilkan pesan \textit{error} yang mengingatkan pengguna untuk memasukkan kode region.
		\item Keluaran perkakas:
		\begin{lstlisting}
Error:
Fitur pencarian lokasi memerlukan pengaturan region lokasi yang ingin dicari.
Mohon pastikan anda sudah memasukkan salah satu dari empat kode region yang tersedia.
Pilihan region: cgk, bdo, mlg, sub
		\end{lstlisting}
		\item Status tes: \textbf{Sukses}
	\end{itemize}
	
	\item Pencarian lokasi dengan region yang tidak valid
	\begin{itemize}
		\item Perintah masukan:
		\begin{lstlisting}
kiritool --mode searchplace --region bdg --query unpar --locale id
		\end{lstlisting}
		\item Keluaran yang diharapkan: \\
		Perkakas akan menampilkan pesan \textit{error} bahwa region yang dimasukkan tidak valid.
		\item Keluaran perkakas:
		\begin{lstlisting}
Error:
Anda telah memasukkan region yang tidak valid.
Mohon periksa kembali apakah kode region yang anda masukkan merupakan salah satu dari empat kode region yang tersedia.
Pilihan region: cgk, bdo, mlg, sub
		\end{lstlisting}
		\item Status tes: \textbf{Sukses}
	\end{itemize}
	
	\item Pencarian lokasi tanpa kata kunci pencarian
	\begin{itemize}
		\item Perintah masukan:
		\begin{lstlisting}
kiritool --mode searchplace --region bdo --locale id
		\end{lstlisting}
		\item Keluaran yang diharapkan: \\
		Perkakas akan menampilkan pesan \textit{error} yang mengingatkan pengguna untuk memasukkan kata kunci pencarian.
		\item Keluaran perkakas:
		\begin{lstlisting}
Error:
Fitur pencarian lokasi memerlukan sebuah kata kunci pencarian.
Mohon pastikan anda sudah memasukkan kata kunci untuk melakukan pencarian lokasi.
		\end{lstlisting}
		\item Status tes: \textbf{Sukses}
	\end{itemize}
	
	\item Pencarian lokasi dengan kata kunci pencarian yang tidak valid
	\begin{itemize}
		\item Perintah masukan:
		\begin{lstlisting}
kiritool --mode searchplace --region bdo --query -6.87520,107.60492 --locale id
		\end{lstlisting}
		\item Keluaran yang diharapkan: \\
		Perkakas akan menampilkan pesan bahwa keluaran API adalah sebuah \textit{error}.
		\item Keluaran perkakas:
		\begin{lstlisting}
Error:
API mengembalikan error sebagai keluarannya.
Silakan cek ulang apakah kata kunci pencarian sudah diformat dengan benar.
		\end{lstlisting}
		\item Status tes: \textbf{Sukses}
		\item Catatan tambahan: \\
		Perkakas tetap menerima kata kunci pencarian yang diawali dengan tanda hubung (`-'), jadi perkakas tidak bisa langsung mengembalikan \textit{error} ke pengguna apabila pengguna (misal dalam kasus ini) memasukkan koordinat \latlon sebagai argumen opsi \verb|-q|.
	\end{itemize}
	
	\item Pencarian lokasi tanpa pengaturan bahasa
	\begin{itemize}
		\item Perintah masukan:
		\begin{lstlisting}
kiritool --mode searchplace --region bdo --query unpar
		\end{lstlisting}
		\item Keluaran yang diharapkan: \\
		Perkakas tetap berfungsi seperti biasa, dengan mengeluarkan keluaran dalam bahasa Indonesia.
		\item Keluaran perkakas:
		\begin{lstlisting}
Lokasi:
--------------------
Nama: Universitas Katolik Parahyangan
Koordinat: -6.87520,107.60492
		\end{lstlisting}
		\item Status tes: \textbf{Sukses}
	\end{itemize}
	
	\item Pencarian lokasi dengan pengaturan bahasa yang tidak valid
	\begin{itemize}
		\item Perintah masukan:
		\begin{lstlisting}
kiritool --mode searchplace --region bdo --query unpar --locale ch
		\end{lstlisting}
		Perkakas akan mengeluarkan pesan \textit{error} yang memberitahu pengguna bahwa bahasa yang dimasukkan tidak valid.
		\item Keluaran perkakas:
		\begin{lstlisting}
Error:
Anda telah memasukkan pilihan bahasa (locale) yang tidak valid.
Mohon periksa kembali apakah pilihan bahasa yang anda masukkan valid atau tidak.
Pilihan locale: id, en
--------------------
You have inputted an invalid language (locale) option.
Please recheck whether the language code you inserted was supported or not.
Locale available: id, en
		\end{lstlisting}
		\item Status tes: \textbf{Sukses}
	\end{itemize}
	
\end{enumerate}

\subsubsection{Mode \textit{findroute}}
\label{sec:testing-experiments-testing-findroute}

Pengujian ini akan dilakukan untuk mengecek apakah fungsi-fungsi perkakas yang berhubungan dengan fitur pencarian rute angkot sudah berfungsi dengan baik.

\begin{enumerate}
	\item Pencarian rute sukses dengan satu kemungkinan rute
	\begin{itemize}
		\item Perintah masukan:
		\begin{lstlisting}
kiritool --mode findroute --start -6.89350,107.60430 --finish -6.87520,107.60492 --locale id
		\end{lstlisting}
		\item Keluaran yang diharapkan: \\
		Perkakas akan menampilkan estimasi durasi rute, serta langkah-langkah yang perlu ditempuh di dalam rutenya, sesuai dengan bahasa yang diminta oleh pengguna.
		\item Keluaran perkakas:
		\begin{lstlisting}
Rute:
--------------------
Estimasi waktu: 25 menit 
--------------------
Langkah 1: Jalan dari tujuan akhir Anda ke Jalan Cihampelas sejauh kurang lebih 203 meter.
Langkah 2: Naik angkot Ciumbuleuit - St. Hall (belok) di Jalan Cihampelas, dan turun di 40141 kurang lebih setelah 3,9 kilometer.
Langkah 3: Jalan dari 40141 ke lokasi mulai Anda sejauh kurang lebih 64 meter.
		\end{lstlisting}
		\item Status tes: \textbf{Sukses}
	\end{itemize}
	
	\item Pencarian rute sukses dengan satu kemungkinan rute yang sangat jauh
	\begin{itemize}
		\item Perintah masukan:
		\begin{lstlisting}
kiritool --mode findroute --start -6.16935,106.78899 --finish -6.87520,107.60492 --locale id
		\end{lstlisting}
		\item Keluaran yang diharapkan: \\
		Jauhnya rute tidak akan mempengaruhi kinerja dari perkakas.
		\item Keluaran perkakas:
		\begin{lstlisting}
Rute:
--------------------
Estimasi waktu: 3 jam 
--------------------
Langkah 1: Jalan dari tujuan akhir Anda ke Jelambar 1 sejauh kurang lebih 365 meter.
Langkah 2: Naik Daytrans Grogol - Cihampelas di Jelambar 1, dan turun di SMA Pasundan 2 Bandung kurang lebih setelah 155,8 kilometer.
Langkah 3: Jalan dari SMA Pasundan 2 Bandung ke Jalan Cihampelas sejauh kurang lebih 35 meter.
Langkah 4: Naik angkot Ciumbuleuit - St. Hall (belok) di Jalan Cihampelas, dan turun di 40141 kurang lebih setelah 3,9 kilometer.
Langkah 5: Jalan dari 40141 ke lokasi mulai Anda sejauh kurang lebih 64 meter.
		\end{lstlisting}
		\item Status tes: \textbf{Sukses}
	\end{itemize}
	
	\item Pencarian rute sukses dengan lebih dari satu kemungkinan rute
	\begin{itemize}
		\item Perintah masukan:
		\begin{lstlisting}
kiritool --mode findroute --start -6.89350,107.60430 --finish -6.91527,107.59454 --locale id
		\end{lstlisting}
		\item Keluaran yang diharapkan: \\
		Perkakas akan menampilkan setiap rutenya, diikuti dengan estimasi waktunya serta langkah-langkah yang perlu ditempuh di dalam rutenya, sesuai dengan bahasa yang diminta oleh pengguna.
		\item Keluaran perkakas:
		\begin{lstlisting}
Rute 1:
--------------------
Estimasi waktu: 30 menit 
--------------------
Langkah 1: Jalan dari tujuan akhir Anda ke Jalan Cihampelas sejauh kurang lebih 56 meter.
Langkah 2: Naik angkot Cicaheum - Ciroyom di Jalan Cihampelas, dan turun di Jalan Arjuna kurang lebih setelah 4,2 kilometer.
Langkah 3: Jalan sedikit di Jalan Arjuna.
Langkah 4: Naik angkot Dago - Caringin di Jalan Arjuna, dan turun di Jalan Kebon Jati kurang lebih setelah 688 meter.
Langkah 5: Jalan dari Jalan Kebon Jati ke lokasi mulai Anda sejauh kurang lebih 138 meter.

Rute 2:
--------------------
Estimasi waktu: 25 menit 
--------------------
Langkah 1: Jalan dari tujuan akhir Anda ke Jalan Cihampelas sejauh kurang lebih 26 meter.
Langkah 2: Naik angkot Cisitu - Tegallega di Jalan Cihampelas, dan turun di Jalan Pasir Kaliki kurang lebih setelah 3,7 kilometer.
Langkah 3: Jalan dari Jalan Pasir Kaliki ke lokasi mulai Anda sejauh kurang lebih 396 meter.
		\end{lstlisting}
		\item Status tes: \textbf{Sukses}
	\end{itemize}
	
	\item Pencarian rute gagal
	\begin{itemize}
		\item Perintah masukan:
		\begin{lstlisting}
kiritool --mode findroute --start -6.89350,107.60430 --finish -6.88307,107.65529 --locale id
		\end{lstlisting}
		\item Keluaran yang diharapkan: \\
		Perkakas akan mengeluarkan pesan kepada pengguna bahwa rute perjalanan dengan angkot tidak berhasil ditemukan.
		\item Keluaran perkakas:
		\begin{lstlisting}
Maaf, kami tidak dapat menemukan rute transportasi publik untuk perjalanan anda.
		\end{lstlisting}
		\item Status tes: \textbf{Sukses}
	\end{itemize}
	
	\item Pencarian rute tanpa masukan lokasi (2 kasus)
	\begin{itemize}
		\item Perintah masukan:
		\begin{lstlisting}
kiritool --mode findroute --finish -6.87520,107.60492 --locale id
kiritool --mode findroute --start -6.89350,107.60430 --locale id
		\end{lstlisting}
		\item Keluaran yang diharapkan: \\
		Perkakas akan mengeluarkan pesan \textit{error} yang mengingatkan pengguna untuk memasukkan koordinat \latlon lokasi awal atau akhir, tergantung dari lokasi mana yang tidak dimasukkan.
		\item Keluaran perkakas: \\
		\textbf{Kasus 1}: Lokasi awal tidak dimasukkan
		\begin{lstlisting}
Error:
Anda belum memasukkan sebuah koordinat untuk lokasi awal.
Mohon masukkan koordinat untuk lokasi awal pencarian rute melalui opsi yang sesuai.
		\end{lstlisting}
		\textbf{Kasus 2}: Lokasi akhir tidak dimasukkan
		\begin{lstlisting}
Error:
Anda belum memasukkan sebuah koordinat untuk lokasi akhir.
Mohon masukkan koordinat untuk lokasi akhir pencarian rute melalui opsi yang sesuai.
		\end{lstlisting}
		\item Status tes: \textbf{Sukses}
	\end{itemize}
	
	\item Pencarian rute dengan koordinat lokasi yang tidak valid
	\begin{itemize}
		\item Perintah masukan:
		\begin{lstlisting}
kiritool --mode findroute --start -6.89350,107.60430 --finish -6.87520,107.6049a --locale id
		\end{lstlisting}
		\item Keluaran yang diharapkan: \\
		Perkakas akan mengeluarkan pesan bahwa keluaran API adalah sebuah \textit{error}.
		\item Keluaran perkakas:
		\begin{lstlisting}
Error:
API mengembalikan error sebagai keluarannya.
Silakan cek ulang apakah koordinat kedua lokasi sudah dimasukkan dengan benar.
		\end{lstlisting}
		\item Status tes: \textbf{Sukses}
		\item Catatan tambahan: \\
		Akibat penggunaan variabel yang sama dengan variabel dalam fitur \verb|direct|, di mana di dalam fitur tersebut opsi \verb|-s| dan opsi \verb|-f| memiliki aturan masukan yang berbeda, maka fitur ini tidak akan langsung mendeteksi \textit{error} apabila pengguna memasukkan koordinat yang tidak valid.
	\end{itemize}
	
	\item Pencarian lokasi tanpa pengaturan bahasa
	\begin{itemize}
		\item Perintah masukan:
		\begin{lstlisting}
kiritool --mode findroute --start -6.89350,107.60430 --finish -6.87520,107.60492
		\end{lstlisting}
		\item Keluaran yang diharapkan: \\
		Perkakas tetap berfungsi seperti biasa, dengan mengeluarkan keluaran dalam bahasa Indonesia.
		\item Keluaran perkakas:
		\begin{lstlisting}
Rute:
--------------------
Estimasi waktu: 25 menit 
--------------------
Langkah 1: Jalan dari tujuan akhir Anda ke Jalan Cihampelas sejauh kurang lebih 203 meter.
Langkah 2: Naik angkot Ciumbuleuit - St. Hall (belok) di Jalan Cihampelas, dan turun di 40141 kurang lebih setelah 3,9 kilometer.
Langkah 3: Jalan dari 40141 ke lokasi mulai Anda sejauh kurang lebih 64 meter.
		\end{lstlisting}
		\item Status tes: \textbf{Sukses}
	\end{itemize}
	
	\item Pencarian lokasi dengan pengaturan bahasa yang tidak valid
	\begin{itemize}
		\item Perintah masukan:
		\begin{lstlisting}
kiritool --mode findroute --start -6.89350,107.60430 --finish -6.87520,107.60492 --locale ch
		\end{lstlisting}
		\item Keluaran yang diharapkan: \\
		Perkakas akan mengeluarkan pesan \textit{error} yang memberitahu pengguna bahwa bahasa yang dimasukkan tidak valid.
		\item Keluaran perkakas:
		\begin{lstlisting}
Error:
Anda telah memasukkan pilihan bahasa (locale) yang tidak valid.
Mohon periksa kembali apakah pilihan bahasa yang anda masukkan valid atau tidak.
Pilihan locale: id, en
--------------------
You have inputted an invalid language (locale) option.
Please recheck whether the language code you inserted was supported or not.
Locale available: id, en
		\end{lstlisting}
		\item Status tes: \textbf{Sukses}
	\end{itemize}
	
\end{enumerate}

\subsubsection{Mode \textit{direct}}
\label{sec:testing-experiments-testing-directroute}

Pengujian ini akan dilakukan untuk mengecek apakah fungsi-fungsi perkakas yang berhubungan dengan fitur pencarian rute angkot langsung dari kata kunci lokasi awal dan akhir sudah berfungsi dengan baik.

\begin{enumerate}
	\item Pencarian rute langsung sukses dengan satu kemungkinan rute
	\begin{itemize}
		\item Perintah masukan:
		\begin{lstlisting}
kiritool --mode direct --regstart bdo --start ciwalk --regfinish bdo --finish unpar --locale id
		\end{lstlisting}
		\item Keluaran yang diharapkan: \\
		Perkakas akan menampilkan nama lokasi awal dan akhir, estimasi durasi rute, serta langkah-langkah yang perlu ditempuh di dalam rutenya, sesuai dengan bahasa yang diminta oleh pengguna.
		\item Keluaran perkakas:
		\begin{lstlisting}
Lokasi awal: Cihampelas Walk Extention
Lokasi akhir: Universitas Katolik Parahyangan

Rute:
--------------------
Estimasi waktu: 25 menit 
--------------------
Langkah 1: Jalan dari tujuan akhir Anda ke Jalan Cihampelas sejauh kurang lebih 203 meter.
Langkah 2: Naik angkot Ciumbuleuit - St. Hall (belok) di Jalan Cihampelas, dan turun di 40141 kurang lebih setelah 3,9 kilometer.
Langkah 3: Jalan dari 40141 ke lokasi mulai Anda sejauh kurang lebih 64 meter.
		\end{lstlisting}
		\item Status tes: \textbf{Sukses}
	\end{itemize}
	
	\item Pencarian rute langsung sukses dengan satu kemungkinan rute yang sangat jauh
	\begin{itemize}
		\item Perintah masukan:
		\begin{lstlisting}
kiritool --mode direct --regstart cgk --start untar --regfinish bdo --finish unpar --locale id
		\end{lstlisting}
		\item Keluaran yang diharapkan: \\
		Jauhnya rute tidak akan berpengaruh ke keluaran perkakas.
		\item Keluaran perkakas:
		\begin{lstlisting}
Lokasi awal: Universitas Tarumanagara
Lokasi akhir: Universitas Katolik Parahyangan

Rute:
--------------------
Estimasi waktu: 3 jam 
--------------------
Langkah 1: Jalan dari tujuan akhir Anda ke Jelambar 1 sejauh kurang lebih 365 meter.
Langkah 2: Naik Daytrans Grogol - Cihampelas di Jelambar 1, dan turun di SMA Pasundan 2 Bandung kurang lebih setelah 155,8 kilometer.
Langkah 3: Jalan dari SMA Pasundan 2 Bandung ke Jalan Cihampelas sejauh kurang lebih 35 meter.
Langkah 4: Naik angkot Ciumbuleuit - St. Hall (belok) di Jalan Cihampelas, dan turun di 40141 kurang lebih setelah 3,9 kilometer.
Langkah 5: Jalan dari 40141 ke lokasi mulai Anda sejauh kurang lebih 64 meter.
		\end{lstlisting}
		\item Status tes: \textbf{Sukses}
	\end{itemize}
	
	\item Pencarian rute langsung sukses dengan lebih dari satu kemungkinan rute
	\begin{itemize}
		\item Perintah masukan:
		\begin{lstlisting}
kiritool --mode direct --regstart bdo --start ciwalk --regfinish bdo --finish paskal --locale id
		\end{lstlisting}
		\item Keluaran yang diharapkan: \\
		Perkakas akan menampilkan nama lokasi awal dan akhir, serta tiap-tiap kemungkinan rutenya, dimulai dari estimasi durasi, diikuti dengan langkah-langkah yang perlu ditempuh di dalamnya, sesuai dengan bahasa yang diminta oleh pengguna.
		\item Keluaran perkakas:
		\begin{lstlisting}
Lokasi awal: Cihampelas Walk Extention
Lokasi akhir: 23 Paskal Shopping Center

Rute 1:
--------------------
Estimasi waktu: 30 menit 
--------------------
Langkah 1: Jalan dari tujuan akhir Anda ke Jalan Cihampelas sejauh kurang lebih 56 meter.
Langkah 2: Naik angkot Cicaheum - Ciroyom di Jalan Cihampelas, dan turun di Jalan Arjuna kurang lebih setelah 4,2 kilometer.
Langkah 3: Jalan sedikit di Jalan Arjuna.
Langkah 4: Naik angkot Dago - Caringin di Jalan Arjuna, dan turun di Jalan Kebon Jati kurang lebih setelah 688 meter.
Langkah 5: Jalan dari Jalan Kebon Jati ke lokasi mulai Anda sejauh kurang lebih 138 meter.

Rute 2:
--------------------
Estimasi waktu: 25 menit 
--------------------
Langkah 1: Jalan dari tujuan akhir Anda ke Jalan Cihampelas sejauh kurang lebih 26 meter.
Langkah 2: Naik angkot Cisitu - Tegallega di Jalan Cihampelas, dan turun di Jalan Pasir Kaliki kurang lebih setelah 3,7 kilometer.
Langkah 3: Jalan dari Jalan Pasir Kaliki ke lokasi mulai Anda sejauh kurang lebih 396 meter.
		\end{lstlisting}
		\item Status tes: \textbf{Sukses}
	\end{itemize}
	
	\item Pencarian rute langsung gagal
	\begin{itemize}
		\item Perintah masukan:
		\begin{lstlisting}
kiritool --mode direct --regstart bdo --start ciwalk --regfinish bdo --finish paku --locale id
		\end{lstlisting}
		\item Keluaran yang diharapkan: \\
		Perkakas akan menampilkan nama lokasi awal dan akhir, tetapi akan mengeluarkan pesan bahwa rute tidak berhasil ditemukan.
		\item Keluaran perkakas:
		\begin{lstlisting}
Lokasi awal: Cihampelas Walk Extention
Lokasi akhir: Curug Paku


Maaf, kami tidak dapat menemukan rute transportasi publik untuk perjalanan anda.
		\end{lstlisting}
		\item Status tes: \textbf{Sukses}
	\end{itemize}
	
	\item Pencarian rute langsung tanpa region lokasi (2 kasus)
	\begin{itemize}
		\item Perintah masukan:
		\begin{lstlisting}
kiritool --mode direct --start ciwalk --regfinish bdo --finish unpar --locale id
kiritool --mode direct --regstart bdo --start ciwalk --finish unpar --locale id
		\end{lstlisting}
		\item Keluaran yang diharapkan: \\
		Perkakas akan menampilkan nama lokasi hingga lokasi yang tidak diberikan regionnya, dan kemudian akan mengeluarkan pesan \textit{error} yang memberitahu pengguna bahwa ada kata kunci pencarian lokasi yang hilang.
		\item Keluaran perkakas: \\
		\textbf{Kasus 1}: Region awal tidak dimasukkan
		\begin{lstlisting}
Error:
Fitur pencarian lokasi memerlukan pengaturan region lokasi yang ingin dicari.
Mohon pastikan anda sudah memasukkan salah satu dari empat kode region yang tersedia.
Pilihan region: cgk, bdo, mlg, sub
		\end{lstlisting}
		\textbf{Kasus 2}: Region akhir tidak dimasukkan
		\begin{lstlisting}
Lokasi awal: Cihampelas Walk Extention

Error:
Fitur pencarian lokasi memerlukan pengaturan region lokasi yang ingin dicari.
Mohon pastikan anda sudah memasukkan salah satu dari empat kode region yang tersedia.
Pilihan region: cgk, bdo, mlg, sub
		\end{lstlisting}
		\item Status tes: \textbf{Sukses}
	\end{itemize}
	
	\item Pencarian rute langsung dengan region lokasi yang tidak valid (2 kasus)
	\begin{itemize}
		\item Perintah masukan:
		\begin{lstlisting}
kiritool --mode direct --regstart bdg --start ciwalk --regfinish bdo --finish unpar --locale id
kiritool --mode direct --regstart bdo --start ciwalk --regfinish bdg --finish unpar --locale id
		\end{lstlisting}
		\item Keluaran yang diharapkan: \\
		Perkakas akan menampilkan nama lokasi hingga lokasi yang tidak diberikan regionnya, dan kemudian akan mengeluarkan pesan \textit{error} yang memberitahu pengguna bahwa ada kata kunci pencarian lokasi yang hilang.
		\item Keluaran perkakas: \\
		\textbf{Kasus 1}: Region awal tidak valid
		\begin{lstlisting}
Error:
Anda telah memasukkan region yang tidak valid.
Mohon periksa kembali apakah kode region yang anda masukkan merupakan salah satu dari empat kode region yang tersedia.
Pilihan region: cgk, bdo, mlg, sub
		\end{lstlisting}
		\textbf{Kasus 2}: Region akhir tidak valid
		\begin{lstlisting}
Lokasi awal: Cihampelas Walk Extention

Error:
Anda telah memasukkan region yang tidak valid.
Mohon periksa kembali apakah kode region yang anda masukkan merupakan salah satu dari empat kode region yang tersedia.
Pilihan region: cgk, bdo, mlg, sub
		\end{lstlisting}
		\item Status tes: \textbf{Sukses}
	\end{itemize}
	
	\item Pencarian rute langsung tanpa kata kunci pencarian lokasi (2 kasus)
	\begin{itemize}
		\item Perintah masukan:
		\begin{lstlisting}
kiritool --mode direct --regstart bdo --regfinish bdo --finish unpar --locale id
kiritool --mode direct --regstart bdo --start ciwalk --regfinish bdo --locale id
		\end{lstlisting}
		\item Keluaran yang diharapkan: \\
		Perkakas akan menampilkan nama lokasi hingga lokasi yang tidak diberikan kata kunci pencariannya, dan kemudian akan mengeluarkan pesan \textit{error} yang memberitahu pengguna bahwa ada kata kunci pencarian lokasi yang hilang.
		\item Keluaran perkakas: \\
		\textbf{Kasus 1}: Lokasi awal tidak dimasukkan
		\begin{lstlisting}
Error:
Fitur pencarian lokasi memerlukan sebuah kata kunci pencarian.
Mohon pastikan anda sudah memasukkan kata kunci untuk melakukan pencarian lokasi.
		\end{lstlisting}
		\textbf{Kasus 2}: Lokasi akhir tidak dimasukkan
		\begin{lstlisting}
Lokasi awal: Cihampelas Walk Extention

Error:
Fitur pencarian lokasi memerlukan sebuah kata kunci pencarian.
Mohon pastikan anda sudah memasukkan kata kunci untuk melakukan pencarian lokasi.
		\end{lstlisting}
		\item Status tes: \textbf{Sukses}
	\end{itemize}
	
	\item Pencarian rute langsung dengan kata kunci pencarian lokasi yang tidak valid (2 kasus)
	\begin{itemize}
		\item Perintah masukan:
		\begin{lstlisting}
kiritool --mode direct --regstart bdo --start -6.89350,107.60430 --regfinish bdo --finish unpar --locale id
kiritool --mode direct --regstart bdo --start ciwalk --regfinish bdo --finish -6.87520,107.60492 --locale id
		\end{lstlisting}
		\item Keluaran yang diharapkan: \\
		Perkakas akan menampilkan nama lokasi hingga lokasi yang tidak diberikan kata kunci pencariannya, dan kemudian akan mengeluarkan pesan yang memberitahu pengguna bahwa API mengembalikan sebuah \textit{error}.
		\item Keluaran perkakas: \\
		\textbf{Kasus 1}: Lokasi awal tidak valid
		\begin{lstlisting}
Error:
API mengembalikan error sebagai koordinat lokasi awal.
Silakan cek ulang apakah koordinat lokasi awal sudah dimasukkan dengan benar.
		\end{lstlisting}
		\textbf{Kasus 2}: Lokasi akhir tidak valid
		\begin{lstlisting}
Lokasi awal: Cihampelas Walk Extention

Error:
API mengembalikan error sebagai koordinat lokasi akhir.
Silakan cek ulang apakah koordinat lokasi akhir sudah dimasukkan dengan benar.
		\end{lstlisting}
		\item Status tes: \textbf{Sukses}
	\end{itemize}
	
	\item Pencarian rute langsung sukses tanpa pengaturan bahasa
	\begin{itemize}
		\item Perintah masukan:
		\begin{lstlisting}
kiritool --mode direct --regstart bdo --start ciwalk --regfinish bdo --finish unpar
		\end{lstlisting}
		\item Keluaran yang diharapkan: \\
		Perkakas tetap berfungsi seperti biasa, dengan mengeluarkan keluaran dalam bahasa Indonesia.
		\item Keluaran perkakas:
		\begin{lstlisting}
Lokasi awal: Cihampelas Walk Extention
Lokasi akhir: Universitas Katolik Parahyangan

Rute:
--------------------
Estimasi waktu: 25 menit
--------------------
Langkah 1: Jalan dari tujuan akhir Anda ke Jalan Cihampelas sejauh kurang lebih 203 meter.
Langkah 2: Naik angkot Ciumbuleuit - St. Hall (belok) di Jalan Cihampelas, dan turun di 40141 kurang lebih setelah 3,9 kilometer.
Langkah 3: Jalan dari 40141 ke lokasi mulai Anda sejauh kurang lebih 64 meter.
		\end{lstlisting}
		\item Status tes: \textbf{Sukses}
	\end{itemize}
	
	\item Pencarian rute langsung sukses dengan pengaturan bahasa yang tidak valid
	\begin{itemize}
		\item Perintah masukan:
		\begin{lstlisting}
kiritool --mode direct --regstart bdo --start ciwalk --regfinish bdo --finish unpar --locale ch
		\end{lstlisting}
		\item Keluaran yang diharapkan: \\
		Perkakas akan mengeluarkan pesan \textit{error} yang memberitahu pengguna bahwa bahasa yang dimasukkan tidak valid.
		\item Keluaran perkakas:
		\begin{lstlisting}
Error:
Anda telah memasukkan pilihan bahasa (locale) yang tidak valid.
Mohon periksa kembali apakah pilihan bahasa yang anda masukkan valid atau tidak.
Pilihan locale: id, en
--------------------
You have inputted an invalid language (locale) option.
Please recheck whether the language code you inserted was supported or not.
Locale available: id, en
		\end{lstlisting}
		\item Status tes: \textbf{Sukses}
	\end{itemize}
	
\end{enumerate}

\subsubsection{Integrasi perkakas \cl\xspace lain}
\label{sec:testing-experiments-testing-integration}

Pengujian ini dilakukan untuk menguji kompatibilitas keluaran perkakas dengan fungsi dari perkakas-perkakas \cl\xspace bawaan yang ada. Perkakas yang telah dibuat akan memiliki keluaran berupa teks biasa, yang berarti bahwa keluaran perkakas seharusnya dapat digunakan langsung sebagai masukan perkakas-perkakas \cl\xspace lainnya, seperti \textit{pipeline} (\verb|>|) dan \verb|findstr|/\verb|grep|. Sebelum tes ini dilakukan, perlu ditetapkan beberapa aturan dasar untuk pengujian ini, yaitu sebagai berikut:

\begin{itemize}
	\item Tes kasus yang akan digunakan sebagai masukan adalah pencarian rute langsung dengan satu kemungkinan rute, dari ``ciwalk'' (Bandung) ke ``unpar'' (Bandung), dalam bahasa Inggris. Masukan ini akan disamakan untuk seluruh bagian ini untuk alasan konsistensi.
	\item Pengujian akan dilakukan sebanyak dua kali untuk tiap kasus, untuk Windows dan Linux.
	\item Pengujian hanya akan dilakukan sekali untuk kasus-kasus tertentu apabila fungsi yang diuji memiliki perintah yang sama, atau fungsi tersebut tidak didukung oleh perkakas \cl\xspace bawaan dalam sistem operasi tersebut.
\end{itemize}
\newpage \noindent
Berikut merupakan hasil pengujian untuk bagian ini.
\begin{enumerate}	
	\item Memasukkan keluaran perkakas ke dalam file output
	\begin{itemize}
		\item Perintah masukan:
		\begin{lstlisting}
kiritool --mode direct --regstart bdo --start ciwalk --regfinish bdo --finish unpar --locale en > out.txt
		\end{lstlisting}
		\item Keluaran yang diharapkan: \\
		Perkakas akan menuliskan keluaran dari perintah tersebut langsung di dalam file \verb|out.txt|.
		\item Keluaran perkakas: Akan dihasilkan sebuah file \verb|out.txt| yang isinya adalah sebagai berikut.
		\begin{lstlisting}
Starting location: Cihampelas Walk Extention
Finish location: Universitas Katolik Parahyangan

Route:
--------------------
Estimated duration: 25 minutes
--------------------
Step 1: Walk about 203 meter from your starting point to Jalan Cihampelas.
Step 2: Take angkot Ciumbuleuit - St. Hall (belok) at Jalan Cihampelas, and alight at 40141 about 3.9 kilometer later.
Step 3: Walk about 64 meter from 40141 to your destination.
		\end{lstlisting}
		\item Status tes: \textbf{Sukses}
	\end{itemize}
	
	\item Mengeluarkan hanya langkah-langkah yang perlu ditempuh (Windows)
	\begin{itemize}
		\item Perintah masukan:
		\begin{lstlisting}
kiritool --mode direct --regstart bdo --start ciwalk --regfinish bdo --finish unpar --locale en | findstr /i
"step"
		\end{lstlisting}
		\item Keluaran yang diharapkan: \\
		Perkakas akan menampilkan hanya baris yang memaparkan durasi rute yang ingin ditempuh. Jika tes berhasil, keluaran untuk Windows maupun Linux (tes selanjutnya) akan sama.
		\item Keluaran perkakas:
		\begin{lstlisting}
Step 1: Walk about 203 meter from your starting point to Jalan Cihampelas.
Step 2: Take angkot Ciumbuleuit - St. Hall (belok) at Jalan Cihampelas, and alight at 40141 about 3.9 kilometer later.
Step 3: Walk about 64 meter from 40141 to your destination.
		\end{lstlisting}
		\item Status tes: \textbf{Sukses}
	\end{itemize}
	
	\item Mengeluarkan hanya langkah-langkah yang perlu ditempuh (Linux)
	\begin{itemize}
		\item Perintah masukan:
		\begin{lstlisting}
kiritool --mode direct --regstart bdo --start ciwalk --regfinish bdo --finish unpar --locale en | grep -i
"step"
		\end{lstlisting}
		\item Keluaran yang diharapkan: \\
		Perkakas akan menampilkan hanya baris yang memaparkan durasi rute yang ingin ditempuh. Jika tes berhasil, keluaran untuk Windows (tes sebelumnya) maupun Linux akan sama.
		\item Keluaran perkakas:
		\begin{lstlisting}
Step 1: Walk about 203 meter from your starting point to Jalan Cihampelas.
Step 2: Take angkot Ciumbuleuit - St. Hall (belok) at Jalan Cihampelas, and alight at 40141 about 3.9 kilometer later.
Step 3: Walk about 64 meter from 40141 to your destination.
		\end{lstlisting}
		\item Status tes: \textbf{Sukses}
	\end{itemize}
	
	\item Mengeluarkan hanya durasi dari rute (Windows)
	\begin{itemize}
		\item Perintah masukan:
		\begin{lstlisting}
kiritool --mode direct --regstart bdo --start ciwalk --regfinish bdo --finish unpar --locale en | findstr /i
"duration"
		\end{lstlisting}
		\item Keluaran yang diharapkan: \\
		Perkakas akan menampilkan hanya baris yang memaparkan durasi rute yang ingin ditempuh. Jika tes berhasil, keluaran untuk Windows maupun Linux (tes selanjutnya) akan sama.
		\item Keluaran perkakas:
		\begin{lstlisting}
Estimated duration: 25 minutes
		\end{lstlisting}
		\item Status tes: \textbf{Sukses}
	\end{itemize}
	
	\item Mengeluarkan hanya durasi dari rute (Linux)
	\begin{itemize}
		\item Perintah masukan:
		\begin{lstlisting}
kiritool --mode direct --regstart bdo --start ciwalk --regfinish bdo --finish unpar --locale en | grep -i
"duration"
		\end{lstlisting}
		\item Keluaran yang diharapkan: \\
		Perkakas akan menampilkan hanya baris yang memaparkan durasi rute yang ingin ditempuh. Jika tes berhasil, keluaran untuk Windows (tes sebelumnya) maupun Linux akan sama.
		\item Keluaran perkakas:
		\begin{lstlisting}
Estimated duration: 25 minutes
		\end{lstlisting}
		\item Status tes: \textbf{Sukses}
	\end{itemize}
	
	\item Menampilkan jumlah langkah yang perlu ditempuh (Linux)
	\begin{itemize}
		\item Perintah masukan:
		\begin{lstlisting}
kiritool --mode direct --regstart bdo --start ciwalk --regfinish bdo --finish unpar --locale en | grep -i "step"
| wc -l
		\end{lstlisting}
		\item Keluaran yang diharapkan: \\
Ada tiga langkah dalam rute yang diuji, jadi perkakas akan mengeluarkan `3'.
		\item Keluaran perkakas:
		\begin{lstlisting}
3
		\end{lstlisting}
		\item Status tes: \textbf{Sukses}
		\item Catatan tambahan: \\
		Tes ini hanya berlaku untuk sistem operasi berbasis Linux, karena tidak ada perintah dasar bawaan dalam \cl\xspace Windows yang dapat menghitung jumlah baris dalam suatu keluaran. Tes ini juga tidak dapat dilakukan ke pencarian rute yang menghasilkan lebih dari satu hasil.
	\end{itemize}
	
\end{enumerate}
