\chapter{Implementasi dan Pengujian}
\label{chap:testing}

Bab ini akan membahas hal-hal mengenai implementasi kode dari tiap-tiap fungsi dalam perkakas, serta skenario-skenario pengujian yang akan digunakan dalam pengujian fungsional perkakas.

\section{Implementasi Kode}
\label{sec:testing-implementation}

Subbab ini akan memaparkan implementasi dari fungsi-fungsi yang ada dalam perkakas secara singkat tetapi menyeluruh, serta membahas secara detail sebagian kecil dari fungsi-fungsi tersebut yang memerlukan penjelasan lebih lanjut.

\subsection{\texttt{print\char`_help()}}
\label{sec:testing-implementation-printhelp}

Fungsi ini akan menampilkan \textit{string-string} (dalam bentuk \textit{array-array}karakter) yang merupakan versi singkat dari bantuan penggunaan perkakas ke pengguna. Implementasi dari fungsi ini ada di lampiran \ref{appdx:A-maincode}, mulai dari baris ke-336 sampai dengan baris ke-394.

\subsection{\texttt{replace\char`_space()}}
\label{sec:testing-implementation-replacespace}

Seperti yang telah dijelaskan di subbab \ref{sec:design-code-replacespace}, fungsi ini akan menerima sebuah \textit{array} karakter, mengganti semua karakter spasi yang ada di dalamnya menjadi `\%20', dan meletakkan hasilnya di variabel global \verb|escape|. Implementasi dari fungsi ini dapat dilihat di \textit{source code} murni perkakas dalam lampiran \ref{appdx:A-maincode}, mulai dari baris ke-623 sampai dengan baris ke-643.

\subsection{\texttt{build\char`_url\char`_searchplace()}}
\label{sec:testing-implementation-buildurl-searchplace}

Fungsi ini bekerja dengan menambahkan URL ke variabel global \verb|url|, sesuai dengan opsi-opsi yang telah dimasukkan (dan diperlukan oleh API) untuk fitur \textit{searchplace}. Adapun implementasi dari fungsi ini dapat dilihat di lampiran \ref{appdx:A-maincode}, di baris 455 sampai baris 523.

\subsection{\texttt{build\char`_url\char`_findroute()}}
\label{sec:testing-implementation-buildurl-findroute}

Fungsi ini bekerja dengan menambahkan URL ke variabel global \verb|url|, sesuai dengan opsi-opsi yang telah dimasukkan (dan diperlukan oleh API) untuk fitur \textit{findroute}. Adapun implementasi dari fungsi ini dapat dilihat di lampiran \ref{appdx:A-maincode}, di baris 547 sampai baris 616.

\subsection{\texttt{reset\char`_url()}}
\label{sec:testing-implementation-buildurl-reset}

Fungsi ini hanya terdiri atas satu baris kode yang akan mengembalikan isi dari variabel \verb|url| ke nilai awalnya. Implementasi dari fungsi ini dapat dilihat langsung di lampiran \ref{appdx:A-maincode}, di baris 618 sampai 621.
	
\subsection{\texttt{execute\char`_curl()}}
\label{sec:testing-implementation-curl-execute}

Fungsi ini merupakan fungsi yang mengatur seluruh proses yang berhubungan langsung dengan cURL dalam perkakas, mulai dari inisialisasi \textit{easy handle} cURL (serta proses pembersihannya di akhir fungsi), pengaturan penerimaan data keluaran proses cURL (\verb|write_memalloc()|), pengaturan variabel apa dalam perkakas yang akan diisi oleh data keluaran, serta \textit{error handler} apabila proses cURL gagal. Selain itu, fungsi ini juga mengatur proses apa yang harus dilakukan setelah data tersebut berhasil diterima, dengan cara memanggil fungsi yang sesuai dengan mode operasional yang telah ditentukan dalam masukan yang diberikan oleh pengguna. Adapun implementasi dari fungsi ini dapat dilihat di lampiran \ref{appdx:A-maincode} dari baris 410 ke baris 453.

\subsection{\texttt{print\char`_curl\char`_error()}}
\label{sec:testing-implementation-curl-error}

Fungsi ini akan dipanggil apabila proses curl dari fungsi \verb|execute_curl()| tidak mengembalikan ``OK'' sebagai kode respons cURLnya. Cara kerja fungsi ini adalah dengan mengecek kode bahasa (variabel \verb|locale|) dan mengeluarkan pesan \textit{error} yang sesuai. Implementasi fungsi ini dapat dilihat di lampiran \ref{appdx:A-maincode}, di baris 396 sampai dengan 408.
	
\subsection{\texttt{write\char`_memalloc()}}
\label{sec:testing-implementation-write-memalloc}

Fungsi ini adalah fungsi yang bertugas untuk memastikan bahwa data keluaran dari API yang diterima tidak melebihi ukuran maksimal yang diperbolehkan untuk dimasukkan ke dalam variabel tujuan. Implementasi dari fungsi ini dapat dilihat di lampiran \ref{appdx:A-maincode}, di baris ke-33 sampai dengan baris ke-47.

Penjelasan detail dari proses yang dilakukan dalam fungsi ini adalah sebagai berikut.

\begin{itemize}
	\item Fungsi ini akan memiliki \textit{return value} bertipe \textit{unsigned integer} \verb|size_t|. Hal ini merupakan kewajiban dari\textit{library} cURL sendiri.
	\item Seperti yang sudah disebutkan di bagian rancangan implementasi, fungsi ini akan memiliki empat variabel masukan, yaitu:
	
	\begin{itemize}
		\item \verb|incomingdata| \\
		Variabel ini merupakan data keluaran dari API yang diterima dalam proses cURL.
		\item \verb|size| \\
		Variabel ini merupakan ukuran dari satu buah objek data. Variabel ini selalu bernilai 1, yang juga merupakan kewajiban dari \textit{library} cURL.
		\item \verb|nmemb| \\
		Variabel ini merupakan ukuran dari data keluaran tersebut.
		\item \verb|userdata| \\
		Variabel ini merupakan penunjuk ke variabel dalam perkakas yang akan diisi oleh data yang diterima proses cURL.
	\end{itemize}
	
	\item Implementasi cara kerja fungsi ini adalah sebagai berikut:
	
	\begin{enumerate}
		\item \textbf{Baris 34}: Hitung ukuran dari data yang masuk.
		\item \textbf{Baris 35\textendash 40}: Cek apakah ukuran data melebihi yang diperbolehkan.
		\item \textbf{Baris 41\textendash 44}: Setor data yang dimasukkan ke dalam variabel tujuan.
		\item \textbf{Baris 46}: Kembalikan ukuran data yang masuk untuk verifikasi keutuhan data.
	\end{enumerate}
\end{itemize}

\subsection{\texttt{write\char`_searchplace()}}
\label{sec:testing-implementation-write-searchplace}

Fungsi ini adalah fungsi yang bertugas untuk memproses respons dari API untuk fitur pencarian lokasi dari mode \verb|searchplace|. Penerapan fungsi ini hanya meliputi pengambilan data yang diterima, mengubah format data tersebut dari JSON ke tipe data yang bisa langsung diproses dalam fungsi-fungsi bawaan bahasa C, dan kemudian menampilkan hasilnya ke pengguna. Implementasi dari fungsi ini dapat dilihat di lampiran kode murni perkakas, baris 49 sampai baris 127.
	
\subsection{\texttt{write\char`_findroute()}}
\label{sec:testing-implementation-write-findroute}

Fungsi ini adalah fungsi yang bertugas untuk memproses respons dari API untuk fitur pencarian rute angkot. Sama seperti fungsi \verb|write_searchplace()|, penerapan fungsi ini hanya meliputi pengambilan data yang diterima, konversi format data tersebut dari JSON, dan kemudian menampilkan hasilnya ke pengguna. Implementasi dari fungsi ini dapat dilihat di lampiran kode murni perkakas, baris 218 sampai baris 334.

Perlu ditekankan kembali bahwa fitur ini memiliki pembetulan \textit{bug} tambahan, yang diimplementasikan di baris 273 sampai dengan baris 292.

\subsection{\texttt{write\char`_searchplace\char`_noreturns()}}
\label{sec:testing-implementation-write-searchplacenoreturns}

Fungsi ini adalah fungsi yang bertugas untuk memproses respons dari API untuk fitur pencarian lokasi dari mode \verb|direct|. Seperti yang sudah dijelaskan di bagian perancangan alur kerja dari fungsi ini (subbab \ref{sec:design-code-write-searchplacenoreturns}), fungsi ini mirip dengan fungsi \verb|write_searchplace()|, hanya saja untuk keluarannya, fungsi ini tidak akan menampilkan koordinat lokasi yang ditemukan. Implementasi dari  fungsi ini ada di lampiran \ref{appdx:A-maincode}, di baris 129 sampai dengan 216.

\subsection{Fungsi utama (\texttt{main})}
\label{sec:testing-implementation-main}

Fungsi \verb|main| di perkakas ini dapat dibagi menjadi dua buah proses utama, yaitu penerimaan masukan dari pengguna, serta penentuan langkah-langkah yang harus dijalankan untuk tiap-tiap mode operasional. Adapun implementasi dari fungsi utama ini ada di baris 645 sampai dengan 918, dengan proses-proses internalnya meliputi langkah-langkah berikut.

\begin{enumerate}
	\item \textbf{Baris 646\textendash 827}: Implementasi \verb|getopt-long| dan pentransferan argumen ke dalam variabel-variabel internal perkakas.
	\item \textbf{Baris 647 \& 648}: Inisialisasi \verb|opterr| dengan nilai 0, untuk mencegah \verb|getopt-long| mengeluarkan pesan \textit{error default}-nya untuk opsi-opsi yang tidak diketahui (\textquotesingle\verb|?|\textquotesingle). 
	\item \textbf{Baris 830\textendash 848}: Pengecekan kelebihan argumen dalam perintah masukan.
	\item \textbf{Baris 850\textendash 915}: Penentuan langkah-langkah yang harus ditempuh untuk setiap kemungkinan mode operasional.
	\item \textbf{Baris 917}: \textit{Return code} 0, menandakan bahwa perkakas berhasil berjalan tanpa ada masalah.
\end{enumerate}

\section{Pengujian}
\label{sec:testing-experiments}

Bagian ini akan menjelaskan hal-hal yang seputar pengujian perkakas yang telah dibuat\textemdash lingkungan pengujian, cara instalasi dan penggunaan perkakas, serta pengujian fungsional melalui berbagai macam kasus tes (\textit{test case}).

\subsection{Lingkungan Perangkat Keras}
\label{sec:testing-experiments-hardware}

Berikut merupakan spesifikasi perangkat keras yang digunakan dalam pengujian perkakas ini:

\begin{itemize}
	\item \textit{Processor}: Intel\logoregistered\xspace Core\logotrademark\xspace i5-10300H @ 2.50 GHz
	\item RAM: 8 GB
	\item \textit{Hard disk}: SSD 512 GB (NVMe\logotrademark\xspace M.2)
	\item Perangkat keras pendukung: Keyboard
\end{itemize}

\subsection{Lingkungan Perangkat Lunak}
\label{sec:testing-experiments-software}

Berikut merupakan spesifikasi perangkat lunak yang digunakan dalam pengujian perkakas ini:

\begin{itemize}
	\item Windows:
	
	\begin{itemize}
		\item OS: Windows 10 Home Single Language (64-bit)
		\item \textit{Compiler}: MinGW (GNU GCC\textemdash versi 12.1.0)
		\item \textit{Library}:
		
		\begin{itemize}
			\item curl (versi 7.83.1)
			\item cmake (versi 3.24.1)
		\end{itemize}
		
	\end{itemize}
	
	\item Linux:
	
	\begin{itemize}
		\item OS: Ubuntu Jammy (22.04)
		\item \textit{Compiler}: GNU GCC\textemdash versi 11.3.0
		\item \textit{Library}:
		
		\begin{itemize}
			\item curl (versi 7.81.0)
			\item cmake (versi 3.22.1)
		\end{itemize}
		
	\end{itemize}
	
\end{itemize}

\subsection{Pembangunan dan Instalasi}
\label{sec:testing-experiments-installation}

\subsubsection{Syarat Instalasi}
\label{sec:testing-experiments-installation-requirements}

Instalasi perkakas ini tentunya mengharuskan \textit{library-library} yang telah dibahas untuk diinstal terlebih dahulu. Karena banyaknya perbedaan dari detil-detil yang ada di dalam persyaratan instalasi untuk kedua sistem operasi yang didukung, maka bagian ini akan dibagi dua, menjadi satu bagian per sistem operasi.

\begin{itemize}
	\item Windows \\
	Untuk sistem operasi Windows, pengguna perlu menginstal hal-hal berikut:
	
	\begin{itemize}
		\item vcpkg
		\item cURL
		\item CMake
	\end{itemize}
	
	Perlu diperhatikan bahwa cURL (di Windows) harus diinstal melalui vcpkg\textemdash cURL bawaan dari Windows tidak mengandung \textit{library-library development} sekunder yang dibutuhkan oleh perkakas ini. Di lain hal, instalasi cURL secara manual hanya memungkinkan perkakas cURL-nya sendiri untuk diakses dari mana saja (melalui variabel \textit{environment}), tetapi hal ini tidak berlaku untuk \textit{library development} sekundernya.
	
	\item Linux \\
	Untuk sistem operasi berbasis Linux, pengguna perlu menginstal hal-hal berikut:
	
	\begin{itemize}
		\item cURL
		\item CMake
		\item libcurl4-openssl-dev
		\item GNU Make (opsional)
	\end{itemize}
	
\end{itemize}
\noindent
Ingat bahwa perkakas ini juga menggunakan \textit{library} cJSON, tetapi untuk alasan kompatibilitas antar Windows dan Linux, \textit{source code} dari \textit{library} ini langsung diikutkan di dalam perkakasnya sendiri, sehingga tidak perlu diinstal oleh pengguna lagi.

\subsubsection{Cara Instalasi}
\label{sec:testing-implementation-installation-howto}

Untuk memakai perkakasnya sendiri, perkakas ini perlu dibangun dan diinstal terlebih dahulu. Berikut merupakan langkah-langkah yang perlu diambil untuk seluruh proses tersebut.

\begin{enumerate}
	\item Buka \textit{folder} ``build'' di dalam \textit{folder} perkakas.
	\item Buka \textit{terminal/command prompt} di dalam \textit{folder} tersebut.
	\item Sesuai dengan sistem operasi tempat perkakas akan digunakan, ketik dan jalankan perintah berikut di \textit{terminal}:
	
	\begin{itemize}
		\item Windows:
		\begin{verbatim}
cmake -DCMAKE_BUILD_TYPE:STRING=Release -DCMAKE_TOOLCHAIN_FILE="<direktori
file toolchain vcpkg>" -G "<compiler>" ../
		\end{verbatim}
	
		\item Linux:
		\begin{verbatim}
cmake ../
		\end{verbatim}
	\end{itemize}		

	Untuk apa yang harus menggantikan variabel \verb|<compiler>| dapat dilihat dengan perintah \verb|cmake --help|. Daftar \textit{compiler} yang didukung oleh cmake dapat dilihat di bagian \mbox{``Generators''}, dan pengguna tinggal menyesuaikan dengan \textit{compiler} yang telah diinstal sebelumnya.  Ada beberapa hal yang perlu dijelaskan/diperhatikan untuk langkah ini.
	
	\begin{itemize}
		\item Windows
			
		\begin{itemize}
			\item Opsi \verb|-DMAKE_TOOLCHAIN_FILE| merupakan metode pengintegrasian vcpkg untuk proyek CMake. Untuk direktori persisnya (dan sintaks lengkap dari opsi ini) dapat dilihat setelah langkah ``Using vcpkg with MSBuild/Visual Studio'' di halaman panduan instalasi vcpkg.\footnote{\href{https://vcpkg.io/en/getting-started.html}{https://vcpkg.io/en/getting-started.html}}
			\item Direkomendasikan untuk menginstal \textit{compiler} \textbf{MinGW}, karena \textit{compiler} ini sudah mengikutkan salah satu file \textit{header} yang dibutuhkan oleh perkakas ini. Apabila pengguna menggunakan \textit{compiler} ini, variabel \verb|<compiler>| harus diisi dengan ``\verb|MinGW Makefiles|''.
			\item \textbf{Jangan menggunakan \textit{compiler} Visual Studio}, karena \textit{compiler} ini tidak mengandung file \textit{header} C yang dibutuhkan di perkakas ini. Perlu diperhatikan juga bahwa compiler Visual Studio ini merupakan nilai \textit{default} dari \verb|<compiler>| untuk sistem operasi Windows, jadi pengguna juga tidak boleh menghilangkan opsi \verb|-G| tersebut begitu saja.
		\end{itemize}
			
		\item Linux \\
		Untuk sistem operasi berbasis Linux, tidak perlu mengatur \textit{compiler}, karena nilai \textit{default} dari variabel \verb|<compiler>| di sistem operasi berbasis Linux (\textbf{Unix Makefiles}) sudah ideal.
	\end{itemize}
	
	\item Lanjutkan dengan instalasi perkakas.
	
	\begin{itemize}
		\item Windows: \\
			Jalankan perintah berikut.
			\begin{verbatim}
cmake --build .
			\end{verbatim}
		\item Linux: \\
		Jalankan kedua perintah berikut.
			\begin{verbatim}
cmake --build .
cmake --install .
			\end{verbatim}
		Jika \textbf{GNU Make} terinstal di perangkat pengguna, maka kedua perintah ini dapat digantikan dengan perintah berikut.
		\begin{verbatim}
make install
		\end{verbatim}
		Jika terjadi \textit{error permission}, cukup tambahkan perintah \verb|sudo| di depan perintah yang ingin dijalankan.
	\end{itemize}
	
	\item File \textit{executable} akan terletak di dalam \textit{folder} ``build'', dan siap dijalankan.
\end{enumerate}

\subsection{Pengujian}
\label{sec:testing-experiments-testing}

Pengujian akan dilakukan untuk setiap fitur untuk memeriksa apakah semua fitur perkakas sudah berfungsi sebagaimanamestinya, serta semua kemungkinan \textit{error} yang ada sudah diatasi dengan benar. Perlu ditekankan bahwa pengujian berikut juga akan dilakukan dengan versi panjang dari opsi-opsi yang ada di dalam perintah (misal \verb|-h| diganti menjadi \verb|--help|). Akan tetapi, untuk alasan keringkasan dokumen, kecuali terjadi kegagalan, tes-tes tersebut tidak akan dicatat.

Tabel \ref{tab:testing-experiments-testing-overview} memaparkan jumlah tes yang akan dilakukan. Adapun penjelasan dari apa persisnya yang akan dites (\textit{scope}) untuk setiap objek tes akan dibahas langsung di tiap-tiap bagiannya.

\begin{table}[H]
    \centering
    \begin{tabular}{| c | c |}
    \hline
        \textbf{Objek tes} & \textbf{Jumlah tes} \\
    \hline
    \hline
        Sintaks dasar & 7 \\
    \hline
        Mode bantuan (\verb|--help|) & 3 \\
    \hline
        Mode \textit{searchplace} & 9 \\
    \hline
        Mode \textit{findroute} & 8 \\
    \hline
        Mode \textit{direct} & 10 \\
    \hline
	\end{tabular}
    \caption{Jumlah kategori dan tes yang dilakukan.}
    \label{tab:testing-experiments-testing-overview}
\end{table}

\subsubsection{Sintaks dasar}
\label{sec:testing-experiments-testing-basic}

Pengujian ini akan dilakukan untuk mengecek apakah perkakas akan merespon terhadap masukan yang sama sekali tidak sesuai dengan apa yang diharapkan oleh perkakas. Beberapa dari kasus-kasus berikut sudah meliputi kemungkinan-kemungkinan kesalahan masukan untuk fitur-fitur yang disediakan perkakas, jadi tes spesifik per fitur nantinya tidak akan mengikutkan pengujian sintaks lagi.

\begin{enumerate}
	\item Perintah tanpa opsi
	\begin{itemize}
		\item Perintah masukan:
		\begin{verbatim}
kiritool
		\end{verbatim}
		\item Keluaran yang diharapkan: \\
		Perkakas akan mengeluarkan pesan \textit{error} yang mengingatkan pengguna untuk memasukkan mode operasional perkakas.
		\item Keluaran perkakas: \\
		Perkakas mengeluarkan pesan \textit{error} yang mengingatkan pengguna untuk memasukkan mode operasional perkakas.
		\item Status tes: \textbf{Sukses}
	\end{itemize}
	
	\item Perintah dengan satu atau lebih opsi tidak valid
	\begin{itemize}
		\item Perintah masukan:
		\begin{verbatim}
kiritool -m searchplace -r bdo -q unpar -b id
		\end{verbatim}
		\item Keluaran yang diharapkan: \\
		Perkakas akan mengeluarkan pesan \textit{error} yang memberi tahu pengguna bahwa ada opsi yang tidak valid di dalam perintah masukan.
		\item Keluaran perkakas: \\
		Perkakas mengeluarkan pesan \textit{error} yang memberi tahu pengguna bahwa ada opsi yang tidak valid di dalam perintah masukan.
		\item Status tes: \textbf{Sukses}
	\end{itemize}
	
	\item Perintah tanpa argumen di akhir perintah
	\begin{itemize}
		\item Perintah masukan:
		\begin{verbatim}
kiritool -m searchplace -r bdo -q unpar -l
		\end{verbatim}
		\item Keluaran yang diharapkan: \\
		Perkakas akan mengeluarkan pesan \textit{error} mengenai adanya opsi yang kehilangan argumennya di dalam perintah masukan.
		\item Keluaran perkakas: \\
		Perkakas mengeluarkan pesan \textit{error} yang mengenai adanya opsi yang kehilangan argumennya di dalam perintah masukan.
		\item Status tes: \textbf{Sukses}
	\end{itemize}
	
	\item Perintah tanpa argumen di tengah perintah
	\begin{itemize}
		\item Perintah masukan:
		\begin{verbatim}
kiritool -m searchplace -r bdo -q -l id
		\end{verbatim}
		\item Keluaran yang diharapkan: \\
		Perkakas akan mengeluarkan pesan \textit{error} mengenai opsi yang kehilangan argumennya di dalam perintah masukan.
		\item Keluaran perkakas: \\
		Perkakas mengeluarkan pesan \textit{error} mengenai kelebihan argumen yang dimasukkan pengguna, serta mendaftarkan argumen apa saja yang berlebih.
		\item Status tes: \textbf{Gagal}
		\item Catatan tambahan: \\
		Akibat batasan teknis, \textit{library} getopt sendiri tidak bisa menangani kasus di mana opsi yang kehilangan argumennya berada di tengah perintah, karena getopt akan menginterpretasikan opsi selanjutnya sebagai argumen dari opsi yang kehilangan argumennya. Satu-satunya solusi yang mungkin adalah mengecek apakah argumen dimulai dengan karakter tanda hubung (`-'), tetapi solusi ini tidak dapat diimplementasikan, karena argumen dari beberapa opsi berpotensi untuk diawali dengan karakter tersebut.
	\end{itemize}
	
	\item Perintah dengan terlalu banyak argumen
	\begin{itemize}
		\item Perintah masukan:
		\begin{verbatim}
kiritool -m searchplace -r bdo -q unpar -l id en
		\end{verbatim}
		\item Keluaran yang diharapkan: \\
		Perkakas akan mengeluarkan pesan \textit{error} mengenai kelebihan argumen yang dimasukkan pengguna, serta mendaftarkan argumen apa saja yang berlebih.
		\item Keluaran perkakas: \\
		Perkakas mengeluarkan pesan \textit{error} mengenai kelebihan argumen yang dimasukkan pengguna, serta mendaftarkan argumen apa saja yang berlebih.
		\item Status tes: \textbf{Sukses}
	\end{itemize}
	
	\item Perintah dengan mode yang tidak valid
	\begin{itemize}
		\item Perintah masukan:
		\begin{verbatim}
kiritool -m help
		\end{verbatim}
		\item Keluaran yang diharapkan: \\
		Perkakas akan mengeluarkan pesan \textit{error} yang memberitahu pengguna bahwa mode yang dimasukkan tidak valid.
		\item Keluaran perkakas: \\
		Perkakas mengeluarkan pesan \textit{error} yang memberitahu pengguna bahwa mode yang dimasukkan tidak valid.
		\item Status tes: \textbf{Sukses}
	\end{itemize}
	
	\item Pengunaan banyak mode sekaligus
	\begin{itemize}
		\item Perintah masukan:
		\begin{verbatim}
kiritool -m searchplace -r bdo -q unpar -h -l en
		\end{verbatim}
		\item Keluaran yang diharapkan: \\
		Perkakas hanya akan merespon terhadap mode operasional pertama yang dimasukkan (beserta opsi-opsinya).
		\item Keluaran perkakas: \\
		Perkakas mengeluarkan hasil pencarian lokasi (keluaran opsi \verb|-m searchplace|) dan tidak mengindahkan panggilan bantuan penggunaan (opsi \verb|-h|).
		\item Status tes: \textbf{Sukses}
	\end{itemize}

\end{enumerate}

\subsubsection{Mode bantuan}
\label{sec:testing-experiments-testing-help}

Pengujian ini akan dilakukan untuk mengecek apakah fungsi-fungsi perkakas yang berhubungan dengan fitur bantuan penggunaan perkakas sudah berfungsi dengan baik.

\begin{enumerate}
	\item Panggilan bantuan normal
	\begin{itemize}
		\item Perintah masukan:
		\begin{verbatim}
kiritool -h
		\end{verbatim}
		\item Keluaran yang diharapkan: \\
		Perkakas akan mengeluarkan bantuan penggunaan perkakas.
		\item Keluaran perkakas: \\
		Perkakas mengeluarkan bantuan penggunaan perkakas.
		\item Status tes: \textbf{Sukses}
	\end{itemize}
	
	\item Panggilan bantuan dengan tambahan opsi valid yang tidak relevan
	\begin{itemize}
		\item Perintah masukan:
		\begin{verbatim}
kiritool -h -s unpar
		\end{verbatim}
		\item Keluaran yang diharapkan: \\
		Perkakas akan mengeluarkan bantuan penggunaan perkakas, tanpa memedulikan opsi serta argumen tambahan di dalam perintah.
		\item Keluaran perkakas: \\
		Perkakas mengeluarkan bantuan penggunaan perkakas seperti biasanya.
		\item Status tes: \textbf{Sukses}
	\end{itemize}
	
	\item Pemanggilan \textit{manual page} (khusus Linux)
	\begin{itemize}
		\item Perintah masukan:
		\begin{verbatim}
man kiritool
		\end{verbatim}
		\item Keluaran yang diharapkan: \\
		\textit{Terminal} akan menampilkan manual page dari perkakas yang sesuai.
		\item Keluaran perkakas: \\
		\textit{Terminal} menampilkan manual page yang sesuai.
		\item Status tes: \textbf{Sukses}
	\end{itemize}

\end{enumerate}

\subsubsection{Mode \textit{searchplace}}
\label{sec:testing-experiments-testing-searchplace}

Pengujian ini akan dilakukan untuk mengecek apakah fungsi-fungsi perkakas yang berhubungan dengan fitur pencarian lokasi sudah berfungsi dengan baik.

\begin{enumerate}
	\item Pencarian lokasi sukses dengan satu hasil
	\begin{itemize}
		\item Perintah masukan:
		\begin{verbatim}
kiritool -m searchplace -r bdo -q unpar -l id
		\end{verbatim}
		\item Keluaran yang diharapkan: \\
		Perkakas akan menampilkan nama dan koordinat \latlon lokasi, sesuai dengan bahasa yang diminta oleh pengguna.
		\item Keluaran perkakas: \\
		Perkakas menampilkan nama dan koordinat \latlon lokasi sesuai dengan bahasa yang diminta.
		\item Status tes: \textbf{Sukses}
	\end{itemize}
	
	\item Pencarian lokasi sukses dengan lebih dari satu hasil
	\begin{itemize}
		\item Perintah masukan:
		\begin{verbatim}
kiritool -m searchplace -r bdo -q ab -l id
		\end{verbatim}
		\item Keluaran yang diharapkan: \\
		Perkakas akan menampilkan nama dan koordinat \latlon semua kemungkinan lokasi, sesuai dengan bahasa yang diminta oleh pengguna.
		\item Keluaran perkakas: \\
		Perkakas menampilkan nama dan koordinat \latlon semua kemungkinan lokasi sesuai dengan bahasa yang diminta.
		\item Status tes: \textbf{Sukses}
	\end{itemize}
	
	\item Pencarian lokasi gagal
	\begin{itemize}
		\item Perintah masukan:
		\begin{verbatim}
kiritool -m searchplace -r bdo -q abasdasd -l id
		\end{verbatim}
		\item Keluaran yang diharapkan: \\
		Perkakas akan mengeluarkan pesan keluaran yang memberitahu pengguna bahwa lokasi tidak berhasil ditemukan.
		\item Keluaran perkakas: \\
		Perkakas mengeluarkan pesan keluaran yang memberitahu bahwa lokasi tidak berhasil ditemukan.
		\item Status tes: \textbf{Sukses}
	\end{itemize}
	
	\item Pencarian lokasi tanpa opsi region
	\begin{itemize}
		\item Perintah masukan:
		\begin{verbatim}
kiritool -m searchplace -q unpar -l id
		\end{verbatim}
		\item Keluaran yang diharapkan: \\
		Perkakas akan menampilkan pesan \textit{error} yang mengingatkan pengguna untuk memasukkan kode region.
		\item Keluaran perkakas: \\
		Perkakas menampilkan pesan \textit{error} yang mengingatkan pengguna untuk memasukkan kode region.
		\item Status tes: \textbf{Sukses}
	\end{itemize}
	
	\item Pencarian lokasi dengan region yang tidak valid
	\begin{itemize}
		\item Perintah masukan:
		\begin{verbatim}
kiritool -m searchplace -r bdg -q unpar -l id
		\end{verbatim}
		\item Keluaran yang diharapkan: \\
		Perkakas akan menampilkan pesan \textit{error} bahwa region yang dimasukkan tidak valid.
		\item Keluaran perkakas: \\
		Perkakas menampilkan pesan \textit{error} bahwa region yang dimasukkan tidak valid.
		\item Status tes: \textbf{Sukses}
	\end{itemize}
	
	\item Pencarian lokasi tanpa kata kunci pencarian
	\begin{itemize}
		\item Perintah masukan:
		\begin{verbatim}
kiritool -m searchplace -r bdo -l id
		\end{verbatim}
		\item Keluaran yang diharapkan: \\
		Perkakas akan menampilkan pesan \textit{error} yang mengingatkan pengguna untuk memasukkan kata kunci pencarian.
		\item Keluaran perkakas: \\
		Perkakas menampilkan pesan \textit{error} yang mengingatkan pengguna untuk memasukkan kata kunci pencarian.
		\item Status tes: \textbf{Sukses}
	\end{itemize}
	
	\item Pencarian lokasi dengan kata kunci pencarian yang tidak valid
	\begin{itemize}
		\item Perintah masukan:
		\begin{verbatim}
kiritool -m searchplace -r bdo -q -6.87520,107.60492 -l id
		\end{verbatim}
		\item Keluaran yang diharapkan: \\
		Perkakas akan menampilkan pesan bahwa keluaran API adalah sebuah \textit{error}.
		\item Keluaran perkakas: \\
		Perkakas menampilkan pesan bahwa keluaran API adalah sebuah \textit{error}.
		\item Status tes: \textbf{Sukses}
		\item Catatan tambahan: \\
		Perkakas tetap menerima kata kunci pencarian yang diawali dengan tanda hubung (`-'), jadi perkakas tidak bisa langsung mengembalikan \textit{error} ke pengguna apabila pengguna (misal dalam kasus ini) memasukkan koordinat \latlon sebagai argumen opsi \verb|-q|.
	\end{itemize}
	
	\item Pencarian lokasi tanpa pengaturan bahasa
	\begin{itemize}
		\item Perintah masukan:
		\begin{verbatim}
kiritool -m searchplace -r bdo -q unpar
		\end{verbatim}
		\item Keluaran yang diharapkan: \\
		Perkakas tetap berfungsi seperti biasa, dengan mengeluarkan keluaran dalam bahasa Indonesia.
		\item Keluaran perkakas: \\
		Perkakas mengeluarkan keluaran seperi biasa, dalam bahasa Indonesia.
		\item Status tes: \textbf{Sukses}
	\end{itemize}
	
	\item Pencarian lokasi dengan pengaturan bahasa yang tidak valid
	\begin{itemize}
		\item Perintah masukan:
		\begin{verbatim}
kiritool -m searchplace -r bdo -q unpar -l ch
		\end{verbatim}
		Perkakas akan mengeluarkan pesan \textit{error} yang memberitahu pengguna bahwa bahasa yang dimasukkan tidak valid.
		\item Keluaran perkakas: \\
		Perkakas mengeluarkan pesan \textit{error} yang sesuai dengan ekspektasi.
		\item Status tes: \textbf{Sukses}
	\end{itemize}
	
\end{enumerate}

\subsubsection{Mode \textit{findroute}}
\label{sec:testing-experiments-testing-findroute}

Pengujian ini akan dilakukan untuk mengecek apakah fungsi-fungsi perkakas yang berhubungan dengan fitur pencarian rute angkot sudah berfungsi dengan baik.

\begin{enumerate}
	\item Pencarian rute sukses dengan satu kemungkinan rute
	\begin{itemize}
		\item Perintah masukan:
		\begin{verbatim}
kiritool -m findroute -s -6.89350,107.60430 -f -6.87520,107.60492 -l id
		\end{verbatim}
		\item Keluaran yang diharapkan: \\
		Perkakas akan menampilkan estimasi durasi rute, serta langkah-langkah yang perlu ditempuh di dalam rutenya, sesuai dengan bahasa yang diminta oleh pengguna.
		\item Keluaran perkakas: \\
		Perkakas menampilkan estimasi durasi rute, serta langkah-langkah dalam rute, sesuai dengan bahasa yang diminta.
		\item Status tes: \textbf{Sukses}
	\end{itemize}
	
	\item Pencarian rute sukses dengan satu kemungkinan rute yang sangat jauh
	\begin{itemize}
		\item Perintah masukan:
		\begin{verbatim}
kiritool -m findroute -s -6.16935,106.78899 -f -6.87520,107.60492 -l id
		\end{verbatim}
		\item Keluaran yang diharapkan: \\
		Jauhnya rute tidak akan mempengaruhi kinerja dari perkakas.
		\item Keluaran perkakas: \\
		Perkakas tetap menampilkan estimasi durasi rute dan langkah-langkah dalam rute seperti seharusnya.
		\item Status tes: \textbf{Sukses}
	\end{itemize}
	
	\item Pencarian rute sukses dengan lebih dari satu kemungkinan rute
	\begin{itemize}
		\item Perintah masukan:
		\begin{verbatim}
kiritool -m findroute -s -6.89350,107.60430 -f -6.91527,107.59454 -l id
		\end{verbatim}
		\item Keluaran yang diharapkan: \\
		Perkakas akan menampilkan setiap rutenya, diikuti dengan estimasi waktunya serta langkah-langkah yang perlu ditempuh di dalam rutenya, sesuai dengan bahasa yang diminta oleh pengguna.
		\item Keluaran perkakas: \\
		Perkakas akan menampilkan setiap rutenya, diikuti dengan estimasi waktunya serta langkah-langkah yang perlu ditempuh di dalam rutenya, sesuai dengan bahasa yang diminta oleh pengguna.
		\item Status tes: \textbf{Sukses}
	\end{itemize}
	
	\item Pencarian rute gagal
	\begin{itemize}
		\item Perintah masukan:
		\begin{verbatim}
kiritool -m findroute -s -6.89350,107.60430 -f -6.88307,107.65529 -l id
		\end{verbatim}
		\item Keluaran yang diharapkan: \\
		Perkakas akan mengeluarkan pesan kepada pengguna bahwa rute perjalanan dengan angkot tidak berhasil ditemukan.
		\item Keluaran perkakas: \\
		Perkakas mengeluarkan pesan bahwa rute perjalanan dengan angkot tidak berhasil ditemukan.
		\item Status tes: \textbf{Sukses}
	\end{itemize}
	
	\item Pencarian rute tanpa masukan lokasi (2 kasus)
	\begin{itemize}
		\item Perintah masukan:
		\begin{verbatim}
kiritool -m findroute -s -6.89350,107.60430 -l id
kiritool -m findroute -f -6.87520,107.60492 -l id
		\end{verbatim}
		\item Keluaran yang diharapkan: \\
		Perkakas akan mengeluarkan pesan \textit{error} yang mengingatkan pengguna untuk memasukkan koordinat \latlon lokasi awal atau akhir, tergantung dari lokasi mana yang tidak dimasukkan.
		\item Keluaran perkakas: \\
		Perkakas  mengeluarkan pesan \textit{error} yang sesuai untuk kedua kasus pengujian.
		\item Status tes: \textbf{Sukses}
	\end{itemize}
	
	\item Pencarian rute dengan koordinat lokasi yang tidak valid
	\begin{itemize}
		\item Perintah masukan:
		\begin{verbatim}
kiritool -m findroute -s -6.89350,107.60430 -f -6.87520,107.6049a -l id
		\end{verbatim}
		\item Keluaran yang diharapkan: \\
		Perkakas akan mengeluarkan pesan bahwa keluaran API adalah sebuah \textit{error}.
		\item Keluaran perkakas: \\
		Perkakas  mengeluarkan pesan bahwa API mengembalikan sebuah \textit{error}.
		\item Status tes: \textbf{Sukses}
		\item Catatan tambahan: \\
		Akibat penggunaan variabel yang sama dengan variabel dalam fitur \verb|direct|, di mana di dalam fitur tersebut opsi \verb|-s| atau opsi \verb|| memiliki aturan masukan yang berbeda, maka fitur ini tidak akan langsung mendeteksi \textit{error} apabila pengguna memasukkan koordinat yang tidak valid.
	\end{itemize}
	
	\item Pencarian lokasi tanpa pengaturan bahasa
	\begin{itemize}
		\item Perintah masukan:
		\begin{verbatim}
kiritool -m findroute -s -6.89350,107.60430 -f -6.87520,107.60492
		\end{verbatim}
		\item Keluaran yang diharapkan: \\
		Perkakas tetap berfungsi seperti biasa, dengan mengeluarkan keluaran dalam bahasa Indonesia.
		\item Keluaran perkakas: \\
		Perkakas mengeluarkan keluaran seperi biasa, dalam bahasa Indonesia.
		\item Status tes: \textbf{Sukses}
	\end{itemize}
	
	\item Pencarian lokasi dengan pengaturan bahasa yang tidak valid
	\begin{itemize}
		\item Perintah masukan:
		\begin{verbatim}
kiritool -m findroute -s -6.89350,107.60430 -f -6.87520,107.60492 -l ch
		\end{verbatim}
		\item Keluaran yang diharapkan: \\
		Perkakas akan mengeluarkan pesan \textit{error} yang memberitahu pengguna bahwa bahasa yang dimasukkan tidak valid.
		\item Keluaran perkakas: \\
		Perkakas mengeluarkan pesan \textit{error} yang sesuai dengan ekspektasi.
		\item Status tes: \textbf{Sukses}
	\end{itemize}
	
\end{enumerate}

\subsubsection{Mode \textit{direct}}
\label{sec:testing-experiments-testing-directroute}

Pengujian ini akan dilakukan untuk mengecek apakah fungsi-fungsi perkakas yang berhubungan dengan fitur pencarian rute angkot langsung dari kata kunci lokasi awal dan akhir sudah berfungsi dengan baik.

\begin{enumerate}
	\item Pencarian rute langsung sukses dengan satu kemungkinan rute
	\begin{itemize}
		\item Perintah masukan:
		\begin{verbatim}
kiritool -m direct -S bdo -s ciwalk -F bdo -f unpar -l id
		\end{verbatim}
		\item Keluaran yang diharapkan: \\
		Perkakas akan menampilkan nama lokasi awal dan akhir, estimasi durasi rute, serta langkah-langkah yang perlu ditempuh di dalam rutenya, sesuai dengan bahasa yang diminta oleh pengguna.
		\item Keluaran perkakas: \\
		Perkakas menampilkan seluruh aspek yang disebutkan dalam keluarannya.
		\item Status tes: \textbf{Sukses}
	\end{itemize}
	
	\item Pencarian rute langsung sukses dengan satu kemungkinan rute yang sangat jauh
	\begin{itemize}
		\item Perintah masukan:
		\begin{verbatim}
kiritool -m direct -S cgk -s untar -F bdo -f unpar -l id
		\end{verbatim}
		\item Keluaran yang diharapkan: \\
		Jauhnya rute tidak akan berpengaruh ke keluaran perkakas.
		\item Keluaran perkakas: \\
		Perkakas menampilkan seluruh aspek dalam keluarannya seperti biasanya.
		\item Status tes: \textbf{Sukses}
	\end{itemize}
	
	\item Pencarian rute langsung sukses dengan lebih dari satu kemungkinan rute
	\begin{itemize}
		\item Perintah masukan:
		\begin{verbatim}
kiritool -m direct -S bdo -s ciwalk -F bdo -f paskal -l id
		\end{verbatim}
		\item Keluaran yang diharapkan: \\
		Perkakas akan menampilkan nama lokasi awal dan akhir, serta tiap-tiap kemungkinan rutenya, dimulai dari estimasi durasi, diikuti dengan langkah-langkah yang perlu ditempuh di dalamnya, sesuai dengan bahasa yang diminta oleh pengguna.
		\item Keluaran perkakas: \\
		Perkakas menampilkan seluruh aspek yang disebutkan dalam keluarannya.
		\item Status tes: \textbf{Sukses}
	\end{itemize}
	
	\item Pencarian rute langsung gagal
	\begin{itemize}
		\item Perintah masukan:
		\begin{verbatim}
kiritool -m direct -S bdo -s ciwalk -F bdo -f paku -l id
		\end{verbatim}
		\item Keluaran yang diharapkan: \\
		Perkakas akan menampilkan nama lokasi awal dan akhir, tetapi akan mengeluarkan pesan bahwa rute tidak berhasil ditemukan.
		\item Keluaran perkakas: \\
		Perkakas menampilkan keluaran sesuai dengan format yang diharapkan.
		\item Status tes: \textbf{Sukses}
	\end{itemize}
	
	\item Pencarian rute langsung tanpa region lokasi (2 kasus)
	\begin{itemize}
		\item Perintah masukan:
		\begin{verbatim}
kiritool -m direct -S bdo -s ciwalk -f unpar -l id
kiritool -m direct -s ciwalk -F bdo -f unpar -l id
		\end{verbatim}
		\item Keluaran yang diharapkan: \\
		Perkakas akan menampilkan nama lokasi hingga lokasi yang tidak diberikan regionnya, dan kemudian akan mengeluarkan pesan \textit{error} yang memberitahu pengguna bahwa ada kata kunci pencarian lokasi yang hilang.
		\item Keluaran perkakas: \\
		Perkakas menampilkan keluaran sesuai dengan format yang diharapkan.
		\item Status tes: \textbf{Sukses}
	\end{itemize}
	
	\item Pencarian rute langsung dengan region lokasi yang tidak valid (2 kasus)
	\begin{itemize}
		\item Perintah masukan:
		\begin{verbatim}
kiritool -m direct -S bdg -s ciwalk -F bdo -f unpar -l id
kiritool -m direct -S bdo -s ciwalk -F bdg -f unpar -l id
		\end{verbatim}
		\item Keluaran yang diharapkan: \\
		Perkakas akan menampilkan nama lokasi hingga lokasi yang tidak diberikan regionnya, dan kemudian akan mengeluarkan pesan \textit{error} yang memberitahu pengguna bahwa ada kata kunci pencarian lokasi yang hilang.
		\item Keluaran perkakas: \\
		Perkakas menampilkan keluaran sesuai dengan format yang diharapkan.
		\item Status tes: \textbf{Sukses}
	\end{itemize}
	
	\item Pencarian rute langsung tanpa kata kunci pencarian lokasi (2 kasus)
	\begin{itemize}
		\item Perintah masukan:
		\begin{verbatim}
kiritool -m direct -S bdo -s ciwalk -F bdo -l id
kiritool -m direct -S bdo -F bdo -f unpar -l id
		\end{verbatim}
		\item Keluaran yang diharapkan: \\
		Perkakas akan menampilkan nama lokasi hingga lokasi yang tidak diberikan kata kunci pencariannya, dan kemudian akan mengeluarkan pesan \textit{error} yang memberitahu pengguna bahwa ada kata kunci pencarian lokasi yang hilang.
		\item Keluaran perkakas: \\
		Perkakas menampilkan keluaran sesuai dengan format yang diharapkan.
		\item Status tes: \textbf{Sukses}
	\end{itemize}
	
	\item Pencarian rute langsung dengan kata kunci pencarian lokasi yang tidak valid (2 kasus)
	\begin{itemize}
		\item Perintah masukan:
		\begin{verbatim}
kiritool -m direct -S bdo -s -6.89350,107.60430 -F bdo -f unpar -l id
kiritool -m direct -S bdo -s ciwalk -F bdo -f -6.87520,107.60492 -l id
		\end{verbatim}
		\item Keluaran yang diharapkan: \\
		Perkakas akan menampilkan nama lokasi hingga lokasi yang tidak diberikan kata kunci pencariannya, dan kemudian akan mengeluarkan pesan yang memberitahu pengguna bahwa API mengembalikan sebuah \textit{error}.
		\item Keluaran perkakas: \\
		Perkakas menampilkan keluaran sesuai dengan format yang diharapkan.
		\item Status tes: \textbf{Sukses}
	\end{itemize}
	
	\item Pencarian rute langsung sukses tanpa pengaturan bahasa
	\begin{itemize}
		\item Perintah masukan:
		\begin{verbatim}
kiritool -m direct -S bdo -s ciwalk -F bdo -f unpar
		\end{verbatim}
		\item Keluaran yang diharapkan: \\
		Perkakas tetap berfungsi seperti biasa, dengan mengeluarkan keluaran dalam bahasa Indonesia.
		\item Keluaran perkakas: \\
		Perkakas mengeluarkan keluaran seperti biasa, dalam bahasa Indonesia.
		\item Status tes: \textbf{Sukses}
	\end{itemize}
	
	\item Pencarian rute langsung sukses dengan pengaturan bahasa yang tidak valid
	\begin{itemize}
		\item Perintah masukan:
		\begin{verbatim}
kiritool -m direct -S bdo -s ciwalk -F bdo -f unpar -l ch
		\end{verbatim}
		\item Keluaran yang diharapkan: \\
		Perkakas akan mengeluarkan pesan \textit{error} yang memberitahu pengguna bahwa bahasa yang dimasukkan tidak valid.
		\item Keluaran perkakas: \\
		Perkakas mengeluarkan pesan \textit{error} yang sesuai dengan ekspektasi.
		\item Status tes: \textbf{Sukses}
	\end{itemize}
	
\end{enumerate}
