\chapter{Pengujian dan Eksperimen}
\label{chap:testing}

Bab ini akan membahas pengujian perkakas yang telah dibangun, serta skenario-skenario pengujian yang akan digunakan untuk pengujian perkakas.

\section{Lingkungan Pengujian}
\label{sec:testing-specs}

Bagian ini akan memaparkan spesifikasi dari perangkat yang digunakan untuk pengujian perkakas yang telah dibuat. 

\subsection{Lingkungan Perangkat Keras}
\label{sec:testing-specs-hardware}

Berikut merupakan spesifikasi perangkat keras yang digunakan dalam pembangunan perkakas ini:

\begin{itemize}
	\item \textit{Processor}: Intel\logoregistered\xspace Core\logotrademark\xspace i5-10300H @ 2.50 GHz
	\item RAM: 8 GB
	\item \textit{Hard disk}: SSD 512 GB (NVMe\logotrademark\xspace M.2)
	\item Perangkat keras pendukung: Keyboard
\end{itemize}

\subsection{Lingkungan Perangkat Lunak}
\label{sec:testing-specs-software}

Berikut merupakan spesifikasi perangkat lunak yang digunakan dalam pembangunan perkakas ini:

\begin{itemize}
	\item Windows:
	
	\begin{itemize}
		\item OS: Windows 10 Home Single Language (64-bit)
		\item \textit{Compiler}: MinGW (GNU\textemdash versi 12.1.0)
		\item \textit{Library}:
		
		\begin{itemize}
			\item curl (versi 7.83.1)
			\item cmake (versi 3.24.1)
		\end{itemize}
		
	\end{itemize}
	
	\item Linux:
	
	\begin{itemize}
		\item OS: Ubuntu Jammy (22.04)
		\item \textit{Compiler}: MinGW (GNU\textemdash versi 12.1.0)
		\item \textit{Library}:
		
		\begin{itemize}
			\item curl (versi 7.81.0)
			\item cmake (versi 3.22.1)
		\end{itemize}
		
	\end{itemize}
	
\end{itemize}

\section{Implementasi}
\label{sec:testing-implementation}

Di subbab ini akan dijelaskan berbagai hal terkait dengan implementasi, seperti implementasi kode perkakas. 
% Uncomment jika perkakas tidak memiliki .exe langsung!
% Karena perkakas ini bukan merupakan perangkat lunak biasa, dalam arti bahwa perkakas ini tidak memiliki sebuah file \textit{executable} yang dapat dijalankan langsung oleh pengguna, maka perlu dipaparkan pula bagaimana cara membangun dan menggunakan perkakas ini.

\subsection{Instalasi dan Pembangunan}
\label{sec:testing-implementation-usage}

\subsubsection{Instalasi}
\label{sec:testing-implementation-usage-instalation}

Proses instalasi dari perkakas ini memiliki garis besar yang sama - instal \textit{compiler} C, kemudian instal \textit{library yang diperlukan}. Karena banyaknya perbedaan dari detil yang perlu dilakukan untuk setiap sistem operasi, maka bagian ini akan dibagi dua, menjadi satu bagian per sistem operasi.

\begin{itemize}
	\item Windows \\
	Untuk sistem operasi Windows, pengguna perlu menginstal hal-hal berikut:
	
	\begin{itemize}
		\item vcpkg
		\item cURL
		\item CMake
	\end{itemize}
	
	Perlu diperhatikan bahwa cURL (di Windows) harus diinstal melalui vcpkg\textemdash cURL bawaan dari Windows tidak mengandung \textit{library-library development} sekunder yang dibutuhkan oleh perkakas ini. Di lain hal, instalasi cURL secara manual hanya memungkinkan perkakas cURL-nya sendiri untuk diakses dari mana saja (melalui variabel \textit{environment}), tetapi hal ini tidak berlaku untuk \textit{library development} sekundernya.
	
	\item Linux \\
	Untuk sistem operasi berbasis Linux, pengguna perlu menginstal hal-hal berikut:
	
	\begin{itemize}
		\item cURL
		\item CMake
		\item libcurl4-openssl-dev (\textit{library development} cURL)
		\item GNU Make
	\end{itemize}
	
\end{itemize}

Ingat bahwa perkakas ini juga menggunakan \textit{library} cJSON, tetapi untuk alasan kompatibilitas antar Windows dan Linux, \textit{library} ini langsung diikutkan di dalam perkakasnya sendiri, sehingga tidak perlu diinstal oleh pengguna lagi.

% Old version
\begin{comment}
Untuk dapat menjalankan perkakas ini, pengguna harus menginstal sebuah \textit{compiler} bahasa C, vcpkg (untuk Windows), serta dua buah \textit{library} umum yang telah didaftarkan di subbab \ref{sec:testing-specs-software}, yaitu:

\begin{itemize}
	\item cURL, dan
	\item cmake.
\end{itemize}
\noindent
Perlu diperhatikan bahwa \textit{library} cURL perlu diinstal melalui vcpkg, untuk menghindari masalah kompatibilitas akibat versi cURL yang tidak sesuai dengan yang dibutuhkan. Penggunaan vcpkg akan menghindari terjadinya masalah ini karena seluruh \textit{library} di vcpkg selalu diperbarui ke versi terbarunya.

Adapun perkakas ini juga membutuhkan adanya \textit{package} tambahan yang spesifik untuk Linux, yaitu:

\begin{itemize}
	\item libcurl4-openssl-dev (\textit{library development} cURL), dan
	\item GNU Make (opsional).
\end{itemize}

Perkakas ini juga menggunakan \textit{library} cJSON, tetapi untuk alasan kompatibilitas antar Windows dan Linux, \textit{library} ini langsung diikutkan di dalam perkakasnya sendiri, sehingga tidak perlu diinstal oleh pengguna lagi.
\end{comment}

\subsubsection{Pembangunan}
\label{sec:testing-implementation-usage-instalation}

Untuk memakai perkakasnya sendiri, perkakas ini perlu dibangun terlebih dahulu. Berikut merupakan langkah-langkah yang perlu diambil untuk proses tersebut.

\begin{enumerate}
	\item Buka \textit{folder} ``build'' di dalam \textit{folder} perkakas.
	\item Buka \textit{terminal/command prompt} di dalam \textit{folder} tersebut.
	\item Sesuai dengan sistem operasi tempat perkakas akan digunakan, ketik dan jalankan perintah berikut di \textit{terminal}:
	
	\begin{itemize}
		\item Windows:
		\begin{verbatim}
cmake -DCMAKE_BUILD_TYPE:STRING=Release -DCMAKE_TOOLCHAIN_FILE="<direktori
file toolchain vcpkg>" -G "<compiler>" ../
		\end{verbatim}
	
		\item Linux:
		\begin{verbatim}
cmake ../
		\end{verbatim}
	\end{itemize}		

	Untuk apa yang harus menggantikan variabel \verb|<compiler>| dapat dilihat dengan perintah \verb|cmake --help|. Daftar \textit{compiler} yang didukung oleh cmake dapat dilihat di bagian \mbox{``Generators''}, dan pengguna tinggal menyesuaikan dengan \textit{compiler} yang telah diinstal sebelumnya.  Ada beberapa hal yang perlu dijelaskan/diperhatikan untuk langkah ini.
	
	\begin{itemize}
		\item Windows
			
		\begin{itemize}
			\item Opsi \verb|-DMAKE_TOOLCHAIN_FILE| merupakan metode pengintegrasian vcpkg untuk proyek CMake. Untuk direktori persisnya (dan sintaks lengkap dari opsi ini) dapat dilihat setelah langkah ``Using vcpkg with MSBuild/Visual Studio'' di halaman panduan instalasi vcpkg.\footnote{\href{https://vcpkg.io/en/getting-started.html}{Getting started with vcpkg}}
			\item Direkomendasikan untuk menginstal \textit{compiler} \textbf{MinGW}, karena \textit{compiler} ini sudah mengikutkan salah satu file \textit{header} yang dibutuhkan oleh perkakas ini. Apabila pengguna menggunakan \textit{compiler} ini, variabel \verb|<compiler>| harus diisi dengan ``\verb|MinGW Makefiles|''.
			\item \textbf{Jangan menggunakan \textit{compiler} Visual Studio}, karena \textit{compiler} ini tidak mengandung file \textit{header} C yang dibutuhkan di perkakas ini. Perlu diperhatikan juga bahwa compiler Visual Studio ini merupakan nilai \textit{default} dari \verb|<compiler>| untuk sistem operasi Windows, jadi pengguna juga tidak boleh menghilangkan opsi \verb|-G| tersebut begitu saja.
		\end{itemize}
			
		\item Linux \\
		Untuk sistem operasi berbasis Linux, tidak perlu mengatur \textit{compiler}, karena nilai \textit{default} dari variabel \verb|<compiler>| di sistem operasi berbasis Linux (\textbf{Unix Makefiles}) sudah ideal.
	\end{itemize}
	
	\item Ketik dan jalankan perintah berikut:
	\begin{verbatim}
cmake --build .
	\end{verbatim}
	Bagi pengguna sistem operasi berbasis Linux, perintah di langkah ini dapat digantikan dengan perintah berikut:
	\begin{verbatim}
make
	\end{verbatim}
	Walaupun pengguna dianjurkan untuk menggunakan perintah biasa dari CMake, perintah alternatif ini akan menghasilkan hasil yang sama. Perlu diperhatikan bahwa jika pengguna ingin menginstal perkakas melalui, pengguna perlu menginstal \textbf{GNU Make} terlebih dahulu.
	
	\item Khusus sistem operasi berbasis \textbf{Linux}, ketik dan jalankan perintah berikut pula untuk menginstal file-file tambahan:
		\begin{verbatim}
cmake --install .
	\end{verbatim}
	\item File \textit{executable} akan terletak di dalam \textit{folder} ``build'', dan siap dijalankan.
\end{enumerate}

\subsection{Implementasi kode}
\label{sec:testing-implementation-code}

\subsection{Pengujian}
\label{sec:testing-experiments}