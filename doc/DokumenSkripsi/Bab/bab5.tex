\chapter{Pengujian dan Eksperimen}
\label{chap:testing}

Bab ini akan membahas pengujian perkakas yang telah dibangun, serta skenario-skenario pengujian yang akan digunakan untuk pengujian perkakas.

\section{Lingkungan Implementasi}
\label{sec:testing-specs}

Bagian ini akan memaparkan spesifikasi dari perangkat yang digunakan untuk pengujian perkakas yang telah dibuat. 

\subsection{Lingkungan Perangkat Keras}
\label{sec:testing-specs-hardware}

Berikut merupakan spesifikasi perangkat keras yang digunakan dalam pembangunan perkakas ini:

\begin{itemize}
	\item \textit{Processor}: Intel\logoregistered\xspace Core\logotrademark\xspace i5-10300H @ 2.50 GHz
	\item RAM: 8 GB
	\item \textit{Hard disk}: SSD 512 GB (NVMe\logotrademark\xspace M.2)
	\item Perangkat keras pendukung: Keyboard
\end{itemize}

\subsection{Lingkungan Perangkat Lunak}
\label{sec:testing-specs-software}

Berikut merupakan spesifikasi perangkat lunak yang digunakan dalam pembangunan perkakas ini:

\begin{itemize}
	\item OS: Windows 10 Home Single Language (64-bit)
	\item Editor: Visual Studio Code
	\item Bahasa pemrograman: C
	\item \textit{Compiler}: gcc (versi 12.1.0)
	\item \textit{Library}:
	
	\begin{itemize}
		\item libcurl:
		\item cmake:
	\end{itemize}
	
\end{itemize}

\section{Implementasi}
\label{sec:testing-implementation}

Di subbab ini akan dijelaskan berbagai hal terkait dengan implementasi, seperti implementasi kode perkakas. 
% Uncomment jika perkakas tidak memiliki .exe langsung!
% Karena perkakas ini bukan merupakan perangkat lunak biasa, dalam arti bahwa perkakas ini tidak memiliki sebuah file \textit{executable} yang dapat dijalankan langsung oleh pengguna, maka perlu dipaparkan bagaimana cara menggunakan perkakas ini.

\subsection{Instalasi}
\label{sec:testing-implementation-installation}

\subsection{Implementasi kode}
\label{sec:testing-implementation-code}

\subsection{Pengujian}
\label{sec:testing-experiments}