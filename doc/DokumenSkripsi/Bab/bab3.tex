\chapter{Analisis}
\label{chap:analysis}

\section{Analisis Aplikasi Sejenis}
\label{sec:analysis-similarapps}

Untuk pembuatan perangkat lunak dalam skripsi ini, ada empat buah perkakas \cl yang akan diamati sebagai aplikasi sejenis. Dua dari empat aplikasi pertama adalah \chromewebstorecli dan \textit{iTunes Search API}.

\subsection{\chromewebstorecli\footnote{\href{https://github.com/pandawing/node-chrome-web-store-item-property-cli}{https://github.com/pandawing/node-chrome-web-store-item-property-cli}}}
\label{sec:similarapps-chromewebstore}

Perkakas \cl ini merupakan ekstensi dari sebuah aplikasi lain yang memiliki fungsi yang sama, yaitu Chrome \textit{Web Store Item Property}.\footnote{\href{https://github.com/pandawing/node-chrome-web-store-item-property}{https://github.com/pandawing/node-chrome-web-store-item-property}} Perangkat lunak \chromewebstorecli ini merupakan perangkat lunak yang akan memanggil fungsi API untuk mendapatkan metadata dari sebuah ekstensi pada \textit{web store} peramban Google Chrome. Perbedaan dari perkakas ini dengan aplikasi dasarnya adalah bahwa perkakas ini dapat digunakan sebagai perkakas \textit{command line}, sedangkan aplikasi dasarnya hanya bisa digunakan dalam perangkat lunak lainnya sebagai pemanggil fungsi API.

\subsubsection{Penggunaan}
\label{sec:similarapps-chromewebstore-usage}

Perkakas ini dapat digunakan melalui \textit{command prompt} dengan cara mengetikkan perintah sebagai berikut.

\begin{verbatim}
                     chrome-web-store-item-property <identifier>
\end{verbatim}

Dengan \verb|identifier| berupa ID dari ekstensi yang diinginkan. Jadi, misalkan pengguna memasukkan \verb|gighmmpiobklfepjocnamgkkbiglidom| sebagai ID yang akan digunakan sebagai \textit{identifier}, maka perkakas ini akan mengembalikan metadata dari ekstensi ``AdBlock'' sebagai keluarannya. Contoh penggunaan perkakas ini dapat dilihat di gambar \ref{fig:similarapps-chromewebstorecli}.

\begin{figure}[ht]
    \centering
    \includegraphics[width=0.75\linewidth]{chromewebstorecli}
    \caption[Contoh penggunaan perkakas \chromewebstorecli]{Contoh penggunaan perkakas \chromewebstorecli.}
    \label{fig:similarapps-chromewebstorecli}
\end{figure}

Sedangkan, keluaran dari perkakas ini merupakan sebuah objek JSON dengan properti-properti sebagai berikut.

\begin{itemize}
	\item \verb|name|\\
	Nama dari ekstensi yang dicari metadatanya.
	\item \verb|url|\\
	URL halaman web dari ekstensi yang dicari di \textit{web store} Google Chrome.
	\item \verb|image|\\
	Logo (dan ikon \textit{thumbnail}) dari ekstensi yang dicari metadatanya.
	\item \verb|version|\\
	Nomor versi dari ekstensi.
	\item \verb|price|\\
	Harga dari ekstensi. Jika ekstensi tidak memiliki harga yang perlu dibayarkan (gratis), properti ini akan bernilai \verb|0|.
	\item \verb|priceCurrency|\\
	Kode mata uang dari harga ekstensi. Jika ekstensi tidak memiliki harga yang perlu dibayarkan, properti ini akan berisi ``\verb|USD|``.
	\item \verb|interactionCount|\\
	Properti ini berisi interaksi-interaksi pengguna yang tercatat sebagai data di halaman \textit{web store} ekstensi. Pada saat pembuatan skripsi ini, properti ini hanya memiliki satu buah subproperti, yaitu \verb|userDownloads|, yang menandakan berapa kali ekstensi ini telah diunduh oleh pengguna di manapun.
	\item \verb|operatingSystems|\\
	Menandakan di peramban mana ekstensi versi ini dapat diinstal. Karena ekstensi-ekstensinya berada di \textit{web store} Chrome,
	\item \verb|ratingValue| (tidak digunakan lagi)\\
	Peringkat yang diberikan oleh para pengguna ekstensi ini. Nilai dari properti ini berupa skala desimal dari 0.00 sampai dengan 5.00. Di versi terbaru dari perkakas ini, properti ini tidak lagi tersedia dalam keluarannya.
	\item \verb|ratingCount| (tidak digunakan lagi)\\
	Jumlah pengguna yang telah menilai/memberi peringkat ke ekstensi ini. Di versi terbaru dari perkakas ini, properti ini tidak lagi tersedia dalam keluarannya.
	\item \verb|id|\\
	Properti ini mengandung ID dari ekstensi tersebut. Nilai dari properti ini akan sama dengan ID yang digunakan sebagai parameter masukan perkakas.
\end{itemize}
\vfill
\subsection{\itunesapi\footnote{\href{https://github.com/awcross/itunes-search-api}{https://github.com/awcross/itunes-search-api}}}
\label{sec:similarapps-itunesapi}

Perkakas \cl ini berfungsi untuk melakukan pencarian melalui API iTunes, sehingga seakan-akan pengguna langsung melakukan pencarian di iTunes sendiri. Hasil pencarian yang dilakukan termasuk judul lagu, nama artis, ataupun nama album, dan pengguna dapat memilih secara spesifik objek apa yang ingin dicari.

\subsubsection{Penggunaan}
\label{sec:similarapps-itunesapi-usage}

Perkakas ini dapat digunakan melalui \textit{command prompt} dengan cara mengetikkan perintah sebagai berikut.

\begin{verbatim}
                      itunes-search-api <input> [<options>]
\end{verbatim}

Dengan \verb|input| berupa nama dari objek yang dicari. Perkakas ini juga memiliki opsi yang masing-masing memiliki parameter tersendiri untuk mempersempit hasil pencarian. Adapun opsi-opsi tersebut dapat dilihat di daftar di bawah ini.

\begin{itemize}
	\item \verb|country|\\
	\textbf{Kemungkinan nilai:} Kode negara dua huruf\\
	Opsi ini menerima parameter berupa kode negara asal dari album atau artis yang dicari.
	\item \verb|entity|\\
	\textbf{Kemungkinan nilai:} \verb|song|, \verb|musicArtist|, atau \verb|album|\\
	Menandakan jenis objek/entitas yang ingin dicari. Opsi ini dapat bernilai \verb|song| untuk pencarian berbasis judul lagu, \verb|musicArtist| untuk pencarian nama artis, atau \verb|album| untuk pencarian nama album. Jika opsi ini tidak dipakai, objek apapun yang memiliki kemiripan dengan \verb|input| dalam salah satu dari ketiga properti ini akan muncul dalam hasil pencarian.
	\item \verb|limit|\\
	\textbf{Kemungkinan nilai:} Bilangan bulat positif\footnote{Opsi ini juga menerima bilangan bulat negatif, tetapi menggunakan sebuah bilangan bulat negatif akan menghilangkan pengaruh opsi ini terhadap hasil keluaran.}\\
	Jumlah hasil pencarian maksimal yang ingin ditampilkan dalam keluaran.
\end{itemize}
\vspace{\baselineskip}
Sedangkan, keluaran dari perkakas ini merupakan sebuah objek JSON yang memiliki dua properti utama, yaitu:

\begin{itemize}
	\item \verb|resultCount|\\
	Properti ini berisi bilangan bulat yang menandakan berapa buah objek yang terdapat dalam hasil pencarian.
	\item \verb|results|\\
	\textit{Array} yang berisi kumpulan objek yang terdapat di dalam hasil pencarian. Objek-objek ini akan dikembalikan berupa sebuah \textit{array} lain yang berisi seluruh properti dari masing-masing objek. Apa saja properti yang diikutkan dalam \textit{array} tersebut tergantung tipe dari objek dalam hasil pencarian.
\end{itemize}
\vspace{\baselineskip}
Adapun contoh penggunaan dan hasil keluaran perkakas ini dapat dilihat di gambar \ref{fig:similarapps-itunesapi}.

\begin{figure}[ht]
    \centering
    \includegraphics[width=0.75\linewidth]{itunesapi}
    \caption[Contoh penggunaan perkakas \itunesapi]{Contoh penggunaan perkakas \itunesapi. Gambar hanya memuat satu objek untuk menghemat tempat.}
    \label{fig:similarapps-itunesapi}
\end{figure}

Selain dua perkakas \cl tadi, ada dua perkakas lainnya yang bisa digunakan sebagai referensi, tetapi tidak dapat dieksplorasi, dikarenakan kedua aplikasi tersebut tidak berhasil dijalankan dengan sempurna. Adapun perkakas-perkakas tersebut adalah sebagai berikut.

\subsection{\ubercli\footnote{\href{https://github.com/jaebradley/uber-cli}{https://github.com/jaebradley/uber-cli}}}
\label{sec:similarapps-ubercli}

\ubercli merupakan sebuah perkakas \cl yang dapat digunakan untuk dua fungsi utama. Fungsi pertama dari perkakas ini adalah untuk mendapatkan estimasi untuk seberapa lama waktu yang diperlukan untuk servis taksi \textit{online} dari Uber untuk mencapai lokasi yang ingin dituju, sedangkan fungsi keduanya adalah untuk mengestimasi berapa harga yang harus dibayarkan untuk memakai servis tersebut. 
\newline\newline\noindent
Fungsi yang pertama dapat dilakukan memanggil perintah dengan format sebagai berikut.

\begin{verbatim}
                              uber time <alamat>
\end{verbatim}

\verb|uber| merupakan perintah dasar dari perkakas ini. \verb|time| merupakan parameter yang menandakan bahwa pengguna ingin menggunakan servis prediksi waktu dari perkakas ini. Selain itu, pengguna harus memasukkan alamat yang ingin dituju sebagai parameter akhir dari perintah yang akan digunakan sebagai masukan. Jika sintaksnya sudah benar, perintah tersebut akan bisa diproses oleh perkakas dengan cara mengirimkan pesan hasil konversi perintah tersebut ke API Uber. Setelah pemrosesan pesan tersebut berhasil, perkakas ini akan menampilkan sebuah keluaran dengan format yang dapat dilihat di gambar \ref{fig:similarapps-ubercli-time}. Perlu diperhatikan juga bahwa keluaran yang dihasilkan oleh perkakas ini akan meliputi seluruh jenis servis yang disediakan oleh Uber.

\begin{figure}[ht]
    \centering
    \includegraphics[width=0.425\linewidth]{ubercli-time}
    \caption[Contoh penggunaan perkakas \ubercli (\textit{time})]{Contoh penggunaan fitur prediksi waktu perjalanan untuk perkakas \ubercli.\protect\footnotemark}
    \label{fig:similarapps-ubercli-time}
\end{figure}
\footnotetext{Gambar diambil dari sumber yang sama dengan \textit{footnote} 5.}
\newpage\noindent % prevent code below from hanging off the very end of page
Sedangkan, untuk memanggil fungsi kedua dari perkakas ini, pengguna dapat dilakukan dengan memanggil perintah dengan format berikut.

\begin{verbatim}
                  uber price -s <alamat awal> -e <alamat akhir>
\end{verbatim}

Untuk sintaks ini, \verb|uber| memiliki fungsi yang sama dengan sintaks untuk fungsi pertama dari perkakas. \verb|price| merupakan penanda untuk perkakas bahwa pengguna ingin menggunakan servis untuk mengetahui perkiraan harga layanan Uber. Selanjutnya, perkakas akan meminta dua buah opsi beserta parameternya masing-masing. Pertama, opsi \verb|-s|, berarti \textit{start}, yang akan meminta sebuah parameter berupa lokasi yang ingin dipakai sebagai lokasi awal perkiraan harga layanan Uber. Sedangkan opsi \verb|-e|, berarti \textit{end}, akan meminta sebuah parameter berupa lokasi yang ingin dipakai sebagai lokasi akhir jasa perkiraan harga.
\newline\newline\noindent
Adapun keluaran dari fungsi kedua ini dapat dilihat di gambar \ref{fig:similarapps-ubercli-price}. Sama seperti keluaran untuk fungsi pertamanya, keluaran untuk fungsi kedua perkakas ini juga meliputi seluruh jasa yang disediakan oleh Uber.

\begin{figure}[ht]
    \centering
    \includegraphics[width=0.75\linewidth]{ubercli-price}
    \caption[Contoh penggunaan perkakas \ubercli (\textit{price})]{Contoh penggunaan fitur prediksi harga perjalanan untuk perkakas \ubercli.\protect\footnotemark}
    \label{fig:similarapps-ubercli-price}
\end{figure}
\footnotetext{Gambar diambil dari \href{https://github.com/jaebradley/uber-cli}{https://github.com/jaebradley/uber-cli}}

\subsubsection{Permasalahan}
\label{sec:similarapps-ubercli-problem}

Seperti telah dijelaskan di awal bab ini, perkakas ini tidak dapat digunakan. Kesimpulan yang diambil oleh penulis mengenai alasan perkakas ini tidak dapat dijalankan adalah dikarenakan oleh penggunaan API dan modul-modul yang telah usang (\textit{deprecated}). Kesimpulan ini diambil oleh penulis karena dua alasan utama. Pertama, pada awalnya, perkakas ini tidak dapat dijalankan karena API Google \textit{Maps} yang dipakai mengandung baris kode berikut didalamnya.

\begin{verbatim}
           exports.placesAutoCompleteSessionToken = require('uuid/v4');
\end{verbatim}

Kode ini merupakan kode yang dipakai untuk mengambil \textit{subpath} dari paket \verb|uuid|, tetapi penggunaannya sudah berubah untuk versi yang lebih barunya. Akan tetapi, setelah diganti baris tersebut ke penggunaan versi barunya pun, perkakas ini masih tetap tidak dapat dijalankan\textemdash sekarang perkakas ini justru mengembalikan kode error seperti dapat dilihat di bagian lampiran \ref{lamp:A}. Singkatnya (seperti tertera di akhir pesan error dalam lampiran tersebut), ini berarti perkakas mencoba untuk mengakses API Uber dengan menggunakan kredensial OAuth 2.0 yang berlaku untuk versi sebelumnya dari API tersebut. Permasalahan ini merupakan permasalahan yang juga ditemukan oleh beberapa pengguna lain, seperti tertera di halaman \textit{GitHub Issues} dari repositori ini.\footnote{\href{https://github.com/jaebradley/uber-cli/issues/87}{https://github.com/jaebradley/uber-cli/issues/87}} Oleh karena hal ini tidak lagi merupakan masalah kode perangkat lunak, maka perkakas ini dianggap tidak dapat dipakai.

\subsection{\googlemapscli\footnote{\href{https://github.com/yujinlim/google-maps-direction-cli}{https://github.com/yujinlim/google-maps-direction-cli}}}
\label{sec:similarapps-googlemapscli}

\googlemapscli merupakan sebuah perkakas \cl yang memiliki kegunaan yang mirip dengan KIRI, hanya saja perkakas ini tidak secara spesifik mengharuskan penggunaan angkot, atau transportasi umum lainnya. Singkatnya, perkakas ini memiliki fungsi seperti sebuah GPS. Untuk menggunakannya, pengguna harus memasukkan perintah dengan bentuk sebagai berikut.

\begin{verbatim}
                      direction <lokasi awal> <lokasi akhir>
\end{verbatim}

Setelah pengguna memasukkan perintah tersebut dengan benar, perkakas ini akan mengirim permintaan ke API Google \textit{Maps}, di mana jika prosesnya berhasil, keluarannya akan berupa langkah-langkah yang harus ditempuh untuk sampai ke lokasi akhir, beserta di jarak berapa langkah tersebut harus diambil, relatif terhadap langkah sebelumnya. Adapun penggunaan dari perkakas ini dapat dilihat di gambar \ref{fig:similarapps-googlemapscli}.

\begin{figure}[ht]
    \centering
    \includegraphics[width=0.66667\linewidth]{googlemapscli}
    \caption[Contoh penggunaan perkakas \googlemapscli]{Contoh penggunaan perkakas \googlemapscli.\protect\footnotemark}
    \label{fig:similarapps-googlemapscli}
\end{figure}
\footnotetext{Gambar diambil dari \href{https://github.com/yujinlim/google-maps-direction-cli}{https://github.com/yujinlim/google-maps-direction-cli}}

\subsubsection{Permasalahan}
\label{sec:similarapps-googlemapscli-problem}

Seperti tertulis di awal bab ini, perkakas ini juga tidak bisa digunakan. Alasan perkakas ini tidak dapat digunakan lagi-lagi merupakan masalah teknikal, yaitu diperbaruinya API Google \textit{Maps}. Lebih spesifiknya, semenjak 2018, \textit{Google} tidak lagi memperbolehkan penggunaan API Google \textit{Maps} tanpa kunci API, yang sayangnya tidak hanya mendasari perkakas ini, tetapi juga kunci API ini tidak bisa didapatkan tanpa membayarkan biaya tertentu. Oleh karena itu, perkakas ini dianggap tidak bisa lagi dijalankan.

\section{Analisis API KIRI\footnote{\href{https://github.com/projectkiri/Tirtayasa/wiki/KIRI-API-v2}{https://github.com/projectkiri/Tirtayasa/wiki/KIRI-API-v2}}}
\label{sec:analysis-kiri}

API KIRI dapat digunakan dengan mengirim sebuah permintaan GET ke alamat API\footnote{\href{https://projectkiri.id/api}{https://projectkiri.id/api}} dari KIRI. Permintaan tersebut harus memiliki parameter-parameter sebagai berikut.

\begin{itemize}
	\item \verb|version|\\
	Parameter ini merupakan tanda bagi API untuk menggunakan protokol versi 2.
	\item \verb|mode|\\
	Parameter ini merupakan mode dari servis/jasa API yang akan digunakan oleh pengguna.
	\item \verb|locale|\\
	Parameter ini mengatur bahasa apa yang akan digunakan dalam keluaran API.
	\item \verb|start|\\
	Parameter ini merupakan nilai \latlon dari titik awal perjalanan pengguna.
	\item \verb|finish|\\
	Parameter ini merupakan nilai \latlon dari titik akhir/tujuan perjalanan pengguna.
	\item \verb|presentation|\\
	Parameter ini hanya digunakan untuk fitur \textit{backwards compatibility}.
	\item \verb|apikey|\\
	Parameter ini berisi kunci API pribadi yang harus digenerasi terlebih dahulu sebelum API dapat digunakan.
\end{itemize}

Perlu diperhatikan bahwa salah satu dari parameter yang harus diikutkan dalam pesan tersebut merupakan parameter yang meminta kunci API. Kunci tersebut harus digenerasikan terlebih dahulu sebelum API KIRI dapat digunakan, melalui halaman \textit{API Keys} KIRI,\footnote{\href{https://projectkiri.id/dev/apikeys}{https://projectkiri.id/dev/apikeys}} yang dapat dilihat di gambar \ref{fig:kiri-apikeypage}.

\begin{figure}[ht]
    \centering
    \includegraphics[width=0.75\linewidth]{projectkiri-apikey}
    \caption[Halaman web \textit{API Keys} KIRI.]{Halaman web \textit{API Keys} KIRI.}
    \label{fig:kiri-apikeypage}
\end{figure}

Untuk mengakses halaman tersebut, pengguna harus membuat sebuah akun terlebih dahulu. Ketika akun sudah dibuat, maka pengguna baru akan dapat membuat kunci API yang dibutuhkan, sekaligus membuat filter \textit{domain}, yang membatasi di \textit{domain} mana saja kunci tersebut dapat digunakan, serta memberikan deskripsi untuk kunci API tersebut. Kunci ini kemudian dapat digunakan sebagai nilai dari parameter \verb|apikey| yang diperlukan dalam permintaan tadi.

Ketika seluruh parameter sudah valid dan berhasil diproses oleh API, sebuah keluaran berupa objek JSON akan dikirim kembali ke user. Adapun keluaran tersebut memiliki variabel-variabel sebagai berikut.

\begin{itemize}
	\item \verb|status|\\
	Variabel ini manandakan apakah permintaan berhasil diproses atau tidak. Jika permintaan berhasil diproses, variabel ini akan bernilai \verb|ok|, dan jika tidak, variabel ini akan bernilai \verb|error|.
	\item \verb|message|/\verb|routingresults|\\
	Isi utama dari respon API. Jika permintaan dari user tidak berhasil diproses, variabel ini akan bernama \verb|message| dan berisi pesan \textit{error} beserta alasan spesifik mengapa \textit{error} tersebut terjadi. Jika permintaan dari user berhasil diproses, nama variabel ini akan menjadi \verb|routingresults|, dan isi dari variabel ini adalah sebuah \textit{array} JSON yang merupakan data keluaran dari API KIRI. \textit{Array} JSON ini sendiri terbagi menjadi beberapa variabel lainnya, yang dapat dilihat di daftar di bawah ini.

\begin{itemize}
	\item \verb|steps|\\
	Variabel ini merupakan array yang tiap-tiap elemennya merepresentasikan satu buah langkah yang harus ditempuh oleh pengguna. Adapun tiap elemen ini sendiri berisi variabel-variabel berikut:
	
	\begin{itemize}
		\item Tipe transportasi\\
		Tipe sarana transportasi yang harus dipakai oleh pengguna. Jika pengguna harus berjalan kaki, variabel ini akan berisi \verb|walk|. Jika pengguna harus menaiki angkot, variabel ini akan berisi \verb|angkot|.
		\newpage % prevent widow
		\item Kode angkot\\
		Variabel ini menunjukkan angkot mana yang harus dinaiki oleh pengguna di langkah tersebut. Jika penggunaan angkot tidak dimungkinkan pada langkah ini (pengguna harus berjalan kaki), variabel ini akan berisi \verb|walk|.
		\item \textit{Array} \latlon lokasi\\
		\textit{Array} nilai-nilai desimal \latlon dari berbagai titik lokasi yang terdapat dalam rute.
		\item Deskripsi langkah\\
		Deskripsi langkah yang harus ditempuh, dalam bahasa natural. Bahasa yang digunakan tergantung parameter \verb|locale| yang diatur dalam masukan.
		\item URL untuk mendapatkan tiket kendaraan\\
		Tautan untuk mendapatkan tiket angkutan umum, jika diperlukan. Jika transportasi pada langkah tersebut tidak memerlukan tiket, variabel ini akan berisi \verb|null|.
		\item URL editor rute\\
		Tautan untuk meng-edit rute, jika situasinya memungkinkan. Jika tidak, variabel ini akan berisi \verb|null|.
	\end{itemize}
		
	\item \verb|traveltime|\\
	\textbf{Tipe:} string\\
	Variabel ini berisi estimasi jangka waktu yang diperlukan untuk menyelesaikan langkah tersebut.
\end{itemize}

\section{Analisis Fungsi dan \textit{Library} Bahasa C}
\label{sec:cmodules}

Di bagian ini akan dilakukan analisis terhadap seluruh fungsi bawaan, serta \textit{library-library} bahasa pemrograman C yang akan digunakan dalam pebuatan perkakas ini.

\subsection{getopt}
\label{sec:cmodules-getopt}

\verb|getopt| merupakan sebuah fungsi yang dapat mengautomasi pekerjaan-pekerjaan yang berhubungan dengan penerimaan opsi-opsi untuk \cl berbasis UNIX.\footnote{\href{https://www.gnu.org/software/libc/manual/html\_node/Getopt.html}{https://www.gnu.org/software/libc/manual/html\_node/Getopt.html}}
\newline\newline\noindent
Fungsi \verb|getopt| dapat dipanggil dengan format sebagai berikut.

\begin{verbatim}
                         getopt (argc, argv, <options>)
\end{verbatim}

Seluruh kode ini dapat dimasukkan ke suatu variabel berupa sebuah karakter yang merepresentasikan opsi yang ingin digunakan. \verb|argc| merupakan jumlah argumen yang terdapat dalam masukan, sedangkan argv merupakan sebuah \textit{array} yang berisi argumen-argumen tersebut.
\newline\newline
Selain itu, penggunaan \verb|getopt| juga memerlukan penggunaan variabel-variabel tertentu, yang dapat dilihat di daftar berikut.\footnote{\href{https://www.gnu.org/software/libc/manual/html\_node/Using-Getopt.html}{https://www.gnu.org/software/libc/manual/html\_node/Using-Getopt.html}}
\newpage % prevent widow
\begin{itemize}
	\item \verb|opterr|\\
	Isi dari variabel ini akan memberi signal ke perangkat lunak/perkakas yang menentukan apakah \verb|getopt| akan mengirim pesan ke \textit{error stream} atau tidak. Jika variabel ini bukan bernilai 0, maka pesan \textit{error} akan dikirim. Sebaliknya, jika variabel ini bernilai 0, \verb|getopt| tidak akan mengirim pesan \textit{error} apapun, tetapi tetap akan mengembalikan sebuah karakter tanda tanya (\verb|?|) sebagai tanda bahwa sebuah \textit{error} telah terjadi.
	\item \verb|optopt|\\
	Ketika \verb|getopt| menemukan sebuah karakter yang tidak didefinisikan dalam kumpulan opsi, atau sebuah opsi yang tidak disertai argumen yang diperlukan, karakter tersebut akan disimpan di variabel ini.
	\item \verb|optind|\\
	Variabel ini digunakan oleh \verb|getopt| sebagai indeks untuk \textit{array} \verb|argv|. Jika seluruh argumen sudah diproses, nilai variabel ini dapat digunakan untuk menentukan argumen mana yang merupakan arguman tambahan yang tidak terpakai. Nilai dari variabel ini dimulai dari 1.
	\item \verb|optarg|\\
	Jika opsi yang sedang diproses memerlukan argumen, variabel ini adalah tempat dimana argumen tersebut akan disimpan.
	\item \verb|<options>|\\
	Variabel ini berupa \textit{string} yang menandakan karakter-karakter apa saja yang menjadi opsi yang mungkin dalam perkakas tersebut, beserta tipenya. Jika karakter opsi:
	
	\begin{itemize}
		\item Diikuti dengan titik dua (\verb|:|), maka opsi tersebut memiliki argumen yang bersifat wajib.
		\item Diikuti dengan titik dua ganda (\verb|::|), maka opsi tersebut memiliki argumen yang bersifat opsional.
		\item Tidak diikuti apa-apa, maka opsi tersebut merupakan opsi tidak berarguman.
	\end{itemize}
	
\end{itemize}

\subsubsection{getopt-long}
\label{sec:cmodules-getopt-long}

Ada pula versi \verb|getopt| yang memungkinkan perangkat lunak untuk menerima dua jenis opsi\textemdash opsi versi pendek berupa sebuah karakter singular, seperti pada \verb|getopt| biasa, dan/atau opsi panjang bergaya GNU, berupa sebuah kata.

\verb|getopt-long| juga memiliki seluruh variabel-variabel yang dimiliki oleh \verb|getopt|, hanya saja \verb|getopt-long| memiliki sebuah variabel tambahan berupa struktur, yaitu \verb|long_options|. Variabel ini merupakan sebuah struktur berupa \textit{array} yang berisi beberapa \textit{array} lainnya, di mana \textit{array-array} lain in merupakan masing-masing opsi dari fungsi \verb|getopt-long| tersebut. Tiap-tiap \textit{array} tersebut memiliki variabel-variabel berikut:

\begin{itemize}
	\item \verb|name|\\
	Variabel ini merupakan nama panjang dari opsi.
	\item \verb|has_arg|\\
	Variabel ini merupakan penanda apakah opsi memerlukan argumen atau tidak. Nilai yang mungkin dalam variabel ini adalah \verb|no_argument|, \verb|required_argument|, atau \verb|optional_argument|.
	\item \verb|flag| \& \verb|val|\\ 
	Kedua variabel ini menandakan bagaimana sebuah opsi akan diberlakukan ketika diterima oleh \verb|getopt-long|. Variabel \verb|flag| dapat diisi dengan penunjuk ke suatu variabel lain yang akan diisi dengan isi dari variabel \verb|val| untuk menandakan bahwa \verb|getopt-long| telah berhasil memroses opsi tersebut. Di lain sisi, jika variabel ini berisi \textit{null pointer}, maka fungsi \verb|getopt-long| akan mengembalikan isi dari variabel \verb|val|.
\end{itemize}
\noindent
Struktur ini harus diakhiri dengan sebuah \textit{array} tambahan yang seluruh variabelnya bernilai 0.

\subsection{libcurl}
\label{sec:cmodules-libcurl}

libcurl merupakan sebuah \textit{library} yang digunakan untuk transfer berkas dengan berbagai protokol, seperti DICT, FILE, FTP, FTPS, GOPHER, HTTP, HTTPS, dan sebagainya. Tidak hanya itu, libcurl juga sangat portabel\textemdash dalam arti bahwa libcurl bisa digunakan (dan berjalan dengan tingkat fungsionalitas yang sama) di banyak sistem operasi, seperti Windows, Linux, MacOS, NetBSD, Solaris, Amiga, dan lain-lain.\footnote{\href{https://curl.se/libcurl/}{https://curl.se/libcurl/}}

\subsubsection{API\footnote{\href{https://curl.se/libcurl/c/libcurl.html}{https://curl.se/libcurl/c/libcurl.html}}}
\label{sec:cmodules-libcurl-CAPI}

libcurl juga memiliki sebuah API yang dapat digunakan untuk perangkat-perangkat lunak dalam bahasa C. API ini memiliki sebuah \textit{environment} global yang konstan. Ini berarti bahwa setiap perangkat lunak yang membutuhkan API ini harus menginisialisasi \textit{environment} ini (dengan memanggil \verb|curl_global_init()|), dan ketika perangkat lunak sudah selesai digunakan/dijalankan, fungsi \verb|curl_global_cleanup()| harus dipanggil pula.

Untuk mulai melakukan transfer berkas, diperlukan juga sebuah ``\textit{easy handle}'' yang dapat digunakan untuk satu orang individu/pihak, dengan memanggil fungsi \verb|curl_easy_init()|. Setelah itu, untuk mengatur opsi-opsi yang perlu diatur sesuai kebutuhan pengguna, seperti URL yang dituju, protokol yang ingin dipakai, koneksi ke port spesifik, dan sebagainya,\footnote{\href{https://curl.se/libcurl/c/curl\_easy\_setopt.html}{https://curl.se/libcurl/c/curl\_easy\_setopt.html}} pengguna harus mengaturnya dengan fungsi \verb|curl_easy_setopt()|. \textit{Handle} yang telah diatur ini dapat digunakan berulang kali dengan konfigurasi yang sama, sampai entah pengguna mengganti konfigurasi opsi-opsinya kembali, atau atau \textit{handle}-nya direset dengan pemanggilan fungsi \verb|curl_easy_reset()|.

\subsection{cJSON}
\label{sec:cmodules-cJSON}

\subsection{CMake}
\label{sec:cmodules-CMake}
	
\end{itemize}