\chapter{Kesimpulan}
\label{chap:conclusion}

Bab ini akan membahas kesimpulan yang telah ditarik selama seluruh proses pembuatan skripsi ini, mulai dari studi literatur, studi kasus (analisis perkakas sejenis), perancangan dan pembuatan perkakas, sampai dengan pengujian perkakas, serta saran-saran yang dapat diperhatikan untuk perkembangan perkakas atau riset lanjutan yang dapat dilakukan.

\section{Kesimpulan}
\label{chap:conclusion-endnotes}

Berikut merupakan kesimpulan yang diambil dari penulisan skripsi ini:

\begin{itemize}
	\item Telah berhasil dibuat sebuah perkakas bernama ``\textit{KIRI Tool}''. Perkakas ini merupakan perkakas \cl\xspace yang memiliki fitur-fitur serupa dengan KIRI, serta menggunakan API KIRI dalam implementasinya.
	\item Karena keluaran dari perkakas \textit{KIRI Tool} berupa teks biasa, maka integrasi dengan perkakas-perkakas \cl\xspace lainnya dapat dilakukan selama perkakas \cl\xspace lainnya menerima masukan berupa teks. Metode yang paling umum adalah menggunakan \verb|grep| (atau perkakas lain dengan fungsi yang sama) untuk mengekstraksi aspek tertentu dari keluaran perkakas \textit{KIRI Tool}.
\end{itemize}

\section{Saran}
\label{chap:conclusion-feedbacks}

Berikut adalah saran yang dapat dipertimbangkan untuk penelitian lanjutan dari skripsi ini:

\begin{itemize}
	\item Mengembangkan perangkat agar bisa digunakan di sistem operasi lain selain Windows dan Linux, misal macOS.
	\item Mencari tahu dan mengimplementasikan optimisasi apa saja yang dapat diberlakukan terhadap perkakas, seperti fungsi-fungsi apa dari \textit{library-library} yang ada yang dapat mengefektifkan penggunaan memori atau kecepatan pemrosesan perkakas, atau bahkan menggunakan \textit{library} yang lebih tepat.
	\item Keluaran API untuk fitur pencarian lokasi mengandung sebuah \textit{array} berisi beberapa koordinat \latlon, yang merupakan lokasi-lokasi yang dilewati di dalam rute. Di versi terakhir perkakas \textit{KIRI Tool} ketika skripsi ini dibuat, data ini tidak dipakai. Sebagai fitur tambahan, mungkin data ini dapat digunakan dalam keluaran perkakas untuk memberitahu persis lokasi-lokasi yang dilewati dalam rute yang didaftarkan.
\end{itemize}