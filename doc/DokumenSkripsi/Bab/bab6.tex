\chapter{Kesimpulan}
\label{chap:conclusion}

Bab ini akan membahas kesimpulan yang telah ditarik selama seluruh proses pembuatan skripsi ini, mulai dari studi literatur, studi kasus (analisis perkakas sejenis), perancangan dan pembuatan perkakas, sampai dengan pengujian perkakas, serta saran-saran yang dapat diperhatikan untuk perkembangan perkakas atau riset lanjutan yang dapat dilakukan.

\section{Kesimpulan}
\label{chap:conclusion-endnotes}

Berikut merupakan kesimpulan yang diambil dari penulisan skripsi ini:

\begin{itemize}
	\item Telah berhasil dibuat sebuah perkakas bernama ``\textit{KIRI Tool}'', yang merupakan perkakas \cl yang mengimplementasikan fitur-fitur API KIRI.
	\item Integrasi perkakas \textit{KIRI Tool} dengan perkakas-perkakas \cl lainnya dapat dilakukan dengan beberapa metode, dengan metode yang paling umum adalah menggunakan \verb|grep| untuk mengekstraksi aspek tertentu dari keluaran perkakas.
	\begin{comment}
	\item Perangkat-perangkat lunak berbasis \cl pada umumnya memiliki aturan-aturan standar yang harus diikuti, seperti memiliki tampilan murni dalam CLI (hanya teks), dioperasikan dengan perintah-perintah (yang juga berupa teks), dan memiliki halaman bantuan (entah berupa \textit{manual page} atau mode operasional simpel yang menampilkan cara penggunaan perkakas, seperti \verb|--help|).
	\item Perangkat-perangkat lunak berbasis \cl lebih sulit untuk dipelajari cara penggunaannya, tetapi pada umumnya lebih fleksibel dalam aspek-aspek fungsionalitasnya, dalam arti bahwa fitur-fitur yang ada tidak terikat pada satu fungsi yang pasti.
	\item KIRI merupakan sebuah aplikasi berbasis web yang dapat menunjukkan rute untuk pergi dari satu lokasi ke lokasi lain menggunakan angkutan kota. Dengan dibuatnya perkakas ini, KIRI akan dapat digunakan tanpa perlu membuka peramban web.
	\item Versi \cl dari perkakas-perkakas berbasis GUI yang ada dapat dianggap sebagai versi lebih ringan dari perkakas-perkakas tersebut, serta lebih cepat untuk digunakan.
	\end{comment}
\end{itemize}

\section{Saran}
\label{chap:conclusion-feedbacks}

Berikut adalah saran yang dapat dipertimbangkan untuk penelitian lanjutan dari skripsi ini:

\begin{itemize}
	\item Mengembangkan perangkat agar bisa digunakan di sistem operasi lain selain Windows dan Linux, misal macOS.
	\item Mencari tahu dan mengimplementasikan optimisasi apa saja yang dapat diberlakukan terhadap perkakas, seperti fungsi-fungsi apa dari \textit{library-library} yang ada yang dapat mengefektifkan penggunaan memori atau kecepatan pemrosesan perkakas, atau bahkan menggunakan \textit{library} yang lebih tepat.
	\item Menambahkan fitur baru kepada perkakas yang mungkin dapat memberitahu persis lokasi-lokasi yang dilewati dalam rute yang didaftarkan.
\end{itemize}