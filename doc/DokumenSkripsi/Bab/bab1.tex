%versi 2 (8-10-2016) 
\chapter{Pendahuluan}
\label{chap:intro}
   
\section{Latar Belakang}
\label{sec:label}
Project KIRI\footnote{\href{https://projectkiri.id}{https://projectkiri.id}} (akan disingkat sebagai KIRI dalam dokumen ini) adalah sebuah perangkat lunak berbasis web yang dibuat untuk \mbox{membantu} mengurangi efek dari kemacetan. KIRI mengurangi dampak kemacetan dengan membantu penggunanya, baik \mbox{masyarakat} maupun turis, dalam menggunakan salah satu sarana transportasi umum yang ada di Indonesia, yaitu angkutan kota (angkot). Cara KIRI \mbox{mempermudah} penggunaan angkot adalah dengan menunjukkan rute yang akan ditempuh, beserta langkah-langkah yang harus dilakukan oleh pengguna yang ingin berpergian dari satu titik ke titik lain, mulai dari seberapa jauh pengguna harus berjalan untuk menaiki angkot yang bersangkutan, di mana pengguna harus naik atau turun, seberapa jauh lagi pengguna harus berjalan sampai ke titik tujuan, dan seberapa lama estimasi waktu perjalanan yang akan ditempuh. Untuk kebutuhan pembuatan perangkat lunak yang memanfaatkan fitur dari KIRI, tersedia juga REST API KIRI yang dapat digunakan secara praktis. Adapun tampilan dari halaman web ini dapat dilihat di gambar \ref{fig:kiripage}. 

\begin{figure}[ht]
    \centering
    \includegraphics[width=0.74\linewidth]{projectkiri}
    \caption[Tampilan halaman web KIRI]{Tampilan halaman web KIRI, yang menunjukkan rute dari Cihampelas Walk ke Universitas Katolik Parahyangan.}
    \label{fig:kiripage}
\end{figure}

Sementara itu, dalam komputer, salah satu dari sekian banyak tipe perangkat lunak adalah \textit{command line}. \textit{\mbox{Command} line} adalah perangkat lunak paling sederhana, yang sudah ada sejak pertama kali komputer \mbox{diciptakan}. Perangkat lunak selalu memiliki tampilan berupa \cli (CLI), yang \mbox{tidak} \mbox{memiliki} tampilan apapun selain sebuah kotak yang memuat teks berupa perintah-perintah tertentu, \mbox{baik} perintah yang meminta masukan dari user untuk dilakukan oleh komputer, maupun perintah yang menampilkan keluaran dari komputer, tanpa ada tambahan gambar grafis apapun, seperti pada perangkat lunak dengan tampilan \textit{graphical user interface} (GUI). Singkatnya, tipe perangkat lunak ini bukan merupakan tipe yang paling indah untuk dilihat oleh para pengguna, tetapi jika digunakan dengan tepat, maka \mbox{jenis} \mbox{perangkat} lunak ini bisa menyuruh komputer untuk melakukan banyak sekali perintah-perintah dengan sangat cepat dan sangat efektif.

Pada skripsi ini akan dibuat sebuah perangkat lunak berupa perkakas \cl (\textit{command line tool}) yang dapat menjalankan fungsi-fungsi API dari KIRI. Perangkat lunak ini, seperti jenisnya, akan dibuat murni sebagai perkakas yang dijalankan dari \cl (terminal, cmd, PowerShell, dll.), dan tampilan akhir dari perangkat lunak akan berupa \cli tanpa tambahan \textit{graphical user interface}. Keseluruhan dari perangkat lunak ini akan dibangun dalam bahasa C.

\section{Rumusan Masalah}
\label{sec:rumusan}
\begin{enumerate}
	\item Bagaimana membangun perkakas \textit{command line} yang dapat mengimplementasikan fitur-fitur API KIRI dalam bahasa C?
	\item Bagaimana integrasi perkakas \textit{command line} KIRI dapat dilakukan dengan perkakas-perkakas \textit{command line} lainnya di Linux?
\end{enumerate}

\section{Tujuan}
\label{sec:tujuan}
Batasan masalah dalam skripsi ini adalah sebagai berikut:
\begin{enumerate}
	\item Membangun perkakas \textit{command line} yang dapat mengimplementasikan fitur-fitur API KIRI dalam bahasa C.
	\item Melakukan integrasi perkakas \textit{command line} KIRI dengan perkakas-perkakas \textit{command line} lainnya di Linux.
\end{enumerate}

\section{Batasan Masalah}
\label{sec:batasan}
Batasan masalah dalam skripsi ini adalah sebagai berikut:
\begin{enumerate}
	\item Perangkat lunak dibuat murni dalam bentuk CLI, tanpa tambahan GUI.
	\item Perangkat lunak yang dibuat tidak menyelesaikan batasan (lokasi tidak terdeteksi, rute tidak berhasil ditemukan, dsb.) yang sudah sejak awal terdapat dalam KIRI.
\end{enumerate}

\section{Metodologi}
\label{sec:metlit}
Metodologi yang akan diikuti dalam skripsi ini adalah sebagai berikut:
	\begin{enumerate}
		\item Melakukan studi dan eksplorasi terhadap fungsi-fungsi yang dimiliki perangkat lunak KIRI serta cara implementasi API KIRI.
		\item Melakukan analisis dan desain perangkat lunak yang akan dibangun.
	    \item Melakukan studi dan eksplorasi terhadap seluruh kemungkinan \textit{library-library} yang memenuhi spesifikasi dalam pembuatan perangkat lunak, berdasarkan analisis dan desain yang telah dilakukan sebelumnya.
		\item Melakukan analisis kebutuhan fitur-fitur perangkat lunak dan melakukan eksplorasi \textit{library} yang dapat digunakan dan memenuhi spesifikasi dalam pembuatan perangkat lunak.
		\item Membangun perangkat lunak berdasarkan rancangan yang sudah dibuat, dengan megimplementasikan seluruh modul dan \textit{library} yang telah ditentukan di tahap sebelumnya dalam bahasa C.
		\item Melakukan pengujian fungsional, perbaikan \textit{bug}, serta rekomendasi perbaikan berdasarkan pengujian yang sudah dilakukan.
		\item Menyelesaikan pembuatan dokumen-dokumen yang berkaitan, seperti dokumen skripsi dan dokumentasi perangkat lunak.
	\end{enumerate}

\section{Sistematika Pembahasan}
\label{sec:sispem}
Rencananya Bab 2 akan berisi petunjuk penggunaan template dan dasar-dasar \LaTeX.
Mungkin bab 3,4,5 dapt diisi oleh ketiga jurusan, misalnya peraturan dasar skripsi atau pedoman penulisan, tentu jika berkenan.
Bab 6 akan diisi dengan kesimpulan, bahwa membuat template ini ternyata sungguh menghabiskan banyak waktu.

\dtext{10}