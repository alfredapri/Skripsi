%versi 3 (22-07-2020)
\chapter{Landasan Teori}
\label{chap:teori}

% Reference test
%\cite{mueller:2007:windowscommandline}
%\cite{shottsjr:2019:linuxcommandline}
%\cite{matthew:2007:beginninglinuxprogramming}
%\cite{kerrisk:2010:linuxprogramminginterface}

\section{\textit{Command Line}}
\label{sec:commandline}
\textit{Command line} (atau \cli) dapat diartikan sebagai tampilan antarmuka/\textit{interface} yang memproses perintah dari pengguna dan meneruskannya langsung ke sistem operasi untuk dijalankan.\cite{shottsjr:2019:linuxcommandline} Seluruh sistem operasi komputer yang ada memiliki sebuah \cli dalam bentuk \textit{shell}, yang dapat digunakan oleh penggunanya untuk langsung mengakses fungsi atau servis yang disediakan oleh sistem operasi.\cite{mueller:2007:windowscommandline} 

%Sedangkan, perangkat lunak yang memproses \cli ini disebut sebagai \cl \textit{interpreter}\cite{mueller:2007:windowscommandline} atau \textit{shell}.\cite{shottsjr:2019:linuxcommandline}

\subsection{\textit{Command Line Interface} dan \textit{Graphical User Interface}}
\label{sec:commandline-comparison}

Ada beberapa dari tipe antarmuka yang masih banyak digunakan di zaman sekarang, tetapi dua tipe yang paling banyak muncul adalah \cli dan \gui. Perangkat lunak berbasis \cl sendiri bisa memiliki berbagai macam tampilan, tetapi semuanya selalu mengikuti satu bentuk antarmuka umum. Bentuk yang dimaksud adalah sebuah area/\textit{window} yang memuat teks berupa perintah-perintah dari user untuk dilakukan oleh komputer, beserta keluarannya yang juga berupa teks, seperti dapat dilihat pada gambar \ref{fig:commandline-cli}. Jenis perangkat lunak seperti ini disebut memiliki antarmuka jenis \cli (CLI). Adapun dekorasi visual yang dimiliki oleh jenis tampilan ini hanya berupa warna pada teks-teks yang ada, tanpa tambahan gambar apapun. Jika perangkat lunak tersebut memiliki dekorasi dan/atau tombol interaktif berupa gambar grafis, seperti pada gambar \ref{fig:commandline-gui}, maka perangkat lunak tersebut dikategorikan sebagai perangkat lunak berbasis \gui.

\begin{figure}[ht]
    \begin{subfigure}[b]{0.475\linewidth}
		\centering
		\includegraphics[height=4.7cm]{terminal-linux}
		\caption{Antarmuka perangkat lunak berbasis \cli.}
		\label{fig:commandline-cli}
	\end{subfigure}
	\hfill
    \begin{subfigure}[b]{0.475\linewidth}
		\centering
		\includegraphics[height=4.5cm]{gui}
		\caption{Antarmuka perangkat lunak berbasis \gui.}
		\label{fig:commandline-gui}
	\end{subfigure}
    \caption[Dua jenis tampilan perangkat lunak]{Contoh dua jenis antarmuka (\textit{interface)} perangkat lunak.}
	\label{fig:commandline-interfacetypes}
\end{figure}

Selain dari tampilannya sendiri, ada beberapa perbedaan utama lain antara perangkat-perangkat lunak berbasis \cli dengan perangkat lunak berbasis \gui. Adapun perbedaan-perbedaan utama dari kedua jenis antarmuka ini adalah sebagai berikut.
\begin{itemize}
	\item Pengunaan sumber daya sistem untuk menjalankan perangkat lunak berbasis \cli lebih rendah dibandingkan dengan perangkat lunak berbasis \gui.
	\item Bagi pengguna pemula (atau pengguna awam pada umumnya), perangkat lunak berbasis \cli akan lebih sulit digunakan karena tidak adanya bantuan apapun dalam bentuk visual, sehingga satu-satunya cara untuk tahu bagaimana cara menggunakan fitur-fiturnya adalah melalui dokumentasi perangkat lunak yang ada. Karena alasan yang sama pula, perangkat lunak berbasis \cli lebih sulit untuk dibiasakan penggunaannya.
	\item Automasi perintah yang bersifat berulan-ulang jauh lebih mudah dilakukan pada perangkat lunak berbasis \cli. Hal ini dikarenakan perangkat lunak berbasis \cli tidak hanya lebih mudah untuk dibuat \textit{script}-nya, tetapi juga lebih efisien untuk digunakan ketika ada banyak sekali perintah yang harus dilakukan pada suatu saat tertentu.\cite{mueller:2007:windowscommandline}
\end{itemize}

\subsection{\textit{Command Line} di Linux}
\label{sec:commandline-linux}

Linux merupakan sebuah sistem operasi yang sangat modular, jadi ada banyak sekali \textit{shell} yang dapat dijalankan dan digunakan di dalamnya. Walaupun begitu, ada satu \textit{shell} yang selalu datang ter-\textit{install} di dalam semua sistem operasi Linux, yaitu \textit{"bash"} (GNU \textit{Bourne Again Shell).}\cite{matthew:2007:beginninglinuxprogramming}

\subsubsection{Tampilan}
\label{sec:commandline-linux-appearance}

Ketika terminal di Linux dijalankan, akan keluar kotak dialog, beserta sebuah baris. Baris ini biasanya berisi sebuah teks dengan format sebagai berikut.

\begin{verbatim}
        <nama pengguna>@<nama perangkat>:<direktori yang sedang diproses>$
\end{verbatim}

Tanda dolar di ujung baris ini menandakan bahwa baris tersebut merupakan baris \textit{shell prompt}, yang merupakan waktu di mana terminal sudah siap menerima masukan dari pengguna untuk diproses. Perlu diingat bahwa di posisi tanda dolar ini, terkadang justru terdapat tanda pagar (\#). Tanda pagar di akhir baris \textit{shell prompt} menandakan bahwa terminal tersebut dijalankan dengan tingkat akses \textit{superuser}, yang berarti bahwa entah pengguna masuk ke sistem sebagai user \textit{root}, atau terminal memiliki izin tingkat \textit{superuser/administrator}.\cite{shottsjr:2019:linuxcommandline}

\begin{figure}[ht]
    \begin{subfigure}[b]{0.475\linewidth}
		\centering
		\includegraphics[width=\linewidth]{linux-shellprompt-normal}
		\caption{\textit{Shell prompt} terminal dengan tingkat izin \mbox{normal}.}
		\label{fig:shellprompt-linux-normal}
	\end{subfigure}
	\hfill
    \begin{subfigure}[b]{0.475\linewidth}
		\centering
		\includegraphics[width=\linewidth]{linux-shellprompt-superuser}
		\caption{\textit{Shell prompt} terminal dengan tingkat izin \textit{\mbox{superuser}}.}
		\label{fig:shellprompt-linux-superuser}
	\end{subfigure}
    \caption{Baris \textit{shell prompt} terminal di sistem operasi Linux.}
	\label{fig:commandline-shellprompt-linux}
\end{figure}

\subsubsection{Navigasi}
\label{sec:commandline-linux-nav}

Sama seperti di Windows, Linux menyimpan file-filenya di sebuah struktur direktori yang bersifat hierarkial. Hal ini berarti bahwa file-file tersebut disimpan dalam direktori-direktori (atau \textit{folder-folder}) yang tersusun seperti sebuah pohon. dalam arti bahwa satu \textit{folder} bisa jadi berada di dalam satu \textit{folder} lain, atau berisi beberapa \textit{folder} lainnya.\cite{shottsjr:2019:linuxcommandline}

Untuk navigasi, terminal Linux memiliki beberapa perintah utama. Adapun perintah-perintah tersebut adalah sebagai berikut.

\begin{itemize}
	\item \verb|pwd| \cite{shottsjr:2019:linuxcommandline}\\
	\verb|pwd| merupakan singkatan dari \textit{print working directory}, yang berarti bahwa perintah ini akan mengeluarkan \textit{working directory}, atau direktori tempat terminal sekarang sedang bekerja/berjalan, sebagai keluaran dari perintah tersebut. Ketika pengguna pertama kali menjalankan terminal, \textit{working directory}-nya selalu merupakan direktori \textit{home} dari perangkat.
	\item \verb|ls| \cite{shottsjr:2019:linuxcommandline}\\
	\verb|ls| digunakan untuk menghasilkan keluaran berupa isi dari folder yang dispesifikasi. Biasanya digunakan ketika pengguna sudah memasuki folder yang diinginkan, walaupun dengan perintah ini, pengguna bisa saja mengintip isi dari folder manapun di direktori manapun, dengan mengikutkan direktori yang diinginkan sebagai parameter dari perintah tersebut. Adapun Isi dari folder yang diikutkan sebagai parameter tidak hanya berupa folder lain, tetapi juga seluruh file-file yang ada, walaupun untuk file-file yang disembunyikan (nama file diawali dengan tanda titik), perlu ditambahkan opsi \verb|-a| agar file-file tersebut muncul pula dalam keluarannya.
	\item \verb|cd| \cite{shottsjr:2019:linuxcommandline}\\
	\verb|cd| adalah perintah yang berfungsi untuk mengganti \textit{working directory} dari terminal. Untuk melakukan hal tersebut, perintah yang perlu dimasukkan adalah sebagai berikut:
	
	\begin{verbatim}
	                     cd <direktori yang diinginkan>
	\end{verbatim}
	
	Direktori yang diinginkan dapat berupa direktori absolut, atau direktori relatif. Perbedaannya adalah direktori absolut selalu dimulai dari folder \textit{root}, mengikuti folder-folder apapun yang ada di antara \textit{root} sampai ke folder yang diinginkan.
	
	Sedangkan, direktori relatif selalu dimulai dari \textit{working directory}. Untuk penggunaan direktori relatif, diperlukan dua buah notasi spesial, yaitu titik (\verb|.|), yang merepresentasikan \textit{working directory} sekarang itu sendiri, dan dua titik (\verb|..|), yang merepresentasikan \textit{parent folder} dari \textit{working directory}.
\end{itemize}

\subsection{\textit{Command Line} di Windows}
\label{sec:commandline-windows}

% Start of block comment (old ver)
\begin{comment}

\subsection{Struktur Perintah \textit{Command Line}}
\label{sec:commandline-commands}

Perangkat lunak \cl memiliki struktur perintah sebagai berikut:\footnote{\href{https://www.gnu.org/savannah-checkouts/gnu/bash/manual/bash.html\#Shell-Commands}{Bash Reference Manual} dan \href{https://docs.microsoft.com/en-us/previous-versions/windows/it-pro/windows-xp/bb490954(v=technet.10)}{Microsoft Docs}}

\begin{quote}
	\centering
	\textit{prompt command parameter1 parameter2 ... parameterN}
\end{quote}

\begin{itemize}
	\item \textit{Prompt} \cite{shottsjr:2019:linuxcommandline} \\
	\textit{Prompt} merupakan keluaran teks dari \cli yang menandakan tempat/waktu di mana \textit{shell} sudah siap untuk menerima input dari pengguna. \textit{Prompt} umumnya berupa karakter tertentu, seperti tanda dolar (\$), tanda pagar (\#), atau tanda kurung sudut kiri (>).\footnote{\href{http://www.cpm.z80.de/manuals/SID86\_User\_Guide.txt}{http://www.cpm.z80.de/manuals/SID86\_User\_Guide.txt}}
	\item \textit{Command} \\
	Merupakan perintah yang diberikan oleh pengguna untuk dijalankan oleh perangkat lunak.
	\item \textit{Parameter} \\
	Parameter yang diberikan oleh pengguna sebagai iringan dari perintah yang diinginkan. Parameter biasanya berupa huruf, kata, atau angka, dan merupakan nilai variabel tambahan yang diperlukan untuk operasi yang akan dilakukan oleh perintah yang dimasukkan sebelumnya. Bersifat opsional, dalam arti bahwa beberapa perintah bisa jadi memerlukan \textit{N}-buah parameter, sedangkan beberapa lainnya tidak perlu parameter apapun.
\end{itemize}

\subsection{Tipe Antarmuka \textit{Command Line}}
\label{sec:commandline-type}

Berdasarkan sumber dari perangkat lunak \cl, antarmuka \cl dibagi menjadi dua, yaitu \cli dari OS dan aplikasi \cli.

\subsubsection{OS (\textit{Built-In})}
\label{sec:commandline-type-os}
Antarmuka \cl tipe ini merupakan antarmuka bawaan, dalam arti bahwa perangkat lunak ini sudah diikutkan dalam instalasi sistem operasi oleh pengguna. Dua contoh dari \cli tipe ini yang paling umum adalah \textit{cmd.exe} dalam sistem operasi Windows dan \textit{bash} dalam sistem operasi Linux.

\subsubsection{Aplikasi}
\label{sec:commandline-type-app}
Antarmuka \cl tipe ini merupakan antarmuka yang datang sebagai fitur dari aplikasi tertentu, baik dari pembuat yang sama dengan sistem operasi yang terdapat di perangkat milik pengguna (tapi bukan bawaan), atau dari pihak ketiga. Tipe ini biasanya banyak ditemukan dalam perangkat lunak zaman dahulu\textemdash spesifiknya, tahun-tahun ketika GUI masih belum merupakan teknologi yang mudah diimplementasikan. Walaupun begitu, pada zaman sekarang pun, masih banyak juga perangkat lunak yang memilih tipe antarmuka CLI dari pada GUI. Salah satu dari contoh aplikasi antarmuka tipe ini adalah Git.
\end{comment}
% End of block comment (old ver)

\section{KIRI}
\label{sec:kiri}

KIRI merupakan sebuah perangkat lunak berbasis web yang berfungsi untuk menyelesaikan (atau setidaknya mengurangi) dampak dari masalah-masalah yang dapat diselesaikan oleh transportasi umum/publik di Indonesia, seperti pemanasan global, kemacetan, atau peningkatan harga bensin. Selain itu, turis mancanegara juga memilih untuk menaiki transportasi umum, karena sarana transportasi tersebut tidak hanya jauh lebih murah, tetapi juga memberikan kesempatan kepada mereka untuk melihat seluk-beluk dari kota-kota yang mereka kunjungi. Walaupun begitu, masih banyak masyarakat lokal sendiri yang segan untuk menaiki transportasi publik, umumnya karena transportasi publik lebih rumit persiapannya dibandingkan dengan transportasi privat, seperti kendaraan pribadi.\footnote{\href{https://projectkiri.github.io/\#about-kiri}{https://projectkiri.github.io/\#about-kiri}}

Di halaman web KIRI, pengguna dapat memasukkan input berupa lokasi awal dan lokasi tujuan dan KIRI akan menghasilkan seluruh langkah yang harus ditempuh oleh pengguna untuk sampai ke lokasi tujuan, dengan menggunakan angkot. Keluaran ini sudah meliputi kode angkot mana saja yang harus dinaiki, dan juga seberapa jauh pengguna harus berjalan kaki untuk sampai ke lokasi rute angkot berikutnya.

\subsection{Tampilan}
\label{sec:kiri-appearance}

Pada saat pertama kali dibuka, hal pertama yang paling mencolok di halaman awal web KIRI adalah sebuah peta besar di sebelah kiri yang dapat diperbesar ataupun diperkecil. Sedangkan, bagian kanan dari halamannya terdiri atas beberapa bagian. Di bagian paling atas terdapat logo KIRI, beserta sepasang menu \textit{dropdown} - yang pertama merupakan pilihan kota tempat pengguna berada (untuk sekarang hanya tersedia pilihan kota Jakarta dan Bandung), dan yang kedua merupakan pilihan bahasa, entah bahasa Indonesia atau Inggris. Di bawahnya merupakan sepasang menu \textit{dropdown} yang merupakan tempat di mana pengguna memasukkan lokasi awal dan tujuan yang akan diproses oleh KIRI. Terakhir, di bawahnya ada sebuah bagian kosong, yang nantinya akan menjadi tempat di mana KIRI akan meletakkan keluaran dari prosesnya. Adapun tampilan awal dari halaman web ini dapat dilihat di gambar \ref{fig:kiri-base}.

\begin{figure}[ht]
    \centering
    \includegraphics[width=0.74\linewidth]{projectkiri-base}
    \caption[Tampilan awal halaman web KIRI]{Tampilan awal halaman web KIRI.}
    \label{fig:kiri-base}
\end{figure}

Ada dua area yang memiliki perbedaan yang signifikan ketika pengguna sudah memasukkan masukan dan menyuruh KIRI untuk memprosesnya. Bagian yang pertama adalah bagian peta, yang setelah pemrosesan masukan, akan memiliki garis-garis berwarna yang menandakan rute angkot maupun tujuan perjalanan kaki yang harus ditempuh oleh pengguna. Bagian kedua adalah bagian keluaran, yang tadinya kosong, sekarang akan berisi langkah-langkah yang harus ditempuh oleh penggunanya untuk pergi dari lokasi awal ke lokasi tujuan. Spesifiknya, perbedaan-perbedaan ini dapat dilihat di gambar \ref{fig:kiri-example}.

\begin{figure}[ht]
    \centering
    \includegraphics[width=0.74\linewidth]{projectkiri-example}
    \caption[Tampilan akhir halaman web KIRI]{Tampilan halaman web KIRI setelah pemrosesan masukan dari pengguna selesai.}
    \label{fig:kiri-example}
\end{figure}

\subsection{API}
\label{sec:kiri-api}

KIRI juga memiliki sebuah API yang dapat digunakan untuk keperluan pengembangan perangkat lunak. Seluruh permintaan (\textit{request}) yang dilakukan melalui API KIRI harus dilakukan sebagai permintaan tipe GET ke \href{https://projectkiri.id/api}{https://projectkiri.id/api}, beserta parameter-parameter yang dibutuhkan. 

Permintaan tersebut harus memiliki parameter-parameter seperti terlihat di bawah ini.

\begin{itemize}
	\item \verb|version|\\
	\textbf{Kemungkinan nilai:} \verb|2|\\
	Parameter ini merupakan tanda bagi API untuk menggunakan protokol versi 2.
	\item \verb|mode|\\
	\textbf{Kemungkinan nilai:} \verb|findroute|\\
	Parameter ini merupakan mode dari servis/jasa API yang akan digunakan oleh pengguna. Untuk mode \verb|findroute|, jasa yang akan digunakan adalah jasa pencarian rute dengan angkot.
	\item \verb|locale|\\
	\textbf{Kemungkinan nilai:} \verb|en| atau \verb|id|\\
	Parameter ini mengatur bahasa apa yang akan digunakan dalam keluaran API nantinya\textemdash\verb|en| berarti keluaran akan menggunakan bahasa Inggris, dan \verb|id| berarti keluaran akan menggunakan bahasa Indonesia.
	\item \verb|start|\\
	\textbf{Kemungkinan nilai:} \verb|lat|, \verb|lng|; dalam bentuk desimal\\
	Parameter ini merupakan nilai \latlon dari titik awal perjalanan pengguna.
	\item \verb|finish|\\
	\textbf{Kemungkinan nilai:} \verb|lat|, \verb|lng|; dalam bentuk desimal\\
	Parameter ini merupakan nilai \latlon dari titik akhir/tujuan perjalanan pengguna.
	\item \verb|presentation|\\
	\textbf{Kemungkinan nilai:} \verb|desktop|\\
	Parameter ini hanya digunakan untuk fitur \textit{backwards compatibility}.
	\item \verb|apikey|\\
	\textbf{Kemungkinan nilai:} angka heksadesimal 16-digit\\
	Parameter ini berisi kunci API pribadi yang harus digenerasi terlebih dahulu sebelum API dapat digunakan.
\end{itemize}

Sedangkan, respon yang diberikan oleh API berupa sebuah objek JSON yang selalu memiliki setidaknya dua variabel, yaitu:

\begin{itemize}
	\item \verb|status|\\
	\textbf{Kemungkinan nilai:} \verb|ok| atau \verb|error|\\
	Variabel ini manandakan apakah permintaan berhasil diproses atau tidak. Jika permintaan berhasil diproses, variabel ini akan bernilai \verb|ok|, dan jika tidak, variabel ini akan bernilai \verb|error|.
	\item \verb|message|\\
	Variabel ini bisa berisi dua macam objek. Jika permintaan dari user tidak berhasil diproses, atau dalam kata lain, terjadi sebuah \textit{error}, maka variabel ini akan berisi string yang merupakan pesan \textit{error} serta alasan spesifik mengapa \textit{error} tersebut terjadi. Di lain sisi, jika permintaan dari user berhasil diproses, variabel ini akan mengalami dua perubahan utama. Pertama, nama variabel ini akan berubah menjadi \verb|routingresults|, dan kedua, isi dari variabel ini akan menjadi sebuah \textit{array} JSON yang merupakan respon dari API KIRI berupa keluaran yang akan dilihat oleh pengguna. \textit{Array} JSON ini sendiri terbagi menjadi beberapa variabel lainnya, yang dapat dilihat di daftar di bawah ini.
	
	\begin{itemize}
		\item \verb|steps|\\
		\textbf{Tipe:} array\\
		Variabel ini merepresentasikan satu buah langkah yang harus ditempuh oleh pengguna. Adapun \textit{array} ini sendiri berisi variabel-variabel berikut:
		
		\begin{itemize}
			\item Tipe transportasi\\
			Tipe sarana transportasi yang harus dipakai oleh pengguna. Jika pengguna harus berjalan kaki, variabel ini akan berisi \verb|walk|. Jika pengguna harus menaiki angkot, variabel ini akan berisi \verb|angkot|.
			\item Kode angkot\\
			Variabel ini menunjukkan angkot mana yang harus dinaiki oleh pengguna di langkah tersebut. Jika penggunaan angkot tidak dimungkinkan pada langkah ini (pengguna harus berjalan kaki), variabel ini akan berisi \verb|walk|.
			\item Array \latlon lokasi\\
			\textit{Array} nilai-nilai desimal \latlon dari berbagai titik lokasi yang terdapat dalam rute.
			\item Deskripsi langkah\\
			Deskripsi langkah yang harus ditempuh, dalam bahasa natural. Bahasa apa yang digunakan untuk deskripsi ini tergantung parameter \verb|locale| yang diatur dalam masukan.
			\item URL untuk mendapatkan tiket kendaraan\\
			Tautan untuk mendapatkan tiket angkutan umum, jika diperlukan. Jika transportasi pada langkah tersebut tidak memerlukan tiket, variabel ini akan berisi \verb|null|.
			\item URL editor rute\\
			Tautan untuk meng-edit rute, jika situasinya memungkinkan. Jika tidak, variabel ini akan berisi \verb|null|.
		\end{itemize}
		
		\item \verb|traveltime|\\
		\textbf{Tipe:} string\\
		Variabel ini berisi estimasi jangka waktu yang diperlukan untuk menyelesaikan langkah tersebut.
	\end{itemize}
	
\end{itemize}

\section{Skripsi}
\label{sec:skripsi} 
 
Rencananya akan diisi dengan penjelasan umum mengenai buku skripsi.

\dtext{11-12} 

\section{\LaTeX}
\label{sec:latex}

Mengapa menggunakan \LaTeX{} untuk buku skripsi dan apa keunggulan/kerugiannya bagi mahasiswa dan pembuat template. 

\dtext{13-14}


\section{Template Skripsi FTIS UNPAR}
\label{sec:template}
 
Akan dipaparkan bagaimana menggunakan template ini, termasuk petunjuk singkat membuat referensi, gambar dan tabel.
Juga hal-hal lain yang belum terpikir sampai saat ini. 
 
\dtext{15-16}

\subsection{Tabel}  
Berikut adalah contoh pembuatan tabel. 
Penempatan tabel dan gambar secara umum diatur secara otomatis oleh \LaTeX{}, perhatikan contoh di file bab2.tex untuk melihat bagaimana cara memaksa tabel ditempatkan sesuai keinginan kita.

Perhatikan bawa berbeda dengan penempatan judul gambar gambar, keterangan tabel harus diletakkan di atas tabel!!
Lihat Tabel~\ref{tab:contoh1} berikut ini:

\begin{table}[H] %atau h saja untuk "kira kira di sini"
	\centering 
	\caption{Tabel contoh}
	\label{tab:contoh1}
	\begin{tabular}{cccc}
		\toprule
		& $v_{start}$ & $\mathcal{S}_{1}$ & $v_{end}$\\

		\midrule
		$\tau_{1}$ & 1 & 12& 20\\
		$\tau_{2}$ & 1 &  & 20\\
		$\tau_{3}$ & 1 & 9 & 20\\
		$\tau_{4}$ & 1 &  & 20\\

		\bottomrule
		
	\end{tabular} 
\end{table}
Tabel~\ref{tab:cthwarna1} dan Tabel~\ref{tab:cthwarna2} berikut ini adalah tabel dengan sel yang berwarna dan ada dua tabel yang bersebelahan. 
\begin{table}[H]
	\begin{minipage}[c]{0.49\linewidth}
		\centering
		\caption{Tabel bewarna(1)}
		\label{tab:cthwarna1}
		\begin{tabular}{ccccc}
			\toprule
			 & $v_{start}$ & $\mathcal{S}_{2}$ & $\mathcal{S}_{1}$ & $v_{end}$\\
			
			\midrule
			$\tau_{1}$ & 1 & 5 \cellcolor{green}& 12& 20\\
			$\tau_{2}$ & 1 & 8 \cellcolor{green}& & 20\\
			$\tau_{3}$ & 1 & 2/8/17 \cellcolor{green}& 9 & 20\\
			$\tau_{4}$ & 1 & \cellcolor{red}& & 20\\
			
			\bottomrule

		\end{tabular}
	\end{minipage}
	\begin{minipage}[c]{0.49\linewidth}
		
		\centering 
		\caption{Tabel bewarna(2)}
		\label{tab:cthwarna2}
		\begin{tabular}{ccccc}
			\toprule
			 & $v_{start}$ & $\mathcal{S}_{1}$ & $\mathcal{S}_{2}$ & $v_{end}$\\
			
			\midrule
			$\tau_{1}$ & 1 & 12& 5 \cellcolor{red} &20\\
			$\tau_{2}$ & 1 &  &  8 \cellcolor{green} &20\\
			$\tau_{3}$ & 1 & 9 & 2/8/17 \cellcolor{green} &20\\
			$\tau_{4}$ & 1 &   & \cellcolor{red} &20\\
			
			\bottomrule
		
		\end{tabular}
	\end{minipage}
\end{table}

 
\subsection{Kutipan}
\label{subs:kutipan} 
Berikut contoh kutipan dari berbagai sumber, untuk keterangan lebih lengkap, silahkan membaca file referensi.bib yang disediakan juga di template ini.
Contoh kutipan:
\begin{itemize}
	\item Buku:~\cite{berg:08:compgeom} 
	\item Bab dalam buku:~\cite{kreveld:04:GIS}
	\item Artikel dari Jurnal:~\cite{buchin:13:median}
	\item Artikel dari prosiding seminar/konferensi:~\cite{kreveld:11:median}
	\item Skripsi/Thesis/Disertasi:~\cite{lionov:02:animasi}~\cite{wiratma:10:following}~\cite{wiratma:22:later}
	\item Technical/Scientific Report:~\cite{kreveld:07:watertight}
	\item RFC (Request For Comments):~\cite{RFC1654}
	\item Technical Documentation/Technical Manual:~\cite{Z.500}~\cite{unicode:16:stdv9}~\cite{google:16:and7}
	\item Paten:~\cite{webb:12:comm}
	\item Tidak dipublikasikan:~\cite{wiratma:09:median}~\cite{lionov:11:cpoly}
	\item Laman web:~\cite{erickson:03:cgmodel}  
	\item Lain-lain:~\cite{agung:12:tango}
\end{itemize}    
  
\subsection{Gambar}

Pada hampir semua editor, penempatan gambar di dalam dokumen \LaTeX{} tidak dapat dilakukan melalui proses {\it drag and drop}.
Perhatikan contoh pada file bab2.tex untuk melihat bagaimana cara menempatkan gambar.
Beberapa hal yang harus diperhatikan pada saat menempatkan gambar:
\begin{itemize}
	\item Setiap gambar {\bf harus} diacu di dalam teks (gunakan {\it field} {\sc label})
	\item {\it Field} {\sc caption} digunakan untuk teks pengantar pada gambar. Terdapat dua bagian yaitu yang ada di antara tanda $[$ dan $]$ dan yang ada di antara tanda $\{$ dan $\}$. Yang pertama akan muncul di Daftar Gambar, sedangkan yang kedua akan muncul di teks pengantar gambar. Untuk skripsi ini, samakan isi keduanya.
	\item Jenis file yang dapat digunakan sebagai gambar cukup banyak, tetapi yang paling populer adalah tipe {\sc png} (lihat Gambar~\ref{fig:ularpng}), tipe {\sc jpg} (Gambar~\ref{fig:ularjpg}) dan tipe {\sc pdf} (Gambar~\ref{fig:ularpdf})
	\item Besarnya gambar dapat diatur dengan {\it field} {\sc scale}.
	\item Penempatan gambar diatur menggunakan {\it placement specifier} (di antara tanda  $[$ dan $]$ setelah deklarasi gambar.
	Yang umum digunakan adalah {\bf H} untuk menempatkan gambar {\bf sesuai} penempatannya di file .tex atau  {\bf h} yang berarti "kira-kira" di sini. \\
	Jika tidak menggunakan {\it placement specifier}, \LaTeX{} akan menempatkan gambar secara otomatis untuk menghindari bagian kosong pada dokumen anda.
	Walaupun cara ini sangat mudah, hindarkan terjadinya penempatan dua gambar secara berurutan. 	
	\begin{itemize}
		\item Gambar~\ref{fig:ularpng} ditempatkan di bagian atas halaman, walaupun penempatannya dilakukan setelah penulisan 3 paragraf setelah penjelasan ini.
		\item Gambar~\ref{fig:ularjpg} dengan skala 0.5 ditempatkan di antara dua buah paragraf. Perhatikan penulisannya di dalam file bab2.tex!
		\item Gambar~\ref{fig:ularpdf} ditempatkan menggunakan {\it specifier} {\bf h}.
	\end{itemize}
\end{itemize}
 
\dtext{17-18}
\begin{figure} 
	\centering  
	\includegraphics[scale=1]{ular-png}  
	\caption[Gambar {\it Serpentes} dalam format png]{Gambar {\it Serpentes} dalam format png} 
	\label{fig:ularpng} 
\end{figure} 

\dtext{19-20}
\begin{figure}[H]
	\centering  
	\includegraphics[scale=0.5]{ular-jpg}  
	\caption[Ular kecil]{Ular kecil} 
	\label{fig:ularjpg} 
\end{figure} 
\dtext{21-22}

\begin{figure}[ht] 
	\centering  
	\includegraphics[scale=1]{ular-pdf}  
	\caption[ {\it Serpentes} betina]{ {\it Serpentes} jantan} 
	\label{fig:ularpdf} 
\end{figure} 
 
\subsection{Kode Program}

Kode program dalam bahasa tertentu seringkali harus ditulis di dalam bab, bukan hanya dilampirkan di bagian Lampiran. 
Kode~\ref{kode:aneh} menampilkan penggunaan karakter-karakter yang umum digunakan dalam sebuah program yang ditulis dengan bahasa C.


\begin{lstlisting}[language=Java, caption=Kode untuk menampilkan karakter-karakter aneh, label=kode:aneh]
// This does not make algorithmic sense, 
// but it shows off significant programming characters.

#include<stdio.h>

void myFunction( int input, float* output ) {
	switch ( array[i] ) {
		case 1: // This is silly code
			if ( a >= 0 || b <= 3 && c != x )
				*output += 0.005 + 20050;
			char = 'g';
			b = 2^n + ~right_size - leftSize * MAX_SIZE;
			c = (--aaa + &daa) / (bbb++ - ccc % 2 );
			strcpy(a,"hello $@?"); 
	}
	count = ~mask | 0x00FF00AA;
}

// Fonts for Displaying Program Code in LATEX
// Adrian P. Robson, nepsweb.co.uk
// 8 October 2012
// http://nepsweb.co.uk/docs/progfonts.pdf

\end{lstlisting}

\subsection{Notasi}

Simbol-simbol (matematika) yang sering digunakan sepanjang penulisan skripsi, dapat dimasukkan ke dalam ``Daftar Notasi''. Daftar ini ada di halaman depan sebelum Bab~\ref{chap:intro}.
Cara memasukkan sebuah simbol ke dalam Daftar Notasi adalah menggunakan perintah \verb|\nomenclature|. Contoh:
\begin{center}
    \verb|\nomenclature[]{$A$}{luas kandang ular}|    
\end{center}
\nomenclature[]{$A$}{luas kandang ular}
\nomenclature[]{$n$}{banyaknya ular}
\nomenclature[]{$k$}{jumlah kepala per seekor ular\nomrefpage}
Argumen opsional digunakan untuk mengurutkan notasi. Silahkan lihat sendiri dokumentasi package \verb|nomencl|

